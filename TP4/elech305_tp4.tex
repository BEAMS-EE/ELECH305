\documentclass[11pt,a4paper]{article}
\usepackage[utf8]{inputenc}
\usepackage[T1]{fontenc}
\usepackage{amsthm} %numéroter les questions
\usepackage[frenchb]{babel}
\usepackage{datetime}
\usepackage{xspace} % typographie IN
\usepackage[hidelinks]{hyperref}% hyperliens
\usepackage[all]{hypcap} %lien pointe en haut des figures
\usepackage[french]{varioref} %voir x p y
\usepackage{fancyhdr}% en têtes
%\input cyracc.def
\usepackage{graphicx} %include pictures
\usepackage{pgfplots}

\usepackage{tikz}
\usetikzlibrary{calc}
\usetikzlibrary{babel}
\usepackage{circuitikz}
% \usepackage{gnuplottex}
\usepackage{float}
\usepackage{ifthen}

\usepackage[top=1.3 in, bottom=1.3 in, left=1.3 in, right=1.3 in]{geometry} % Yeah, that's bad to play with margins
\usepackage[]{pdfpages}
\usepackage[]{attachfile}

\usepackage{amsmath}
\usepackage{enumitem}
\setlist[enumerate]{label=\alph*)}% If you want only the x-th level to use this format, use '[enumerate,x]'
\usepackage{multirow}

\usepackage{askmaps} % https://github.com/parastuffs/askmaps_custom

\newcommand\version{v1.0.0}

%cyr
\newcommand\textcyr[1]{{\fontencoding{OT2}\fontfamily{wncyr}\selectfont #1}}

%Numero du TP :
\def \tpnumber {TP 4 }

\newboolean{corrige}
\ifx\correction\undefined
\setboolean{corrige}{false}% pas de corrigé
\else
\setboolean{corrige}{true}%corrigé
\fi

%\setboolean{corrige}{false}% pas de corrigé

\newboolean{annexes}
\setboolean{annexes}{true}%annexes
%\setboolean{annexes}{false}% pas de annexes

\definecolor{darkblue}{rgb}{0,0,0.5}

\newboolean{mos}
%\setboolean{mos}{true}%annexes
\setboolean{mos}{false}% pas de annexes

\usepackage{aeguill} %guillemets

%% fancy header & foot
\pagestyle{fancy}
\lhead{[ELEC-H-305] Circuits logiques et numériques\\ \tpnumber}
\rhead{\version\\ page \thepage}
\chead{\ifthenelse{\boolean{corrige}}{Corrigé}{}}
\cfoot{}
%%

\pdfinfo{
/Author (Quentin Delhaye, ULB -- BEAMS)
/Title (\tpnumber, ELEC-H-305)
/ModDate (D:\pdfdate)
}

\hypersetup{
pdftitle={\tpnumber [ELEC-H-305] Choucroute logique et numérique},
pdfauthor={Quentin Delhaye, ULB -- BEAMS  },
pdfsubject={}
}

\theoremstyle{definition}% questions pas en italique
\newtheorem{Q}{Question}[] % numéroter les questions [section] ou non []

\newcommand{\reponse}[1]{% pour intégrer une réponse : \reponse{texte} : sera inclus si \boolean{corrige}
	\ifthenelse {\boolean{corrige}} {\paragraph{Réponse :} \color{darkblue}   #1\color{black}} {}
 }

\newcommand{\addcontentslinenono}[4]{\addtocontents{#1}{\protect\contentsline{#2}{#3}{#4}{}}}

\date{}
\title{\vspace{-2cm}Circuits logiques et numériques [ELEC-H-305] \\  \tpnumber Diagrammes de Karnaugh \ifthenelse{\boolean{corrige}}{-- Corrigé}{} \\ \small{\version}}

\setlength{\parskip}{0.2cm plus2mm minus1mm} %espacement entre §
\setlength{\parindent}{0pt}

\begin{document}
% \pagestyle{empty}
\maketitle
\vspace*{-1cm}


\begin{Q}
Construire les K-Maps pour les fonctions logiques suivantes.
\begin{enumerate}
	\item $F(a,b) = \overline{a} \cdot (a+\overline{b}) \cdot (a+b)$
	\reponse{
		Afin d'établir la K-Map de la fonction logique, on peut d'abord remplir sa table de vérité~:
		\begin{center}
			\begin{tabular}{|c|c|c|} \hline
			$a$ & $b$ & $F$ \\ \hline
			0 & 0 & 0 \\ \hline
			0 & 1 & 0 \\ \hline
			1 & 0 & 0 \\ \hline
			1 & 1 & 0 \\ \hline
			\end{tabular}
		\end{center}

		Ensuite, chaque case de la K-Map correspond à une ligne de la table de vérité.

		\begin{center}
			\askmapii{}{a b}{}{0000}{}
		\end{center}


	}
	\item $F(a,b,c) = (\overline{a} + bc) \cdot (a+b)$
	\reponse{
		\begin{center}
			\begin{tabular}{|c|c|c|c|} \hline
			$a$ & $b$ & $c$ & $F$ \\ \hline
			0 & 0 & 0 & 0 \\ \hline
			0 & 0 & 1 & 0 \\ \hline
			0 & 1 & 0 & 1 \\ \hline
			0 & 1 & 1 & 1 \\ \hline
			1 & 0 & 0 & 0 \\ \hline
			1 & 0 & 1 & 0 \\ \hline
			1 & 1 & 0 & 0 \\ \hline
			1 & 1 & 1 & 1 \\ \hline
			\end{tabular}
		\end{center}

		\begin{center}
			\askmapiii{}{a b c}{}{00110001}{}
		\end{center}
	}
	\item $F(a,b,c,d) = \overline{(\overline{a} + bd)} \cdot (b+c)$
	\reponse{
		\begin{center}
			\begin{tabular}{|c|c|c|c|c|} \hline
			$a$ & $b$ & $c$ & $d$ & $F$ \\ \hline
			0 & 0 & 0 & 0 & 0 \\ \hline
			0 & 0 & 0 & 1 & 0 \\ \hline
			0 & 0 & 1 & 0 & 0 \\ \hline
			0 & 0 & 1 & 1 & 0 \\ \hline
			0 & 1 & 0 & 0 & 0 \\ \hline
			0 & 1 & 0 & 1 & 0 \\ \hline
			0 & 1 & 1 & 0 & 0 \\ \hline
			0 & 1 & 1 & 1 & 0 \\ \hline
			1 & 0 & 0 & 0 & 0 \\ \hline
			1 & 0 & 0 & 1 & 0 \\ \hline
			1 & 0 & 1 & 0 & 1 \\ \hline
			1 & 0 & 1 & 1 & 1 \\ \hline
			1 & 1 & 0 & 0 & 1 \\ \hline
			1 & 1 & 0 & 1 & 0 \\ \hline
			1 & 1 & 1 & 0 & 1 \\ \hline
			1 & 1 & 1 & 1 & 0 \\ \hline
			\end{tabular}
		\end{center}

		\begin{center}
			\askmapiv{}{a b c d}{}{0000000000111010}{}
		\end{center}
	}
\end{enumerate}

\end{Q}




\begin{Q}
	Simplifier $F(a,b)$ à l'aide des K-maps.
	\begin{enumerate}
		\item $F(a,b) = a + \overline{a}b + \overline{ab}$
		\reponse{
			On peut aussi remplir la K-Map en développant la fonction logique sous l'une de ses formes canoniques.
			Par exemple, $F(a,b) = a + \overline{a}b + \overline{ab} = a\overline{b} + ab + \overline{a}b + \overline{ab}$.
			Chaque minterme est une case de la K-Map mise à $1$.

			\begin{center}
				\askmapii{$F=1$}{a b}{}{1111}{}
			\end{center}
		}
		\item $F(a,b) = (a + b) \cdot (a + \overline{b})$
		\reponse{
			\begin{center}
				\askmapii{$F=a$}{a b}{}{0011}{% raise Z input
				\color{red}\put(1.1,0.1){\dashbox{0.1}(0.8,1.8){}}%
				}
			\end{center}
		}
		\item $F(a,b) = a + \overline{a}b$
		\reponse{
			\begin{center}
				\askmapii{$F=a+b$}{a b}{}{0111}{% raise Z input
				\color{red}\put(0.1,0.1){\dashbox{0.1}(1.8,0.8){}}%
				\color{green}\put(1.15,0.15){\dashbox{0.2}(0.7,1.7){}}%
				}
			\end{center}
		}
	\end{enumerate}
	\reponse{}%R
\end{Q}




\begin{Q}
	Simplifier $F(a,b,c)$ à l'aide des K-maps~:
	\begin{enumerate}
		\item $F(a,b,c) = \overline{a}c + \overline{abc}$
		\reponse{%TODO Mettre les termes dans la même couleur que les cadres dans les K-maps.
			\begin{center}
				\askmapiii{$F=\overline{ab}+\overline{a}c$}{b c a}{}{10100010}{% raise Z input
				\color{red}\put(0.1,1.1){\dashbox{0.1}(1.8,0.8){}}%
				\color{green}\put(1.15,1.15){\dashbox{0.2}(1.7,0.7){}}%
				}
			\end{center}
		}
		\item $F(a,b,c) = a\overline{b}c + \overline{a}b\overline{c} + \overline{a}bc + \overline{ab}c$
		\reponse{
			\begin{center}
				\askmapiii{$F=\overline{a}b+\overline{b}c$}{b c a}{}{00111010}{% raise Z input
				\color{red}\put(1.1,0.1){\dashbox{0.1}(0.8,1.8){}}%
				\color{green}\put(2.1,1.1){\dashbox{0.2}(1.8,0.8){}}%
				}
			\end{center}
		}
		\item $F(a,b,c) = ab\overline{c} + \overline{abc} + \overline{a}b\overline{c} + a\overline{bc}$
		\reponse{
			\begin{center}
				\askmapiii{$F=\overline{c}$}{b c a}{}{11001100}{% raise Z input
				\color{red}\put(0.1,0.1){\dashbox{0.1}(0.8,1.8){}}%
				\color{red}\put(3.1,0.1){\dashbox{0.1}(0.8,1.8){}}%
				}
			\end{center}
		}
	\end{enumerate}
	\reponse{}%
\end{Q}





\begin{Q}
	Simplifier $F(a,b,c,d)$ à l'aide des K-maps~:
	\begin{enumerate}
		\item $F(a,b,c,d) = abd + acd + bcd + ab + \overline{a}cd + \overline{ab}cd$
		\reponse{
			\begin{center}
				\askmapiv{$F=ab+cd$}{c d a b}{}{0001000100011111}{% raise Z input
				\color{red}\put(0.1,1.1){\dashbox{0.1}(3.8,0.8){}}%
				\color{green}\put(2.1,0.1){\dashbox{0.2}(0.8,3.8){}}%
				}
			\end{center}
		}
		\item $F(a,b,c,d) = \overline{abcd} + \overline{ac}d + \overline{a}b\overline{c} + abc + a\overline{b}c + abcd$
		\reponse{
			\begin{center}
				\askmapiv{$F=ac+\overline{ac}$}{c d a b}{}{1100110000110011}{% raise Z input
				\color{red}\put(0.1,2.1){\dashbox{0.1}(1.8,1.8){}}%
				\color{green}\put(2.1,0.1){\dashbox{0.2}(1.8,1.8){}}%
				}
			\end{center}
		}
		\item $F(a,b,c,d) = \overline{bcd} + \overline{a}c\overline{d} + \overline{ac}d + a\overline{d} + \overline{a}b\overline{d}$
		\reponse{
			\begin{center}
				\askmapiv{$F=\overline{d}+\overline{ac}$}{c d a b}{}{1111110011110000}{% raise Z input
				\color{red}\put(0.1,0.1){\dashbox{0.1}(0.8,3.8){}}%
				\color{green}\put(0.15,2.15){\dashbox{0.2}(1.7,1.7){}}%
				\color{red}\put(3.1,0.1){\dashbox{0.1}(0.8,3.8){}}%
				}
			\end{center}
		}
	\end{enumerate}
\end{Q}




\begin{Q}
	Simplifier $F(a,b,c,d,e)$ à l'aide des K-maps~:
	\begin{enumerate}
		\item $F(a,b,c,d,e) = a\overline{e} + b\overline{e} + a\overline{b}ce + a\overline{b}cde + a\overline{b}c\overline{e} + \overline{acde} + \overline{abe} + \overline{ab}ce$
		\reponse{
			\begin{center}
				\askmapv{$F = \overline{e} + \overline{b}c$}{a b c d e}{}{10101111101010101010111110101010}{%
				\color{red}\put(0.1,3.1){\dashbox{0.1}(7.8,0.8){}}%
				\color{red}\put(0.1,0.1){\dashbox{0.1}(7.8,0.8){}}%
				\color{green}\put(1.1,0.2){\dashbox{0.2}(0.8,3.6){}}%
				\color{green}\put(5.1,0.2){\dashbox{0.2}(0.8,3.6){}}%
				}
			\end{center}
		}
	\end{enumerate}
\end{Q}
\end{document}
