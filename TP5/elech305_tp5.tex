\documentclass[11pt,a4paper]{article}
\usepackage[utf8]{inputenc}
\usepackage[T1]{fontenc}
\usepackage{amsthm} %numéroter les questions
\usepackage[frenchb]{babel}
\usepackage{datetime}
\usepackage{xspace} % typographie IN
\usepackage[hidelinks]{hyperref}% hyperliens
\usepackage[all]{hypcap} %lien pointe en haut des figures
\usepackage[french]{varioref} %voir x p y
\usepackage{fancyhdr}% en têtes
%\input cyracc.def
\usepackage{graphicx} %include pictures
\usepackage{pgfplots}

\usepackage{tikz}
\usetikzlibrary{calc}
\usetikzlibrary{babel}
\usepackage{circuitikz}
% \usepackage{gnuplottex}
\usepackage{float}
\usepackage{ifthen}

\usepackage[top=1.3 in, bottom=1.3 in, left=1.3 in, right=1.3 in]{geometry} % Yeah, that's bad to play with margins
\usepackage[]{pdfpages}
\usepackage[]{attachfile}

\usepackage{amsmath}
\usepackage{enumitem}
\setlist[enumerate]{label=\alph*)}% If you want only the x-th level to use this format, use '[enumerate,x]'
\usepackage{multirow}

\usepackage{askmaps} % https://github.com/parastuffs/askmaps_custom

\usepackage{SevenSeg} %http://sciences-indus-cpge.papanicola.info/Afficheur-7-Segments-avec-PGF-TIKZ

\newcommand\version{v1.1.0}

%cyr
\newcommand\textcyr[1]{{\fontencoding{OT2}\fontfamily{wncyr}\selectfont #1}}

%Numero du TP :
\def \tpnumber {TP 5 }

\newboolean{corrige}
\ifx\correction\undefined
\setboolean{corrige}{false}% pas de corrigé
\else
\setboolean{corrige}{true}%corrigé
\fi

%\setboolean{corrige}{false}% pas de corrigé

\newboolean{annexes}
\setboolean{annexes}{true}%annexes
%\setboolean{annexes}{false}% pas de annexes

\definecolor{darkblue}{rgb}{0,0,0.5}

\newboolean{mos}
%\setboolean{mos}{true}%annexes
\setboolean{mos}{false}% pas de annexes

\usepackage{aeguill} %guillemets

%% fancy header & foot
\pagestyle{fancy}
\lhead{[ELEC-H-305] Circuits logiques et numériques\\ \tpnumber}
\rhead{\version\\ page \thepage}
\chead{\ifthenelse{\boolean{corrige}}{Corrigé}{}}
\cfoot{}
%%

\pdfinfo{
/Author (Quentin Delhaye, ULB -- BEAMS)
/Title (\tpnumber, ELEC-H-305)
/ModDate (D:\pdfdate)
}

\hypersetup{
pdftitle={\tpnumber [ELEC-H-305] Choucroute logique et numérique},
pdfauthor={Quentin Delhaye, ULB -- BEAMS  },
pdfsubject={}
}

\theoremstyle{definition}% questions pas en italique
\newtheorem{Q}{Question}[] % numéroter les questions [section] ou non []

\newcommand{\reponse}[1]{% pour intégrer une réponse : \reponse{texte} : sera inclus si \boolean{corrige}
	\ifthenelse {\boolean{corrige}} {\paragraph{Réponse :} \color{darkblue}   #1\color{black}} {}
 }

\newcommand{\addcontentslinenono}[4]{\addtocontents{#1}{\protect\contentsline{#2}{#3}{#4}{}}}

\date{}
\title{\vspace{-2cm}Circuits logiques et numériques [ELEC-H-305] \\  \tpnumber Diagrammes de Karnaugh et cahiers des charges \ifthenelse{\boolean{corrige}}{-- Corrigé}{} \\ \small{\version}}

\setlength{\parskip}{0.2cm plus2mm minus1mm} %espacement entre §
\setlength{\parindent}{0pt}

\begin{document}
% \pagestyle{empty}
\maketitle
\vspace*{-1cm}


\begin{Q}
Utilisez les K-Maps pour simplifier les fonctions suivantes\footnote{$\sum m$ indique une somme de mintermes, tandis que $\sum d$ indique une somme de « \textit{don't care} ».}.
\begin{enumerate}
	\item $F(a,b,c,d) = \sum m (2,4,8,9,10,11,13,15)$
	\reponse{
		Les équivalents binaires des mintermes renseignés sont les suivants~:

		\begin{center}
		\begin{tabular}{|c|c|} \hline
		Base 10 & Base 2 \\ \hline
		2 & 0010 \\ \hline
		4 & 0100 \\ \hline
		8 & 1000 \\ \hline
		9 & 1001 \\ \hline
		10 & 1010 \\ \hline
		11 & 1011 \\ \hline
		13 & 1101 \\ \hline
		15 & 1111 \\ \hline
		\end{tabular}\end{center}

		Ce sont donc ces cases qui sont mises à « 1 » dans la K-Map.

		\begin{center}
			\askmapiv{$F = a\overline{b} + ad + \overline{b}c\overline{d} + \overline{a}b\overline{cd}$}{a b c d}{}{0010100011110101}{% raise Z input
				\color{blue}\put(0.2,0.2){\dashbox{0.1}(0.6,0.6){}}%
				\color{blue}\put(3.2,0.2){\dashbox{0.1}(0.6,0.6){}}%
				\color{green}\put(2.1,1.1){\dashbox{0.2}(1.7,1.7){}}%
				\color{red}\put(3.1,0.1){\dashbox{0.3}(0.8,3.8){}}%
				\color{orange}\put(1.1,3.1){\dashbox{0.1}(0.8,0.8){}}
				}
		\end{center}
	}

	\item $F(a,b,c,d) = \sum m (2,3,4,10,12,13) + \sum d (11,14,15)$
	\reponse{Lors de la réduction de la fonction logique, les « don't care » ne doivent être couverts que s'ils permettent d'agrandir un n-cube.
	On ne peut avoir aucun n-cube ne contenant \textit{que} des « don't care ».

		\begin{center}
			\askmapiv{$F = \overline{b}c + ab + b\overline{cd}$}{a b c d}{}{00111000001-11--}{% raise Z input
				\color{blue}\put(0.1,0.1){\dashbox{0.1}(0.8,1.8){}}%
				\color{blue}\put(3.1,0.1){\dashbox{0.1}(0.8,1.8){}}%
				\color{green}\put(1.1,3.1){\dashbox{0.2}(1.7,0.7){}}%
				\color{red}\put(2.1,0.1){\dashbox{0.3}(0.8,3.8){}}%
			}
		\end{center}
	}

	\item $F(a,b,c,d) = \sum m (0,2,4,5,6,9,10) + \sum d (7, 11, 12, 13, 14, 15)$
	\reponse{~\\

		\begin{center}
			\askmapiv{$F = b + \overline{ad} + ad + ac$}{a b c d}{}{1010111-011-----}{% raise Z input
				\color{blue}\put(2.1,0.2){\dashbox{0.1}(1.8,1.7){}}%
				\color{orange}\put(2.2,1.1){\dashbox{0.05}(1.6,1.8){}}%
				\color{green}\put(0.1,3.1){\dashbox{0.2}(1.8,0.7){}}%
				\color{green}\put(0.1,0.2){\dashbox{0.2}(1.8,0.7){}}%
				\color{red}\put(1.1,0.1){\dashbox{0.3}(1.8,3.8){}}%
			}
		\end{center}
	}

	\item $F(a,b,c,d,e) = \sum m (0,1,2,4,11,15,17,20,21,31) + \sum d (5,6,16,18,22,27)$
	\reponse{~\\

		\begin{center}
			\askmapv{$F = \overline{bd} + \overline{be} + bde$}{a b c d e}{}{11101--000010001-1-011-0000-0001}{% raise Z input
				\color{blue}\put(0.1,0.1){\dashbox{0.1}(1.8,0.8){}}%
				\color{blue}\put(0.1,3.1){\dashbox{0.1}(1.8,0.8){}}%
				\color{blue}\put(4.1,0.1){\dashbox{0.1}(1.8,0.8){}}%
				\color{blue}\put(4.1,3.1){\dashbox{0.1}(1.8,0.8){}}%
				\color{green}\put(2.1,1.1){\dashbox{0.2}(1.8,0.8){}}%
				\color{green}\put(6.1,1.1){\dashbox{0.2}(1.8,0.8){}}%
				\color{red}\put(0.2,2.1){\dashbox{0.3}(1.6,1.7){}}%
				\color{red}\put(4.2,2.1){\dashbox{0.3}(1.6,1.7){}}%
			}
		\end{center}
	}

	\item $F(a,b,c,d,e) = \sum m (0,2,5,7,8,9,10,11,13,23,26,27,29) + \sum d (3,12,15,18,19,21,22,31)$
	\reponse{~\\

		\begin{center}
			\askmapv{$F = ce + \overline{a}b\overline{c} + \overline{c}d + \overline{ace}$}{a b c d e}{}{101-01011111-10-00--0--10011010-}{% raise Z input
				\color{blue}\put(3.1,0.1){\dashbox{0.2}(0.8,3.8){}}%
				\color{green}\put(0.2,0.2){\dashbox{0.1}(0.6,1.7){}}%
				\color{green}\put(3.2,0.2){\dashbox{0.1}(0.6,1.7){}}%
				\color{green}\put(4.2,0.2){\dashbox{0.1}(0.6,1.7){}}%
				\color{green}\put(7.2,0.2){\dashbox{0.1}(0.6,1.7){}}%
				\color{red}\put(0.3,3.3){\dashbox{0.05}(0.4,0.5){}}%
				\color{red}\put(3.3,3.3){\dashbox{0.05}(0.4,0.5){}}%
				\color{red}\put(0.3,0.3){\dashbox{0.05}(0.4,0.5){}}%
				\color{red}\put(3.3,0.3){\dashbox{0.05}(0.4,0.5){}}%
				\color{orange}\put(1.1,1.1){\dashbox{0.3}(1.8,1.8){}}%
				\color{orange}\put(5.1,1.1){\dashbox{0.3}(1.8,1.8){}}%
			}
		\end{center}
	}
\end{enumerate}
\end{Q}




\begin{Q}
On vous demande de construire un afficheur sept segments capable de reconnaître un chiffre exprimé en mode binaire.

Les différents chiffres sont représentés de la façon suivante~:

\begin{center}
	\begin{tikzpicture}
	\foreach \x in{0,...,4}
	{\coordinate (L\x) at({(\x-0)*2.5}, -3*1);
	\SSGNb{L\x}{\x}}
	\foreach \x in{5,...,9}
	{\coordinate (L\x) at({(\x-5)*2.5}, -3*2);
	\SSGNb{L\x}{\x}}
	\end{tikzpicture}
\end{center}

Établissez les K-Maps de chacun des segments.

\small{\textit{Sur combien de bits représentez-vous le chiffre à afficher ?}}

\small{\textit{Combien de fonctions logiques seront nécessaires ?}}

\small{\textit{Quand un segment est-il allumé ?}}

\reponse{
	Chaque segment de l'afficheur est contrôlé par une fonction logique différente, tel que représenté ci-dessous~:

	\begin{center}
		\begin{tikzpicture}
		\SSGLeg{}
		\end{tikzpicture}
	\end{center}

	Étant donné qu'il faut quatre bits pour représenter nos chiffres, chaque fonction logique aura quatre variables.

	Il convient ensuite d'identifier pour quels chiffres chacun des segments est allumé, c'est-à-dire pour quelles combinaisons de variables la fonction logique de chaque segment est égale à 1.

	Par exemple, ci-suit la table de vérité du segment supérieur régi par la fonction logique $a(A, B, C, D)$\footnote{$A$, $B$, $C$ et $D$ sont les quatre bits de la représentation binaire de notre chiffre.}.

	\begin{center}
		\begin{tabular}{|l|c|c|c|c|c|} \hline
		Décimal & $A$ & $B$ & $C$ & $D$ & $a$ \\ \hline
		0 & 0 & 0 & 0 & 0 & 1 \\ \hline
		1 & 0 & 0 & 0 & 1 & 0 \\ \hline
		2 & 0 & 0 & 1 & 0 & 1 \\ \hline
		3 & 0 & 0 & 1 & 1 & 1 \\ \hline
		4 & 0 & 1 & 0 & 0 & 0 \\ \hline
		5 & 0 & 1 & 0 & 1 & 1 \\ \hline
		6 & 0 & 1 & 1 & 0 & 1 \\ \hline
		7 & 0 & 1 & 1 & 1 & 1 \\ \hline
		8 & 1 & 0 & 0 & 0 & 1 \\ \hline
		9 & 1 & 0 & 0 & 1 & 1 \\ \hline
		\end{tabular}
	\end{center}

	Cette table de vérité peut être représentée sous la forme de la K-Map suivante~:

	\begin{center}
		\askmapiv{$a = A + C + \overline{BD} + BD$}{A B C D}{}{1011011111------}{% raise Z input
			\color{blue}\put(0.1,0.2){\dashbox{0.1}(3.7,1.7){}}%
			\color{orange}\put(0.3,0.3){\dashbox{0.05}(0.4,0.4){}}%
			\color{orange}\put(3.3,0.3){\dashbox{0.05}(0.4,0.4){}}%
			\color{orange}\put(0.3,3.3){\dashbox{0.05}(0.4,0.4){}}%
			\color{orange}\put(3.3,3.3){\dashbox{0.05}(0.4,0.4){}}%
			\color{green}\put(1.1,1.1){\dashbox{0.2}(1.8,1.8){}}%
			\color{red}\put(2.1,0.1){\dashbox{0.3}(1.8,3.8){}}%
		}
	\end{center}

	Toutes les valeurs supérieures à 9 ne nous intéressant pas, elles sont remplacées par des « don't care ».

	En procédant de manière similaire pour les six autres segments, on obtient les K-Maps qui suivent.

	\begin{center}
		\askmapiv{$b = \overline{B} + \overline{CD} + CD$}{A B C D}{}{1111100111------}{% raise Z input
			\color{blue}\put(0.1,0.1){\dashbox{0.1}(0.8,3.8){}}%
			\color{blue}\put(3.1,0.1){\dashbox{0.1}(0.8,3.8){}}%
			\color{green}\put(0.2,1.1){\dashbox{0.2}(3.6,0.8){}}%
			\color{red}\put(0.2,3.1){\dashbox{0.3}(3.6,0.7){}}%
		}
	\end{center}

	\begin{center}
		\askmapiv{$c = B + \overline{C} + D$}{A B C D}{}{1101111111------}{% raise Z input
			\color{blue}\put(1.1,0.1){\dashbox{0.1}(1.8,3.8){}}%
			\color{green}\put(0.1,1.1){\dashbox{0.2}(3.8,1.8){}}%
			\color{red}\put(0.2,2.1){\dashbox{0.3}(3.6,1.7){}}%
		}
	\end{center}

	\begin{center}
		\askmapiv{$d = A + \overline{BD} + \overline{B}C + C\overline{D} + B\overline{C}D$}{A B C D}{}{1011011011------}{% raise Z input
			\color{blue}\put(2.1,-0.1){\dashbox{0.1}(2,3.9){}}%
			\color{green}\put(0.1,0.1){\dashbox{0.2}(0.8,0.8){}}%
			\color{green}\put(3.1,0.1){\dashbox{0.2}(0.8,0.8){}}%
			\color{green}\put(3.1,3.1){\dashbox{0.2}(0.8,0.8){}}%
			\color{green}\put(0.1,3.1){\dashbox{0.2}(0.8,0.8){}}%
			\color{red}\put(0.2,0.2){\dashbox{0.3}(0.6,1.7){}}%
			\color{red}\put(3.2,0.2){\dashbox{0.3}(0.6,1.7){}}%
			\color{purple}\put(0.3,0.3){\dashbox{0.05}(3.4,0.5){}}%
			\color{yellow}\put(1.1,2.1){\dashbox{0.01}(1.8,0.8){}}%
		}
	\end{center}

	\begin{center}
		\askmapiv{$e = \overline{BD} + C\overline{D}$}{A B C D}{}{1010001010------}{% raise Z input
			\color{blue}\put(0.1,0.1){\dashbox{0.1}(0.8,0.8){}}%
			\color{blue}\put(3.1,0.1){\dashbox{0.1}(0.8,0.8){}}%
			\color{blue}\put(3.1,3.1){\dashbox{0.1}(0.8,0.8){}}%
			\color{blue}\put(0.1,3.1){\dashbox{0.1}(0.8,0.8){}}%
			\color{green}\put(0.2,0.2){\dashbox{0.2}(3.6,0.6){}}%
		}
	\end{center}

	\begin{center}
		\askmapiv{$f = A + \overline{CD} + B\overline{C} + B\overline{D}$}{A B C D}{}{1000111011------}{% raise Z input
			\color{blue}\put(2.1,0.1){\dashbox{0.3}(2,4){}}%
			\color{green}\put(0.1,3.1){\dashbox{0.2}(3.8,0.8){}}%
			\color{red}\put(1.1,2.1){\dashbox{0.1}(1.8,1.7){}}%
			\color{purple}\put(1.2,0.2){\dashbox{0.05}(1.6,0.5){}}%
			\color{purple}\put(1.2,3.2){\dashbox{0.05}(1.6,0.5){}}%
			% \color{yellow}\put(1.1,2.1){\dashbox{0.01}(1.8,0.8){}}%
		}
	\end{center}

	\begin{center}
		\askmapiv{$g = A + B\overline{C} + \overline{B}C + C\overline{D}$}{A B C D}{}{0011111011------}{% raise Z input
			\color{blue}\put(2.1,0.1){\dashbox{0.3}(1.8,3.8){}}%
			\color{green}\put(0.1,0.2){\dashbox{0.2}(3.7,0.7){}}%
			\color{red}\put(1.1,2.1){\dashbox{0.1}(1.8,1.7){}}%
			\color{purple}\put(0.2,0.3){\dashbox{0.05}(0.5,1.6){}}%
			\color{purple}\put(3.2,0.3){\dashbox{0.05}(0.5,1.6){}}%
			% \color{yellow}\put(1.1,2.1){\dashbox{0.01}(1.8,0.8){}}%
		}
	\end{center}
}
\end{Q}





\begin{Q}
Votre mission~: établir la fonction logique d'un système de vote majoritaire à cinq variables.
La fonction doit être égale à « 1 » si une majorité de variables sont égales à « 1 ».

\reponse{
	Commençons par remplir la table de vérité de notre fonction $F(a,b,c,d,e)$.
	\begin{center}
		\begin{tabular}{|c|c|c|c|c|c|} \hline
		$a$ & $b$ & $c$ & $d$ & $e$ & $F$ \\ \hline
		0 & 0 & 0 & 0 & 0 & 0 \\ \hline
		0 & 0 & 0 & 0 & 1 & 0 \\ \hline
		0 & 0 & 0 & 1 & 0 & 0 \\ \hline
		0 & 0 & 0 & 1 & 1 & 0 \\ \hline
		0 & 0 & 1 & 0 & 0 & 0 \\ \hline
		0 & 0 & 1 & 0 & 1 & 0 \\ \hline
		0 & 0 & 1 & 1 & 0 & 0 \\ \hline
		0 & 0 & 1 & 1 & 1 & 1 \\ \hline
		0 & 1 & 0 & 0 & 0 & 0 \\ \hline
		0 & 1 & 0 & 0 & 1 & 0 \\ \hline
		0 & 1 & 0 & 1 & 0 & 0 \\ \hline
		0 & 1 & 0 & 1 & 1 & 1 \\ \hline
		0 & 1 & 1 & 0 & 0 & 0 \\ \hline
		0 & 1 & 1 & 0 & 1 & 1 \\ \hline
		0 & 1 & 1 & 1 & 0 & 1 \\ \hline
		0 & 1 & 1 & 1 & 1 & 1 \\ \hline
		\end{tabular}
		\begin{tabular}{|c|c|c|c|c|c|} \hline
		$a$ & $b$ & $c$ & $d$ & $e$ & $F$ \\ \hline
		1 & 0 & 0 & 0 & 0 & 0 \\ \hline
		1 & 0 & 0 & 0 & 1 & 0 \\ \hline
		1 & 0 & 0 & 1 & 0 & 0 \\ \hline
		1 & 0 & 0 & 1 & 1 & 1 \\ \hline
		1 & 0 & 1 & 0 & 0 & 0 \\ \hline
		1 & 0 & 1 & 0 & 1 & 1 \\ \hline
		1 & 0 & 1 & 1 & 0 & 1 \\ \hline
		1 & 0 & 1 & 1 & 1 & 1 \\ \hline
		1 & 1 & 0 & 0 & 0 & 0 \\ \hline
		1 & 1 & 0 & 0 & 1 & 1 \\ \hline
		1 & 1 & 0 & 1 & 0 & 1 \\ \hline
		1 & 1 & 0 & 1 & 1 & 1 \\ \hline
		1 & 1 & 1 & 0 & 0 & 1 \\ \hline
		1 & 1 & 1 & 0 & 1 & 1 \\ \hline
		1 & 1 & 1 & 1 & 0 & 1 \\ \hline
		1 & 1 & 1 & 1 & 1 & 1 \\ \hline
		\end{tabular}
	\end{center}


	\begin{center}
		\askmapv{$F$}{a b c d e}{}{00000001000101110001011101111111}{%
		}
	\end{center}

	Avec $F = cde + bce + bde + bcd + abc + ade + ace + abe + acd + abd$.
}
\end{Q}





\begin{Q}
Vous avez deux nombres $A$ et $B$ codés chacun sur deux bits.
Concevez un système permettant de comparer ces deux nombres en utilisant trois bits~: $S_0 = 1$ si $A = B$, $S_1 = 1$ si $A>B$ et $S_2 = 1$ si $A < B$.

\reponse{
	Nous avons besoin de trois fonctions logiques ; une pour chaque bit du système de comparaison.

	\begin{center}
		\begin{tabular}{|c|c|c|c|c|} \hline
		$a_1$ & $a_0$ & $b_1$ & $b_0$ & $S_0$ \\ \hline
		0 & 0 & 0 & 0 & 1 \\ \hline
		0 & 0 & 0 & 1 & 0 \\ \hline
		0 & 0 & 1 & 0 & 0 \\ \hline
		0 & 0 & 1 & 1 & 0 \\ \hline
		0 & 1 & 0 & 0 & 0 \\ \hline
		0 & 1 & 0 & 1 & 1 \\ \hline
		0 & 1 & 1 & 0 & 0 \\ \hline
		0 & 1 & 1 & 1 & 0 \\ \hline
		1 & 0 & 0 & 0 & 0 \\ \hline
		1 & 0 & 0 & 1 & 0 \\ \hline
		1 & 0 & 1 & 0 & 1 \\ \hline
		1 & 0 & 1 & 1 & 0 \\ \hline
		1 & 1 & 0 & 0 & 0 \\ \hline
		1 & 1 & 0 & 1 & 0 \\ \hline
		1 & 1 & 1 & 0 & 0 \\ \hline
		1 & 1 & 1 & 1 & 1 \\ \hline
		\end{tabular}
		\begin{tabular}{|c|c|c|c|c|} \hline
		$a_1$ & $a_0$ & $b_1$ & $b_0$ & $S_1$ \\ \hline
		0 & 0 & 0 & 0 & 0 \\ \hline
		0 & 0 & 0 & 1 & 0 \\ \hline
		0 & 0 & 1 & 0 & 0 \\ \hline
		0 & 0 & 1 & 1 & 0 \\ \hline
		0 & 1 & 0 & 0 & 1 \\ \hline
		0 & 1 & 0 & 1 & 0 \\ \hline
		0 & 1 & 1 & 0 & 0 \\ \hline
		0 & 1 & 1 & 1 & 0 \\ \hline
		1 & 0 & 0 & 0 & 1 \\ \hline
		1 & 0 & 0 & 1 & 1 \\ \hline
		1 & 0 & 1 & 0 & 0 \\ \hline
		1 & 0 & 1 & 1 & 0 \\ \hline
		1 & 1 & 0 & 0 & 1 \\ \hline
		1 & 1 & 0 & 1 & 1 \\ \hline
		1 & 1 & 1 & 0 & 1 \\ \hline
		1 & 1 & 1 & 1 & 0 \\ \hline
		\end{tabular}
		\begin{tabular}{|c|c|c|c|c|} \hline
		$a_1$ & $a_0$ & $b_1$ & $b_0$ & $S_2$ \\ \hline
		0 & 0 & 0 & 0 & 0 \\ \hline
		0 & 0 & 0 & 1 & 1 \\ \hline
		0 & 0 & 1 & 0 & 1 \\ \hline
		0 & 0 & 1 & 1 & 1 \\ \hline
		0 & 1 & 0 & 0 & 0 \\ \hline
		0 & 1 & 0 & 1 & 0 \\ \hline
		0 & 1 & 1 & 0 & 1 \\ \hline
		0 & 1 & 1 & 1 & 1 \\ \hline
		1 & 0 & 0 & 0 & 0 \\ \hline
		1 & 0 & 0 & 1 & 0 \\ \hline
		1 & 0 & 1 & 0 & 0 \\ \hline
		1 & 0 & 1 & 1 & 1 \\ \hline
		1 & 1 & 0 & 0 & 0 \\ \hline
		1 & 1 & 0 & 1 & 0 \\ \hline
		1 & 1 & 1 & 0 & 0 \\ \hline
		1 & 1 & 1 & 1 & 0 \\ \hline
		\end{tabular}
	\end{center}

	\begin{center}
		\askmapiv{$S_0 = \overline{a_1 a_0 b_1 b_0} + \overline{a_1} a_0 \overline{b_1} b_0 + a_1 a_0 b_1 b_0 + a_1 \overline{a_1} b_1 \overline{b_0}$}{$a_1$ $a_0$ $b_1$ $b_0$}{}{1000010000100001}{% raise Z input
			\color{blue}\put(0.1,3.1){\dashbox{0.3}(0.8,0.8){}}%
			\color{green}\put(1.1,2.1){\dashbox{0.2}(0.8,0.8){}}%
			\color{red}\put(2.1,1.1){\dashbox{0.1}(0.8,0.8){}}%
			\color{purple}\put(3.1,0.1){\dashbox{0.05}(0.8,0.8){}}%
		}
	\end{center}

	\begin{center}
		\askmapiv{$S_1 = a_0 \overline{b_1 b_0} + a_1\overline{b_1} + a_1 a_0 \overline{b_0}$}{$a_1$ $a_0$ $b_1$ $b_0$}{}{0000100011001110}{% raise Z input
			\color{blue}\put(1.1,3.1){\dashbox{0.3}(1.8,0.8){}}%
			\color{green}\put(2.1,2.1){\dashbox{0.2}(1.8,1.7){}}%
			\color{red}\put(2.2,3.2){\dashbox{0.1}(0.6,0.5){}}%
			\color{red}\put(2.2,0.2){\dashbox{0.1}(0.6,0.5){}}%
		}
	\end{center}

	\begin{center}
		\askmapiv{$S_2 = \overline{a_1} b_0 + \overline{a_1 a_0} b_0 + \overline{a_0} b_1 b_0$}{$a_1$ $a_0$ $b_1$ $b_0$}{}{0111001100010000}{% raise Z input
			\color{blue}\put(0.1,1.1){\dashbox{0.3}(0.8,1.8){}}%
			\color{green}\put(0.2,0.1){\dashbox{0.2}(1.7,1.8){}}%
			\color{red}\put(0.3,1.2){\dashbox{0.1}(0.5,0.6){}}%
			\color{red}\put(3.3,1.2){\dashbox{0.1}(0.5,0.6){}}%
		}
	\end{center}
}
\end{Q}

\end{document}
