\documentclass[11pt,a4paper]{article}
\usepackage[utf8]{inputenc}
\usepackage[T1]{fontenc}
\usepackage{amsthm} %numéroter les questions
\usepackage[frenchb]{babel}
\usepackage{datetime}
\usepackage{xspace} % typographie IN
\usepackage[hidelinks]{hyperref}% hyperliens
\usepackage[all]{hypcap} %lien pointe en haut des figures
\usepackage[french]{varioref} %voir x p y
\usepackage{fancyhdr}% en têtes
%\input cyracc.def
\usepackage{graphicx} %include pictures
\usepackage{pgfplots}

\usepackage{tikz}
\usetikzlibrary{calc,arrows,automata}
\usetikzlibrary{babel}
\usepackage{circuitikz}
% \usepackage{gnuplottex}
\usepackage{float}
\usepackage{ifthen}

\usepackage[top=1.3 in, bottom=1.3 in, left=1.3 in, right=1.3 in]{geometry} % Yeah, that's bad to play with margins
\usepackage[]{pdfpages}
\usepackage[]{attachfile}

\usepackage{amsmath}
\usepackage{amssymb} % checkmark
\usepackage{enumitem}
\setlist[enumerate]{label=\alph*)}% If you want only the x-th level to use this format, use '[enumerate,x]'
\usepackage{multirow}
\usepackage{bigdelim}%Braces in tabular

\usepackage{askmaps} % https://github.com/parastuffs/askmaps_custom

\newcommand\version{v1.0.1}

%cyr
\newcommand\textcyr[1]{{\fontencoding{OT2}\fontfamily{wncyr}\selectfont #1}}

%Numero du TP :
\def \tpnumber {TP 7 }

\newboolean{corrige}
\ifx\correction\undefined
\setboolean{corrige}{false}% pas de corrigé
\else
\setboolean{corrige}{true}%corrigé
\fi

%\setboolean{corrige}{false}% pas de corrigé

\newboolean{annexes}
\setboolean{annexes}{true}%annexes
%\setboolean{annexes}{false}% pas de annexes

\definecolor{darkblue}{rgb}{0,0,0.5}

\newboolean{mos}
%\setboolean{mos}{true}%annexes
\setboolean{mos}{false}% pas de annexes

\usepackage{aeguill} %guillemets

%% fancy header & foot
\pagestyle{fancy}
\lhead{[ELEC-H-305] Circuits logiques et numériques\\ \tpnumber}
\rhead{\version\\ page \thepage}
\chead{\ifthenelse{\boolean{corrige}}{Corrigé}{}}
\cfoot{}
%%

\pdfinfo{
/Author (Quentin Delhaye, ULB -- BEAMS)
/Title (\tpnumber, ELEC-H-305)
/ModDate (D:\pdfdate)
}

\hypersetup{
pdftitle={\tpnumber [ELEC-H-305] Choucroute logique et numérique},
pdfauthor={Quentin Delhaye, ULB -- BEAMS  },
pdfsubject={}
}

\theoremstyle{definition}% questions pas en italique
\newtheorem{Q}{Question}[] % numéroter les questions [section] ou non []

\newcommand{\reponse}[1]{% pour intégrer une réponse : \reponse{texte} : sera inclus si \boolean{corrige}
	\ifthenelse {\boolean{corrige}} {\paragraph{Réponse :} \color{darkblue}   #1\color{black}} {}
 }

\newcommand{\addcontentslinenono}[4]{\addtocontents{#1}{\protect\contentsline{#2}{#3}{#4}{}}}

\date{}
\title{\vspace{-2cm}Circuits logiques et numériques [ELEC-H-305] \\  \tpnumber Synthèse de circuits \ifthenelse{\boolean{corrige}}{-- Corrigé}{} \\ \small{\version}}

\setlength{\parskip}{0.2cm plus2mm minus1mm} %espacement entre §
\setlength{\parindent}{0pt}

\begin{document}
% \pagestyle{empty}
\maketitle
\vspace*{-1cm}




\begin{Q}
Pour les fonctions proposées dessiner les K-maps, optimiser les fonctions et prévoir les termes redondants (problème des aléas) :
	\begin{enumerate}
		\item $f(a,b,c,d)=\sum_m(0,1,2,6,8,9,10,14)$
		\reponse{
			\begin{center}
				\askmapiv{$F=c\overline{d} + \overline{bc} + \overline{bd}$}{c d a b}{}{1010101011110000}{% raise Z input
				\color{red}\put(0.1,0.1){\dashbox{0.1}(1.8,0.8){}}%
				\color{red}\put(0.1,3.1){\dashbox{0.1}(1.8,0.8){}}%
				\color{green}\put(3.1,0.1){\dashbox{0.1}(0.8,3.8){}}%
				\color{blue}\put(0.2,0.2){\dashbox{0.2}(0.6,0.6){}}%
				\color{blue}\put(0.2,3.2){\dashbox{0.2}(0.6,0.6){}}%
				\color{blue}\put(3.2,0.2){\dashbox{0.2}(0.6,0.6){}}%
				\color{blue}\put(3.2,3.2){\dashbox{0.2}(0.6,0.6){}}%
				}
			\end{center}
		}
		\item $f(a,b,c,d)=\sum_m(1,3,5,7,8,9,12,13)$
		\reponse{
			\begin{center}
				\askmapiv{$F=\overline{a}d + a\overline{c} + \overline{c}d$}{c d a b}{}{0011111100001100}{% raise Z input
				\color{red}\put(0.1,0.1){\dashbox{0.1}(1.8,1.8){}}%
				\color{green}\put(1.1,2.1){\dashbox{0.1}(1.8,1.8){}}%
				\color{blue}\put(1.2,0.2){\dashbox{0.2}(0.6,3.6){}}%
				}
			\end{center}
		}
	\end{enumerate}

\end{Q}





\begin{Q}
Dresser la table de Huffman pour les graphes suivants :

\begin{enumerate}
\item ~\\
\begin{center}
\begin{tikzpicture}[->,>=stealth',shorten >=1pt,auto,node distance=3.5cm,semithick,align=center]
  \tikzstyle{every state}=[fill=white,text=black]

  \node[state]         (A)              {$A$\\ Z=0};
  \node[state]         (B) [right of=A] {$B$\\ Z=0};
  \node[state]         (D) [below of=A] {$D$\\ Z=0};
  \node[state]         (C) [below of=B] {$C$\\ Z=1};

  \path (A) edge [bend left]  node {ab=01} (B)
            edge [bend right]  node {11} (D)
			edge [loop left]  node {00,10} (A)
        (B) edge [loop right] node {01,11} (B)
            edge [bend left]  node {00} (A)
			edge [bend left]  node {10} (C)
        (C) edge [bend left]  node {11} (D)
		    edge [loop right] node {10,00,01} (C)
        (D) edge [loop left]  node {11,00} (D)
            edge              node {01} (B)
			edge [bend left]  node {10} (C);
\end{tikzpicture}
\end{center}

\reponse{
\begin{center}
\begin{tabular}{lllllll}
\hline
\multicolumn{1}{|l|}{} & \multicolumn{1}{l|}{00}          & \multicolumn{1}{l|}{01}          & \multicolumn{1}{l|}{11}          & \multicolumn{1}{l|}{10}          & \multicolumn{1}{l|}{ab} & \multicolumn{1}{l|}{Z} \\ \hline
\multicolumn{1}{|l|}{A}      & \multicolumn{1}{l|}{\textbf{A}} & \multicolumn{1}{l|}{B}          & \multicolumn{1}{l|}{D}          & \multicolumn{1}{l|}{\textbf{A}} & \multicolumn{1}{l|}{}   & \multicolumn{1}{l|}{0} \\ \cline{1-5} \cline{7-7}
\multicolumn{1}{|l|}{B}      & \multicolumn{1}{l|}{A}          & \multicolumn{1}{l|}{\textbf{B}} & \multicolumn{1}{l|}{\textbf{B}}          & \multicolumn{1}{l|}{C}          & \multicolumn{1}{l|}{}   & \multicolumn{1}{l|}{0} \\ \cline{1-5} \cline{7-7}
\multicolumn{1}{|l|}{C}      & \multicolumn{1}{l|}{\textbf{C}}          & \multicolumn{1}{l|}{\textbf{C}}          & \multicolumn{1}{l|}{D} & \multicolumn{1}{l|}{\textbf{C}} & \multicolumn{1}{l|}{}   & \multicolumn{1}{l|}{1} \\ \cline{1-5} \cline{7-7}
\multicolumn{1}{|l|}{D}      & \multicolumn{1}{l|}{\textbf{D}}          & \multicolumn{1}{l|}{B} & \multicolumn{1}{l|}{\textbf{D}} & \multicolumn{1}{l|}{C}          & \multicolumn{1}{l|}{}   & \multicolumn{1}{l|}{0} \\ \cline{1-5} \cline{7-7}
                      &                                  &                                  &                                  &                                  &                         &
\end{tabular}
\end{center}
}

\item ~\\
\begin{center}
\begin{tikzpicture}[->,>=stealth',shorten >=1pt,auto,node distance=3.5cm,semithick,align=center]
  \tikzstyle{every state}=[fill=white,text=black]

  \node[state]         (A)              {$A$\\ Z=0};
  \node[state]         (B) [right of=A] {$B$\\ Z=0};
  \node[state]         (C) [right of=B] {$C$\\ Z=1};

  \path (A) edge			  node {x=1} (B)
        	edge [loop below] node {0} (A)
        (B) edge			  node {0} (C)
			edge [loop below] node {1} (B)
        (C) edge [loop below]  node {1,0} (C);
\end{tikzpicture}
\end{center}

\reponse{
\begin{center}
\begin{tabular}{lllll} \hline
\multicolumn{1}{|l|}{}  & \multicolumn{1}{l|}{0}          & \multicolumn{1}{l|}{1}          & \multicolumn{1}{l|}{x} & \multicolumn{1}{l|}{Z} \\ \hline
\multicolumn{1}{|l|}{A} & \multicolumn{1}{l|}{\textbf{A}} & \multicolumn{1}{l|}{B}          & \multicolumn{1}{l|}{}  & \multicolumn{1}{l|}{0} \\ \hline
\multicolumn{1}{|l|}{B} & \multicolumn{1}{l|}{C}          & \multicolumn{1}{l|}{\textbf{B}} & \multicolumn{1}{l|}{}  & \multicolumn{1}{l|}{0} \\ \hline
\multicolumn{1}{|l|}{C} & \multicolumn{1}{l|}{\textbf{C}} & \multicolumn{1}{l|}{\textbf{C}} & \multicolumn{1}{l|}{}  & \multicolumn{1}{l|}{1} \\ \hline
\end{tabular}
\end{center}
}

\end{enumerate}
\end{Q}








\begin{Q}
À partir de cette table de Huffman codée trouver les équations et le graphe d'états correspondant :

\begin{center}
\begin{tabular}{lllllll}
\hline
\multicolumn{1}{|l|}{$Y_1$ $Y_2$} & \multicolumn{1}{l|}{00}          & \multicolumn{1}{l|}{01}          & \multicolumn{1}{l|}{11}          & \multicolumn{1}{l|}{10}          & \multicolumn{1}{l|}{ab} & \multicolumn{1}{l|}{Z} \\ \hline
\multicolumn{1}{|l|}{00}      & \multicolumn{1}{l|}{\textbf{00}} & \multicolumn{1}{l|}{01}          & \multicolumn{1}{l|}{01}          & \multicolumn{1}{l|}{\textbf{00}} & \multicolumn{1}{l|}{}   & \multicolumn{1}{l|}{0} \\ \cline{1-5} \cline{7-7}
\multicolumn{1}{|l|}{01}      & \multicolumn{1}{l|}{00}          & \multicolumn{1}{l|}{\textbf{01}} & \multicolumn{1}{l|}{11}          & \multicolumn{1}{l|}{11}          & \multicolumn{1}{l|}{}   & \multicolumn{1}{l|}{1} \\ \cline{1-5} \cline{7-7}
\multicolumn{1}{|l|}{11}      & \multicolumn{1}{l|}{\textbf{11}}          & \multicolumn{1}{l|}{10}          & \multicolumn{1}{l|}{\textbf{11}} & \multicolumn{1}{l|}{\textbf{11}} & \multicolumn{1}{l|}{}   & \multicolumn{1}{l|}{1} \\ \cline{1-5} \cline{7-7}
\multicolumn{1}{|l|}{10}      & \multicolumn{1}{l|}{11}          & \multicolumn{1}{l|}{\textbf{10}} & \multicolumn{1}{l|}{\textbf{10}} & \multicolumn{1}{l|}{00}          & \multicolumn{1}{l|}{}   & \multicolumn{1}{l|}{1} \\ \cline{1-5} \cline{7-7}
$y_1$ $y_2$                      &                                  &                                  &                                  &                                  &                         &
\end{tabular}
\end{center}

\reponse{
	\begin{center}
	\begin{tikzpicture}[->,>=stealth',shorten >=1pt,auto,node distance=3.5cm,semithick,align=center]
	  \tikzstyle{every state}=[fill=white,text=black]

	  \node[state]         (A)              {$00$\\ Z=0};
	  \node[state]         (B) [right of=A] {$01$\\ Z=1};
	  \node[state]         (D) [below of=A] {$10$\\ Z=1};
	  \node[state]         (C) [below of=B] {$11$\\ Z=1};

	  \path (A) edge [bend left]  node {ab=01,11} (B)
	            % edge [bend right]  node {11} (D)
				edge [loop left]  node {00,10} (A)
	        (B) edge [loop right] node {01} (B)
	            edge [bend left]  node {00} (A)
				edge [bend left]  node {11,10} (C)
				edge [bend left]  node {00} (A)
	        (C) edge [bend left]  node {01} (D)
			    edge [loop right] node {10,00,11} (C)
	        (D) edge [loop left]  node {11,01} (D)
	            edge [bend left]  node {10} (A)
				edge [bend left]  node {00} (C);
	\end{tikzpicture}
	\end{center}

	Afin de déterminer l'équation correspondante, on peut dériver des tables de Karnaugh de la table de Huffman.

	\begin{center}
		\askmapiv{$Y_1=y_2a + y_1 \overline{a} + y_1b$}{a b $y_1$ $y_2$}{}{0011001101010111}{% raise Z input
		\color{red}\put(0.1,0.1){\dashbox{0.1}(1.8,1.8){}}%
		\color{green}\put(2.1,1.1){\dashbox{0.1}(1.8,1.8){}}%
		\color{blue}\put(1.1,0.2){\dashbox{0.2}(1.8,1.6){}}%
		}

		\askmapiv{$Y_2=y_1\overline{ab} + \overline{y_1}b + y_2a$}{a b $y_1$ $y_2$}{}{0011110001011101}{% raise Z input
		\color{red}\put(0.1,0.1){\dashbox{0.1}(0.8,1.8){}}%
		\color{green}\put(1.1,2.1){\dashbox{0.1}(1.8,1.8){}}%
		\color{blue}\put(2.1,1.1){\dashbox{0.2}(1.8,1.8){}}%
		}

		\askmapiv{$Z=y_2+y_1$}{a b $y_1$ $y_2$}{}{0-11-11101-1-111}{% raise Z input
		\color{red}\put(0.1,0.1){\dashbox{0.1}(3.8,1.8){}}%
		\color{blue}\put(0.2,1.1){\dashbox{0.2}(3.6,1.8){}}%
		}
	\end{center}
}
\end{Q}





\end{document}
