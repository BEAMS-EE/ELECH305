\documentclass[11pt,a4paper]{article}
\usepackage[utf8]{inputenc}
\usepackage[T1]{fontenc}
\usepackage{amsthm} %numéroter les questions
\usepackage[frenchb]{babel}
\usepackage{datetime}
\usepackage{xspace} % typographie IN
\usepackage{hyperref}% hyperliens
\usepackage[all]{hypcap} %lien pointe en haut des figures
\usepackage[french]{varioref} %voir x p y
\usepackage{fancyhdr}% en têtes
%\input cyracc.def
\usepackage{graphicx} %include pictures
\usepackage{pgfplots}

\usepackage{tikz}
\usetikzlibrary{calc}
\usetikzlibrary{babel}
\usepackage{circuitikz}
% \usepackage{gnuplottex}
\usepackage{float}
\usepackage{ifthen}

\usepackage[top=1.3 in, bottom=1.3 in, left=1.3 in, right=1.3 in]{geometry} % Yeah, that's bad to play with margins
\usepackage[]{pdfpages}
\usepackage[]{attachfile}

\usepackage{amsmath}
\usepackage{enumitem}
\setlist[enumerate]{label=\alph*)}% If you want only the x-th level to use this format, use '[enumerate,x]'
\setlist[itemize]{label=$\bullet$}
\usepackage{multirow}

\newcommand\version{v1.0.0}

%cyr
\newcommand\textcyr[1]{{\fontencoding{OT2}\fontfamily{wncyr}\selectfont #1}}

%Numero du TP :
\def \tpnumber {TP 1 }

\newboolean{corrige}
\ifx\correction\undefined
\setboolean{corrige}{false}% pas de corrigé
\else
\setboolean{corrige}{true}%corrigé
\fi

%\setboolean{corrige}{false}% pas de corrigé

\newboolean{annexes}
\setboolean{annexes}{true}%annexes
%\setboolean{annexes}{false}% pas de annexes

\definecolor{darkblue}{rgb}{0,0,0.5}

\newboolean{mos}
%\setboolean{mos}{true}%annexes
\setboolean{mos}{false}% pas de annexes

\usepackage{aeguill} %guillemets

%% fancy header & foot
\pagestyle{fancy}
\lhead{[ELEC-H-310] Électronique numérique\\ \tpnumber}
\rhead{\mydate\today\\ page \thepage}
\chead{\ifthenelse{\boolean{corrige}}{Corrigé}{}}
\cfoot{}
%%

\pdfinfo{
/Author (Quentin Delhaye, ULB -- BEAMS)
/Title (\tpnumber, ELEC-H-305)
/ModDate (D:\pdfdate)
}

\hypersetup{
pdftitle={\tpnumber [ELEC-H-305] Choucroute logique et numérique},
pdfauthor={Quentin Delhaye 2017 ULB -- BEAMS  },
pdfsubject={}
}

\theoremstyle{definition}% questions pas en italique
\newtheorem{Q}{Question}[] % numéroter les questions [section] ou non []

\newcommand{\reponse}[1]{% pour intégrer une réponse : \reponse{texte} : sera inclus si \boolean{corrige}
	\ifthenelse {\boolean{corrige}} {\paragraph{Réponse :} \color{darkblue}   #1\color{black}} {}
 }

\newcommand{\addcontentslinenono}[4]{\addtocontents{#1}{\protect\contentsline{#2}{#3}{#4}{}}}

\date{}
\title{\vspace{-2cm}Circuits logiques et numériques [ELEC-H-305] \\  \tpnumber : Systèmes de numérotation \ifthenelse{\boolean{corrige}}{-- Corrigé}{} \\ \small{\version}}

\setlength{\parskip}{0.2cm plus2mm minus1mm} %espacement entre §
\setlength{\parindent}{0pt}

\begin{document}
\pagestyle{empty}
\maketitle
\vspace*{-1cm}

\begin{Q}
	Convertir dans les autres bases utiles les nombres suivants :
	\begin{enumerate}
	\item $(82)_{10}$
	\item $(122)_{10}$
	\item $(1001110001)_2$
	\item $(762)_8$
	\item $(214)_8$
	\item $(F6D)_{16}$
	\item $(B65F)_{16}$
	\item $(0.625)_{10}$
	\item $(10110001101011.111100000110)_2$
	\item $(127.4)_8$
	\item $(673.12)_8$
	\item $(3A6.C)_{16}$
	\end{enumerate}

	\reponse{
	\begin{itemize}
		\item Base 10 $\rightarrow$ base 2 par divisions successives : $(82)_{10} = (1010010)_2$

			\begin{tabular}{c|c|l}
			82 & :2 & 0 (LSB)\\
			41 & :2 & 1 \\
			20 & :2 & 0 \\
			10 & :2 & 0 \\
			5  & :2 & 1 \\
			2  & :2 & 0 \\
			1  & :2 & 1 (MSB)\\
			0  &    &   \\
			\end{tabular}
		\item Base 10 $\rightarrow$ base 8 par divisions successives : $(82)_{10} = (122)_8$

			\begin{tabular}{c|c|c}
			82 & :8 & 2 \\
			10 & :8 & 2 \\
			1  & :8 & 1 \\
			0  &    &   \\
			\end{tabular}

		\item Base 10 $\rightarrow$ base 16 par divisions successives : $(82)_{10} = (52)_{16}$

			\begin{tabular}{c|c|c}
			82 & :16 & 2 \\
			5  & :16 & 5 \\
			0  &     &   \\
			\end{tabular}

		\item Base 2 $\rightarrow$ base 8 par groupements : $(1001110001)_2 = (1161)_8$
		\[\underbrace{001}_{1}\underbrace{001}_{1}\underbrace{110}_{6}\underbrace{001}_{1}\]

		\item Base 2 $\rightarrow$ base 16 par groupements : $(1001110001)_2 = (271)_{16}$
		\[ \underbrace{0010}_{2}\underbrace{0111}_{7}\underbrace{0001}_{1} \]

		\item Base 8 $\rightarrow$ base 2 par séparation : $(762)_8 = (111110010)_2$
		\[ \underbrace{7}_{111}\underbrace{6}_{110}\underbrace{2}_{010} \]

		\item Base 16 $\rightarrow$ base 2 par séparation : $(F6D)_{16} = (111101101101)_2$
		\[\underbrace{F}_{1111}\underbrace{6}_{0110}\underbrace{D}_{1101} \]

		\item Partie décimale en base 10 $\rightarrow$ base 2 par multiplications successives : $(0.625)_{10} = (0.101)_2$

			\begin{tabular}{l|c|l}
			.625 & x2 & 1 (MSB)\\
			.25  & x2 & 0 \\
			.5   & x2 & 1 (LSB)\\
			0    &    &   \\
			\end{tabular}




	\end{itemize}

	Le tableau suivant reprend tous les résultats de l'exercice.
		\begin{center}
			\begin{tabular}{|l|l|l|l|}\hline
				Base 2 & Base 8 & Base 10 & Base 16 \\ \hline
				1010010 & 122 & \textbf{82} & 52 \\ \hline
				1111010 & 172 & \textbf{122} & 7A \\ \hline
				\textbf{1001110001} & 1161 & 625 & 271 \\ \hline
				111110010 & \textbf{762} & 498 & 1F2 \\ \hline
				10001100 & \textbf{214} & 140 & 8C \\ \hline
				111101101101 & 7555 & 3949 & \textbf{F6D} \\ \hline
				1011011001011111 & 133137 & 46687 & \textbf{B65F} \\ \hline
				0.101 & 0.5 & \textbf{0.625} & 0.A \\ \hline
				\textbf{10110001101011.111100000110} & 26153.7406 & 11371.93896484375 & 2C6B.F06 \\ \hline
				1010111.1 & \textbf{127.4} & 87.5 & 57.8 \\ \hline
				110111011.001010 & \textbf{673.12} & 443.15625 & 1BB.28 \\ \hline
				001110100110.1100 & 1646.6 & 934.75 & \textbf{3A6.C} \\ \hline
			\end{tabular}
		\end{center}
	}
\end{Q}

\begin{Q}
Effectuer l’addition suivante dans toutes les bases utiles. Vérifier les résultats en
les convertissant en base 10 :
$$(3633)_{10} + (254)_{10}$$

\reponse{~\\
	Base 2~:
	\begin{tabular}{ccccccccccccc}
		& 1 & 1 & 1 & 0 & 0 & 0 & 1 & 1 & 0 & 0 & 0 & 1 \\
		+ & &   &   &   & 1 & 1 & 1 & 1 & 1 & 1 & 1 & 0\\ \hline
		& 1 & 1 & 1 & 1 & 0 & 0 & 1 & 0 & 1 & 1 & 1 & 1 \\
	\end{tabular}

	Base 8~:
	\begin{tabular}{ccccc}
		  & 7 & 0 & 6 & 1 \\
		+ &   & 3 & 7 & 6 \\ \hline
		  & 7 & 4 & 5 & 7 \\
		\end{tabular}

	Base 16~:
	\begin{tabular}{cccc}
		& E & 3 & 1 \\
		+ & & F & E \\ \hline
		& F & 2 & F\\
	\end{tabular}
}
\end{Q}

\begin{Q}
Représenter $(0.345)_{10}$ en base 2 et en base 8.
\reponse{
	\begin{itemize}
		\item $(0.010110\dots)_2$
		\item $(0.260507\dots)_8$
	\end{itemize}
}
\end{Q}


\begin{Q}
Effectuer les opérations suivantes~:
\begin{enumerate}
	\item $(10110)_2 - (10010)_2$
	\item $(10110)_2 - (10011)_2$
	\item $(5475)_8 - (3764)_8$
	\item $(540045)_8 - (325654)_8$
	\item $(E46)_{16} - (59F)_{16}$
	\item $(4321)_{16} - (2ECD)_{16}$
	\item $(1011)_2 * (1001)_2$
	\item $(762)_8 * (45)_8$
	\item $(543)_8 * (27)_8$
	\item $(1CF)_{16} * (B6)_{16}$
	\item $(2ECD)_{16} * (4321)_{16}$
	\item $(1100)_2 : (011)_2$
	\item $(110101)_2 : (111)_2$
	\item $(533)_8 : (26)_8$
	\item $(2ECD)_{16} : (12)_{16}$
\end{enumerate}

\reponse{
	\begin{enumerate}
		\item $(00100)_2$
		\item $(00011)_2$
		\item $(1511)_8$
		\item $(212171)_8$
		\item $(8A7)_{16}$
		\item $(1454)_{16}$
		\item $(1100011)_2$

		\begin{tabular}{ccccccc}
		  &  &   & 1 & 0 & 1 & 1 \\
		  &  & x & 1 & 0 & 0 & 1 \\ \hline
		  &  &   & 1 & 0 & 1 & 1 \\
		1 & 0 & 1 & 1 & $\cdot$ & $\cdot$ & $\cdot$ \\ \hline
		1 & 1 & 0 & 0 & 0 & 1 & 1 \\
		\end{tabular}
		
		\item $(43772)_8$

		\begin{tabular}{ccccc}
		&& 7 & 6 & 2 \\
		& x & & 4 & 5 \\ \hline
		& 4 & 6 & 7 & 2 \\
		3 & 7 & 1 & 0 & $\cdot$ \\ \hline
		4 & 3 & 7 & 7 & 2 \\
		\end{tabular}

		\item $(17745)_8$
		\item $(1492A)_{16}$
		\item $(C45AF6D)_{16}$
		\item $(100)_2$
		\item $(111.100)_2$

		\begin{tabular}{cccccccccccccccccc}
		  & 1 & 1 & 0 & 1 & 0 & 1 &         &   & \multicolumn{1}{c|}{} & 1 & 1 & 1 &         &   &   &   &   \\ \cline{11-18}
		- &   & 1 & 1 & 1 & 0 & 0 &         &   & \multicolumn{1}{c|}{} &   & 1 & 0 & 0 &         &   &   &   \\ \cline{1-7}
		  &   & 1 & 1 & 0 & 0 & 1 &         &   & \multicolumn{1}{c|}{} &   &   &   &   &         &   &   &   \\
		- &   &   & 1 & 1 & 1 & 0 &         &   & \multicolumn{1}{c|}{} &   &   & 1 & 0 &         &   &   &   \\ \cline{1-7}
		  &   &   & 1 & 0 & 1 & 1 &         &   & \multicolumn{1}{c|}{} &   &   &   &   &         &   &   &   \\
		- &   &   &   & 1 & 1 & 1 &         &   & \multicolumn{1}{c|}{} &   &   &   & 1 &         &   &   &   \\\cline{1-9}
		  &   &   &   & 1 & 0 & 0 & $\cdot$ & 0 & \multicolumn{1}{c|}{} &   &   &   &   &         &   &   &   \\
		- &   &   &   &   & 1 & 1 & $\cdot$ & 1 &                       & + &   &   &   & $\cdot$ & 1 & 0 & 0 \\ \cline{11-18}
		  &   &   &   &   &   &   &         &   &                       &   & 1 & 1 & 1 & $\cdot$ & 1 & 0 & 0 \\
		\end{tabular}

		\item $(17.21)_8$
		\item $(299.B)_{16}$
	\end{enumerate}
}
\end{Q}





\begin{Q}
Représentation des nombres négatifs
\begin{enumerate}
	\item Représenter $(-14)_{10}$ sur 8 bits en base 2 dans les 3 modes de représentation.
	\reponse{
		\begin{tabular}{|c|c|c|}\hline
			SVA & C1 & C2 \\ \hline
			10001110 & 11110001 & 11110010 \\ \hline
		\end{tabular}
	}
	\item Si on utilise 4 bits, quels sont, dans les 3 modes de représentation, les plus
	petites et les plus grandes valeurs représentables ? Comment se représente la
	valeur 0 ?
	\reponse{
		\begin{center}
			\begin{tabular}{|c|c|c|c|}\hline
				& min & 0 & max \\ \hline
				\multirow{2}{*}{SVA} & \multirow{2}{*}{1111} & 0000 & \multirow{2}{*}{0111} \\
				& & 1000 & \\ \hline
				\multirow{2}{*}{C1} & \multirow{2}{*}{10000} & 0000 & \multirow{2}{*}{0111} \\
				& & 1111 & \\ \hline
				C2 & 1000 & 0000 & 0111 \\ \hline
			\end{tabular}
		\end{center}

		À titre de bonus, ci-suit un tableau comparatif des différents modes de représentation (sur 4 bits)~:
		\begin{center}
			\begin{tabular}{|c|c|c|c|} \hline
				Base 10 & Signé & C1 & C2 \\ \hline
				7 & 0111 & 0111 & 0111 \\ \hline
				6 & 0110 & 0110 & 0110 \\ \hline
				5 & 0101 & 0101 & 0101 \\ \hline
				4 & 0100 & 0100 & 0100 \\ \hline
				3 & 0011 & 0011 & 0011 \\ \hline
				2 & 0010 & 0010 & 0010 \\ \hline
				1 & 0001 & 0001 & 0001 \\ \hline
				\multirow{2}{*}{0} & 0000 & 0000 & \multirow{2}{*}{0000} \\
				& 1000 & 1111 & \\ \hline
				-1 & 1001 & 1110 & 1111 \\ \hline
				-2 & 1010 & 1101 & 1110 \\ \hline
				-3 & 1011 & 1100 & 1101 \\ \hline
				-4 & 1100 & 1011 & 1100 \\ \hline
				-5 & 1101 & 1010 & 1011 \\ \hline
				-6 & 1110 & 1001 & 1010 \\ \hline
				-7 & 1111 & 1000 & 1001 \\ \hline
				-8 & N/A & N/A & 1000 \\ \hline
			\end{tabular}
		\end{center}
	}
	\item Effectuer les additions suivantes (sur 8 bits) dans les trois modes de représentation et repérez les problèmes d'overflow~:
	\reponse{
		\begin{tabular}{|l|c|c|c|c|} \hline
		Base 10 & 52 & -52 & 84 & -84 \\ \hline
		SVA & 00110100 & 10110100 & 01010100 & 11010100 \\ \hline
		C1 & 00110100 & 11001011 & 01010100 & 10101011 \\ \hline
		C2 & 00110100 & 11001100 & 01010100 & 10101100 \\ \hline
		\end{tabular}
	}
		\begin{itemize}
			\item $52 + 84$
			\reponse{
			\begin{itemize}
				\item SVA : Overflow.
				\item C1 et C2 : overflow car le résultat n'est pas du même signe que les opérandes.
			\end{itemize}
			}

			\item $52 - 84$
			\reponse{
			\begin{itemize}
				\item SVA : Soustraction classique.
				\item C1 et C2 : Additionner 52 et -84.
			\end{itemize}
			}

			\item $84 - 52$
			\reponse{
			\begin{itemize}
				\item SVA : Soustraction classique.
				\item C1 : $84 + (-52)$ provoque un report de $1$ sur le MSB qu'il faut ajouter au LSB.
				\item C2 : Provoque aussi un report, mais on l'ignore.
			\end{itemize}
			}
			
			\item $-84 - 52$
			\reponse{
			\begin{itemize}
				\item SVA : Les deux opérandes étant du même signe, on peut additionner 52 et 84, mais on a de nouveau un overflow.
				\item C1 et C2 : On additionne $-52$ et $-84$, mais le résultat n'est pas du même signe $\rightarrow$ underflow.
			\end{itemize}
			}
		\end{itemize}
\end{enumerate}
\end{Q}

\begin{Q}
Représentation en virgule flottante (IEEE standard)

\begin{enumerate}
	\item Si on vous donne le nombre $(10100.10011)_2$ et sachant que la partie entière et fractionnaire sont toutes les deux exprimées sur six bits, quel est son équivalent en virgule flottante ?
	\reponse{
		\begin{itemize}
			\item Signe : 0
			\item Exposant : 127 (base) + 4 (décalage) = $131_{10} = 10000011_2$
			\item Mantisse : 01001001100000000000000
		\end{itemize}
		$(10100.10011)_2 = 0 10000011 01001001100000000000000$
	}

	\item Convertir en binaire les nombres suivants représentés en virgule flottante~:
		\begin{itemize}
			\item $0\ 10000010\ 10000010\ \dots \ 000$
			\reponse{
				\begin{itemize}
					\item Signe : +
					\item Exposant : $10000010_2 = 129_{10}$
					\item Mantisse : 10000010000000000000000
				\end{itemize}
				$0\ 10000010\ 10000010\ \dots \ 000 = (110.00001)_2$
			}
			\item $1\ 01111000\ 01000110\ \dots \ 000$
			\reponse{
				\begin{itemize}
					\item Signe : -
					\item Exposant : $01111000_2 = 120_{10}$ % Décalage de -7
					\item Mantisse : 01000110000000000000000
				\end{itemize}
				$1\ 01111000\ 01000110\ \dots \ 000 = -(0.00000010100011)_2$% On part de 1.0100011 et on décale de 7 vers la droite => 0.00000010100011
			}
		\end{itemize}
\end{enumerate}
\end{Q}

\end{document}
