\documentclass[11pt,a4paper]{article}
\usepackage[utf8]{inputenc}
\usepackage[T1]{fontenc}
\usepackage{amsthm} %numéroter les questions
\usepackage[frenchb]{babel}
\usepackage{datetime}
\usepackage{xspace} % typographie IN
\usepackage[hidelinks]{hyperref}% hyperliens
\usepackage[all]{hypcap} %lien pointe en haut des figures
\usepackage[french]{varioref} %voir x p y
\usepackage{fancyhdr}% en têtes
%\input cyracc.def
\usepackage{graphicx} %include pictures
\usepackage{pgfplots}

\usepackage{tikz}
\usetikzlibrary{calc}
\usetikzlibrary{babel}
\usepackage{circuitikz}
% \usepackage{gnuplottex}
\usepackage{float}
\usepackage{ifthen}

\usepackage[top=1.3 in, bottom=1.3 in, left=1.3 in, right=1.3 in]{geometry} % Yeah, that's bad to play with margins
\usepackage[]{pdfpages}
\usepackage[]{attachfile}

\usepackage{amsmath}
\usepackage{enumitem}
\setlist[enumerate]{label=\alph*)}% If you want only the x-th level to use this format, use '[enumerate,x]'
\usepackage{multirow}

\newcommand\version{v1.0.0}

%cyr
\newcommand\textcyr[1]{{\fontencoding{OT2}\fontfamily{wncyr}\selectfont #1}}

%Numero du TP :
\def \tpnumber {TP 2 }

\newboolean{corrige}
\ifx\correction\undefined
\setboolean{corrige}{false}% pas de corrigé
\else
\setboolean{corrige}{true}%corrigé
\fi

%\setboolean{corrige}{false}% pas de corrigé

\newboolean{annexes}
\setboolean{annexes}{true}%annexes
%\setboolean{annexes}{false}% pas de annexes

\definecolor{darkblue}{rgb}{0,0,0.5}

\newboolean{mos}
%\setboolean{mos}{true}%annexes
\setboolean{mos}{false}% pas de annexes

\usepackage{aeguill} %guillemets

%% fancy header & foot
\pagestyle{fancy}
\lhead{[ELEC-H-305] Circuits logiques et numériques\\ \tpnumber}
\rhead{\version\\ page \thepage}
\chead{\ifthenelse{\boolean{corrige}}{Corrigé}{}}
\cfoot{}
%%

\pdfinfo{
/Author (Quentin Delhaye, ULB -- BEAMS)
/Title (\tpnumber, ELEC-H-305)
/ModDate (D:\pdfdate)
}

\hypersetup{
pdftitle={\tpnumber [ELEC-H-305] Choucroute logique et numérique},
pdfauthor={Quentin Delhaye, ULB -- BEAMS  },
pdfsubject={}
}

\theoremstyle{definition}% questions pas en italique
\newtheorem{Q}{Question}[] % numéroter les questions [section] ou non []

\newcommand{\reponse}[1]{% pour intégrer une réponse : \reponse{texte} : sera inclus si \boolean{corrige}
	\ifthenelse {\boolean{corrige}} {\paragraph{Réponse :} \color{darkblue}   #1\color{black}} {}
 }

\newcommand{\addcontentslinenono}[4]{\addtocontents{#1}{\protect\contentsline{#2}{#3}{#4}{}}}

\date{}
\title{\vspace{-2cm}Circuits logiques et numériques [ELEC-H-305] \\  \tpnumber : Théorèmes de l'algèbre de Boole \ifthenelse{\boolean{corrige}}{-- Corrigé}{} \\ \small{\version}}

\setlength{\parskip}{0.2cm plus2mm minus1mm} %espacement entre §
\setlength{\parindent}{0pt}

\begin{document}
% \pagestyle{empty}
\maketitle
\vspace*{-1cm}





\begin{Q}
\textbf{Codes pondérés}
\begin{enumerate}
	\item Comment se représentent les chiffres 0 \dots 9 dans le code auto-complémentaire 1224 ?
	\reponse{
	Un code « 1224 » signifie une représentation sur quatre bits ayant pour poids 1, 2, 2 et 4 (du MSB vers le LSB).
	Étant donné que chaque poids de chaque nombre codé est une valeur binaire naturelle, ce code est « pondéré ».
	Pour qu'il soit auto-complémentaire, le 0 codé doit être le complémentaire du 9 codé, le 1 codé celui du 8 codé, etc. (\textit{Notez que le code BCD est certes pondéré, mais il n'est pas auto-complémentaire.})

	En suivant ces propriétés, on peut trouver deux codages 1224 adéquats~:
		\begin{center}
			\begin{tabular}{c|c}
			Base 10 & Code 1224 \\ \hline
			0 & 0000 \\
			1 & 1000 \\
			2 & 0010 \\
			3 & 1010 \\
			4 & 0110 \\
			5 & 1001 \\
			6 & 0101 \\
			7 & 1101 \\
			8 & 0111 \\
			9 & 1111 \\
			\end{tabular}
			\begin{tabular}{c|c}
			Base 10 & Code 1224 \\ \hline
			0 & 0000 \\
			1 & 1000 \\
			2 & 0100 \\
			3 & 1100 \\
			4 & 0001 \\
			5 & 1110 \\
			6 & 0011 \\
			7 & 1011 \\
			8 & 0111 \\
			9 & 1111 \\
			\end{tabular}
		\end{center}

	Ce code n'est cependant pas utilisable en pratique.
	\begin{itemize}
		\item Avec le code du tableau de gauche, certaines additions ne sont pas cohérentes.
		Par exemple, $2+4$ devient $0010_2 + 0110_2 = 1000_2 = 1_{10}$.
		\item Certaines additions comme un simple $7+1$ provoquent un overflow.
	\end{itemize}
	}

	\item Effectuer l'opération 199 + 124 après avoir représenté les nombres en BCD.
	\reponse{
	\begin{center}
		\begin{tabular}{cccc|l}
		  & 0001 & 1001 & 1001 & $199_{10}$ \\
		+ & 0001 & 0010 & 0100 & $124_{10}$ \\ \hline
		  & 0010 & $\underbrace{1011}_{>9}$ & $\underbrace{1101}_{>9}$ & \\
		+ &      & 0110 & 0110 & Correction \\ \hline
		  & $\underbrace{0011}_{3}$ & $\underbrace{0010}_{2}$ & $\underbrace{0011}_{3}$ & \\
		\end{tabular}
	\end{center}
	}
\end{enumerate}
\end{Q}








\begin{Q}
\textbf{Détection d'erreurs}

Les quatre mots suivants de quatre bits sont transmis : 1011, 1001, 1011 et 1101.
Le bit le moins significatif de chaque mot est un bit de parité (impaire). %Dans la convention du cours, le bit de parité n'est pas compté dans la parité.
Le dernier mot est composé de bits de parité (paire) portant sur les bits de même position des mots précédemment transmis.\\
Une erreur s'est-elle produite lors de la transmission ?
Si oui, corrigez-la.

\reponse{
\begin{center}
	\begin{tabular}{ccccl}\cline{4-4}
	1 & 0 & 1 & \multicolumn{1}{|c|}{1} & OK \\ \cline{2-2}
	1 & \multicolumn{1}{|c|}{\textbf{0}} & 0 & \multicolumn{1}{|c|}{1} & $\leftarrow$ \\ \cline{2-2}
	1 & 0 & 1 & \multicolumn{1}{|c|}{1} & OK \\ \cline{1-4}
	\multicolumn{1}{|c}{1} & 1 & 0 & \multicolumn{1}{|c|}{1} & OK \\ \cline{1-4}
	OK & $\uparrow$ & OK & OK & \\
	\end{tabular}
\end{center}

La case entourée en gras devrait être un « 1 » au lieu de « 0 ».
}

\end{Q}


\begin{Q}
	En utilisant les axiomes prouver les théorèmes\footnote{Les théorèmes 5 et 6 peuvent être prouvés avec des tables de vérité.} :

\begin{minipage}[t]{0.5\linewidth}
	\centering Axiomes :
	$$A1a : x \in B, y \in B \rightarrow x+y \in B$$
	$$A1b : x \in B, y \in B \rightarrow x \cdot y \in B$$
	$$A2a : x+0 = x$$
	$$A2b : x \cdot 1 = x$$
	$$A3a : x \cdot (y+z) = (x \cdot y) + (x \cdot z)$$
	$$A3b : x+ (y \cdot z) = (x+y)(x+z)$$
	$$A4a : x+y = y+x$$
	$$A4b : x \cdot y = y \cdot x$$
	$$A5a : x+\overline{x} = 1$$
	$$A5b : x \cdot \overline{x} = 0$$
	$$A6 : \exists x, y \in B : x \neq y$$
\end{minipage}
\begin{minipage}[t]{0.5\linewidth}
	\centering Théorèmes :
	$$T1a : x+x=x$$
	$$T1b : x \cdot x=x$$
	$$T2a : x+1=1$$
	$$T2b : x \cdot 0=0$$
	$$T3a : x+y \cdot x=x$$
	$$T3b : x \cdot (x+y)=x$$
	$$T4 : \overline{(\overline{x})}=x$$
	$$T5a : (x+y)+z = x + (y+z)$$
	$$T5b : x \cdot (y \cdot z) = (x \cdot y) \cdot z$$
	$$T6a : \overline{(x+y)} = \overline{x} \cdot \overline{y}$$
	$$T6b : \overline{(x \cdot y)} = \overline{x} + \overline{y}$$
\end{minipage}

\reponse{
\begin{itemize}
	\item $T1a : x+x=x$
	\begin{align*}
		x + x & = (x + x) \cdot 1&\mbox{, A2b}\\
		& = (x + x) \cdot (x + \overline{x})&\mbox{, A5a}\\
		& = x + x \cdot \overline{x}&\mbox{, A3b}\\
		& = x + 0 & \mbox{, A5b}\\
		& = x &\mbox{, A2a}
	\end{align*}

	\item $T1b : x \cdot x=x$
	\begin{align*}
		x \cdot x & = x \cdot x  + 0&\mbox{, A2a}\\
		& = x\cdot x + x \cdot \overline{x} &\mbox{, A5b}\\
		& = x \cdot (x + \overline{x}) &\mbox{, A3a} \\
		& = x \cdot 1&\mbox{, A5a}\\
		& = x&\mbox{, A2b}
	\end{align*}

	\item $T2a : x+1=1$
	\begin{align*}
		x + 1 & = (x+1) \cdot 1&\mbox{, A2b}\\
		& = (x+1) \cdot (x+\overline{x})&\mbox{, A5a}\\
		& = x + 1 \cdot \overline{x} &\mbox{, A3b}\\
		& = x + \overline{x} &\mbox{, A2b}\\
		& = 1 &\mbox{, A5a}
	\end{align*}

	\item $T2b : x \cdot 0=0$
	\begin{align*}
		x \cdot 0 & = x \cdot 0 + 0 &\mbox{, A2a} \\
		& = x \cdot 0 + x \cdot \overline{x}&\mbox{, A5b} \\
		& = x \cdot (0 + \overline{x}) &\mbox{, A3a} \\
		& = x \cdot (\overline{x}) &\mbox{, A2a} \\
		& = 0 &\mbox{, A5b}
	\end{align*}

	\item $T3a : x+y \cdot x=x$
	\begin{align*}
		x + x \cdot y & = x \cdot 1 + x \cdot y &\mbox{, A2b}\\
		& = x \cdot (1 + y)&\mbox{, A3a}\\
		& = x \cdot 1&\mbox{, T2a} \\
		& = x &\mbox{, A2b}
	\end{align*}

	\item $T3b : x \cdot (x+y)=x$
	\begin{align*}
		x \cdot (x+y) & = (x+0) \cdot (x+y) &\mbox{, A2a}\\
		& = x + 0 \cdot y &\mbox{, A3b}\\
		& = x + 0 &\mbox{, T2b} \\
		& = x &\mbox{, A2a}
	\end{align*}

	\item $T4 : \overline{(\overline{x})}=x$
	\begin{align*}
		\overline{(\overline{x})} & = \overline{(\overline{x})} \cdot 1 &\mbox{, A2b}\\
		& = \overline{(\overline{x})} \cdot (x + \overline{x}) &\mbox{, A5a}\\
		& = \overline{(\overline{x})} \cdot x + \overline{(\overline{x})} \cdot \overline{x} &\mbox{, A3a}\\
		\mbox{En posant } y = \overline{x} : && \\
		& = \overline{(\overline{x})} \cdot x + 0 &\mbox{, A5b}\\
		& = \overline{(\overline{x})} \cdot x + x \cdot \overline{x} &\mbox{, A5b}\\
		& = x\cdot (\overline{(\overline{x})} + \overline{x})&\mbox{, A3b}\\
		\mbox{En posant } y = \overline{x} : && \\
		& = x \cdot 1 &\mbox{, A5a}\\
		& = x & \mbox{, A2b}
	\end{align*}

	\item $T5a : (x+y)+z = x + (y+z)$
	\begin{center}
		\begin{tabular}{|c|c|c|c|c|c|c|} \hline
		x & y & z & (x+y) & \textbf{(x+y)+z} & (y+z) & \textbf{x+(y+z)} \\ \hline
		0 & 0 & 0 & 0     & \textbf{0}       & 0     & \textbf{0} \\
		0 & 0 & 1 & 0     & \textbf{1}       & 1     & \textbf{1} \\
		0 & 1 & 0 & 1     & \textbf{1}       & 1     & \textbf{1} \\
		0 & 1 & 1 & 1     & \textbf{1}		 & 1	 & \textbf{1} \\
		1 & 0 & 0 & 1     & \textbf{1}		 & 1	 & \textbf{1} \\
		1 & 0 & 1 & 1	  & \textbf{1} 		 & 1 	 & \textbf{1} \\
		1 & 1 & 0 & 1	  & \textbf{1} 		 & 1	 & \textbf{1} \\
		1 & 1 & 1 & 1	  & \textbf{1}		 & 1 	 & \textbf{1} \\ \hline
		\end{tabular}
	\end{center}

	\item $T5b : x \cdot (y \cdot z) = (x \cdot y) \cdot z$
	\begin{center}
		\begin{tabular}{|c|c|c|c|c|c|c|} \hline
		x & y & z & $(y \cdot z)$ & \textbf{$x \cdot (y \cdot z)$} & $(x \cdot y)$ & \textbf{$(x \cdot y) \cdot z$} \\ \hline
		0 & 0 & 0 & 0           & \textbf{0}                   & 0           & \textbf{0} \\
		0 & 0 & 1 & 0           & \textbf{0}                   & 0           & \textbf{0} \\
		0 & 1 & 0 & 0           & \textbf{0}                   & 0           & \textbf{0} \\
		0 & 1 & 1 & 1           & \textbf{0}                   & 0           & \textbf{0} \\
		1 & 0 & 0 & 0           & \textbf{0}                   & 0           & \textbf{0} \\
		1 & 0 & 1 & 0           & \textbf{0}                   & 0           & \textbf{0} \\
		1 & 1 & 0 & 0           & \textbf{0}                   & 1           & \textbf{0} \\
		1 & 1 & 1 & 1           & \textbf{1}                   & 1           & \textbf{1} \\ \hline
		\end{tabular}
	\end{center}

	\item $T6a : \overline{(x+y)} = \overline{x} \cdot \overline{y}$
	\begin{center}
	\renewcommand{\arraystretch}{1.3}
		\begin{tabular}{|c|c|c|c|} \hline
		x & y & \textbf{$\overline{(x+y)}$} & \textbf{$\overline{x} \cdot \overline{y}$} \\ \hline
		0 & 0 & \textbf{1} & \textbf{1} \\
		0 & 1 & \textbf{0} & \textbf{0} \\
		1 & 0 & \textbf{0} & \textbf{0} \\
		1 & 1 & \textbf{0} & \textbf{0} \\ \hline
		\end{tabular}
	\end{center}

	\item $T6b : \overline{(x \cdot y)} = \overline{x} + \overline{y}$
	\begin{center}
	\renewcommand{\arraystretch}{1.3}
		\begin{tabular}{|c|c|c|c|} \hline
		x & y & \textbf{$\overline{(x \cdot y)}$} & \textbf{$\overline{x} + \overline{y}$} \\ \hline
		0 & 0 & \textbf{1} & \textbf{1} \\
		0 & 1 & \textbf{1} & \textbf{1} \\
		1 & 0 & \textbf{1} & \textbf{1} \\
		1 & 1 & \textbf{0} & \textbf{0} \\ \hline
		\end{tabular}
	\end{center}
\end{itemize}
}
\end{Q}

\end{document}
