\documentclass[11pt,a4paper]{article}
\usepackage[utf8]{inputenc}
\usepackage[T1]{fontenc}
\usepackage{amsthm} %numéroter les questions
\usepackage[frenchb]{babel}
\usepackage{datetime}
\usepackage{xspace} % typographie IN
\usepackage[hidelinks]{hyperref}% hyperliens
\usepackage[all]{hypcap} %lien pointe en haut des figures
\usepackage[french]{varioref} %voir x p y
\usepackage{fancyhdr}% en têtes
%\input cyracc.def
\usepackage{graphicx} %include pictures
\usepackage{pgfplots}

\usepackage{tikz}
\usetikzlibrary{calc,arrows,automata}
\usetikzlibrary{babel}
\usepackage{circuitikz}
% \usepackage{gnuplottex}
\usepackage{float}
\usepackage{ifthen}

\usepackage[top=1.3 in, bottom=1.3 in, left=1.3 in, right=1.3 in]{geometry} % Yeah, that's bad to play with margins
\usepackage[]{pdfpages}
\usepackage[]{attachfile}

\usepackage{amsmath}
\usepackage{amssymb} % checkmark
\usepackage{enumitem}
\setlist[enumerate]{label=\alph*)}% If you want only the x-th level to use this format, use '[enumerate,x]'
\usepackage[]{xcolor}
\usepackage{colortbl}
\usepackage{multirow}
\usepackage{bigdelim}%Braces in tabular

\usepackage{askmaps} % https://github.com/parastuffs/askmaps_custom

\newcommand\encircle[1]{%http://tex.stackexchange.com/questions/123924/indexed-letters-inside-circles
  \tikz[baseline=(X.base)] 
    \node (X) [draw, shape=circle, inner sep=0] {\strut #1};}


\newcommand\version{v1.4.0}

%cyr
\newcommand\textcyr[1]{{\fontencoding{OT2}\fontfamily{wncyr}\selectfont #1}}

%Numero du TP :
\def \tpnumber {TP 9 }

\newboolean{corrige}
\ifx\correction\undefined
\setboolean{corrige}{false}% pas de corrigé
\else
\setboolean{corrige}{true}%corrigé
\fi

% \setboolean{corrige}{true}% pas de corrigé

\newboolean{annexes}
\setboolean{annexes}{true}%annexes
%\setboolean{annexes}{false}% pas de annexes

\definecolor{darkblue}{rgb}{0,0,0.5}

\newboolean{mos}
%\setboolean{mos}{true}%annexes
\setboolean{mos}{false}% pas de annexes

\usepackage{aeguill} %guillemets

%% fancy header & foot
\pagestyle{fancy}
\lhead{[ELEC-H-305] Circuits logiques et numériques\\ \tpnumber}
\rhead{\version\\ page \thepage}
\chead{\ifthenelse{\boolean{corrige}}{Corrigé}{}}
\cfoot{}
%%

\pdfinfo{
/Author (Quentin Delhaye, ULB -- BEAMS)
/Title (\tpnumber, ELEC-H-305)
/ModDate (D:\pdfdate)
}

\hypersetup{
pdftitle={\tpnumber [ELEC-H-305] Choucroute logique et numérique},
pdfauthor={Quentin Delhaye, ULB -- BEAMS  },
pdfsubject={}
}

\theoremstyle{definition}% questions pas en italique
\newtheorem{Q}{Question}[] % numéroter les questions [section] ou non []

\newcommand{\reponse}[1]{% pour intégrer une réponse : \reponse{texte} : sera inclus si \boolean{corrige}
	\ifthenelse {\boolean{corrige}} {\paragraph{Réponse :} \color{darkblue}   #1\color{black}} {}
 }

\newcommand{\addcontentslinenono}[4]{\addtocontents{#1}{\protect\contentsline{#2}{#3}{#4}{}}}

\date{}
\title{\vspace{-2cm}Circuits logiques et numériques [ELEC-H-305] \\  \tpnumber Synthèse de circuits \ifthenelse{\boolean{corrige}}{-- Corrigé}{} \\ \small{\version}}

\setlength{\parskip}{0.2cm plus2mm minus1mm} %espacement entre §
\setlength{\parindent}{0pt}

\newlength{\gvs}% Gate Vertical Space
\gvs=6em
\newlength{\ghs}% Gate Horizontal Space
\ghs=10em

\begin{document}
% \pagestyle{empty}
\maketitle
\vspace*{-1cm}



\begin{Q}
Un système de traitement d'image fournit des points (pixels) à quatre niveaux de gris~: $ab = 00, 01, 10, 11$ (respectivement noir, foncé, clair et blanc).
On veut construire un automate qui reçoit les valeurs de $ab$ en séquence, ligne après ligne, et met à $1$ une sortie $Z$ lorsqu'il détecte la fin d'une rampe montante ou descendante complète~: $00-01-10-11$ ou $11-10-01-00$.
On fera l'hypothèse simplificatrice que deux pixels successifs n'auront jamais la même valeur.

Dresser la table de Huffman de ce problème.

\reponse{
	\begin{center}
		\begin{tabular}{|l|l|l|l|l|l|} \hline
			& \multicolumn{4}{c}{$ab$} & \\ \hline
			& 00 & 01 & 11 & 10 & $Z$ \\ \hline
			1 & 2 & \textbf{1} & 6 & \textbf{1} & 0 \\ \hline
			2 & \textbf{2} & 3 & 1 & 1 & 0 \\ \hline
			3 & 1 & \textbf{3} & 1 & 4 & 0 \\ \hline
			4 & 1 & 1 & 5 & \textbf{4} & 0 \\ \hline
			5 & 1 & 1 & \textbf{5} & 7 & 1 \\ \hline
			6 & 1 & 1 & \textbf{6} & 7 & 0 \\ \hline
			7 & 1 & 8 & 1 & \textbf{7} & 0 \\ \hline
			8 & 9 & \textbf{8} & 1 & 1 & 0 \\ \hline
			9 & \textbf{9} & 3 & 1 & 1 & 1 \\ \hline
		\end{tabular}
	\end{center}
	%TODO Table des conditions d'équivalence
}
\end{Q}

% \begin{Q}
% Dans la table des états de Huffman suivante~:
% \begin{center}
% 	\begin{tabular}{|l|l|l|l|l|l|} \hline
% 		& \multicolumn{4}{c}{$ab$} & \\ \hline
% 		& 00 & 01 & 11 & 10 & $Z$ \\ \hline
% 		1 & \textbf{1} & \textbf{1} & 3 & - & 0 \\ \hline
% 		2 & \textbf{2} & \textbf{2} & 3 & 4 & 1 \\ \hline
% 		3 & 2 & 1 & \textbf{3} & \textbf{3} & 1 \\ \hline
% 		4 & 1 & - & \textbf{4} & \textbf{4} & 0 \\ \hline
% 	\end{tabular}
% \end{center}
% \begin{enumerate}
% 	\item Trouver un codage n'offrant pas de problème de course.
% 	Déterminer les fonctions de rétroactions $Y_1$ et $Y_2$.

% 	\item Résoudre les problèmes de course critique si le codage~: $1=00$, $2=01$, $3=11$ et $4=10$ est imposé.
% 	Donner les expressions pour les nouveaux $Y_1$ et $Y_2$.
% \end{enumerate}
% \end{Q}


\begin{Q}
	Deux loutres mutantes attaquent Metropolis.
	Superman est le seul à pouvoir sauver la ville.
	Les terribles mustélidés sont cependant particulièrement résistants et Superman décide de quérir l'aide du Dr Hamilton pour les vaincre.
	Ce dernier lui donne alors une table de Huffman sur la sequence de pouvoirs à utiliser pour triompher ($Z_1$).
	Cependant, il y a aussi introduit une séquence qui rendrait les loutres immortelles et leur permettrait de cracher du feu ($Z_2$).
	La notice suivante est jointe~:
	\begin{verse}
		$a$~: vision laser~; $b$~: souffle glacé~; $Z_1$~: loutres vaincues~; $Z_2$~: loutres invulnérables.
	\end{verse}

	Que doit faire Superman pour sauver Metropolis des loutres infernales~?

	\begin{center}
		\begin{tabular}{|c|c|c|c|c|c|c|} \hline
			$Y_1Y_2$& \multicolumn{4}{c|}{$ab$} & & \\ \hline
			$y_1y_2$& 00 & 01 & 11 & 10 & $Z_1$ & $Z_2$\\ \hline
			1 & \textbf{1} & 2 & - & 5 & 0 & 0 \\ \hline
			2 & 1 & \textbf{2} & - & 3 & 0 & 0 \\ \hline
			3 & 1 & 4 & - & \textbf{3} & 0 & 0 \\ \hline
			4 & \textbf{4} & \textbf{4} & \textbf{4} & \textbf{4} & 1 & 0 \\ \hline
			5 & 1 & 6 & - & \textbf{5} & 0 & 0\\ \hline
			6 & 1 & \textbf{6} & - & 7 & 0 & 0 \\ \hline
			7 & \textbf{7} & \textbf{7} & \textbf{7} & \textbf{7} & 0 & 1 \\ \hline
		\end{tabular}
	\end{center}

	\reponse{
		Si Superman veut éliminer ces perfides créatures, il devra d'abord les geler à l'aide de son souffle, puis les brûler avec son regard de Kryptonien et finalement les geler à nouveau.
	}
\end{Q}

\begin{Q}
	Selon une légende créée spécialement pour ce TP, jaloux du succès du code Konami\footnote{$\uparrow \uparrow \downarrow \downarrow \leftarrow \rightarrow \leftarrow \rightarrow$ \encircle{B} \encircle{A}}, Nintendo aurait implémenté un code similaire dans ses jeux développés sur NES.
	Le «~code Nintendo~» serait cependant plus simple~: pour activer une fonctionnalité cachée du jeu, il suffirait d'appuyer sur le bouton \encircle{A}, puis d'appuyer sur \encircle{B} en maintenant \encircle{A} enfoncé et enfin d'appuyer sur \framebox{\textsc{select}} toujours en maintenant les deux premiers boutons enfoncés.
	Si l'un des boutons est relâché ou enfoncé dans le mauvais ordre, la séquence est annulée.

	Dresser le circuit logique du code permettant d'activer le bonus.

	\reponse{
		Afin de pouvoir créer le circuit logique, il convient d'abord de remplir la table de Huffman décrivant la séquence de boutons, et en dériver les équations logiques.

		En considérant que la touche \framebox{select} est représentée par la variable $c$, on obtient~:
		\begin{center}
			\begin{tabular}{|c|c|c|c|c|c|c|c|c|c|}\hline
			$Y_1Y_2$ & \multicolumn{8}{c|}{$abc$} & \\ \hline
			$y_1y_2$ & 000 & 001 & 011 & 010 & 100 & 101 & 111 & 110 & Z \\ \hline
			00 & \textbf{00} & \textbf{00} & \textbf{00} & \textbf{00} & 01 & \textbf{00} & \textbf{00} & \textbf{00} & 0 \\ \hline
			01 & 00 & 00 & 00 & 00 & \textbf{01} & 00 & 00 & 11 & 0 \\ \hline
			11 & 00 & 00 & 00 & 00 & 00 & 00 & 10 & \textbf{11} & 0 \\ \hline
			10 & 00 & 00 & 00 & 00 & 00 & 00 & \textbf{10} & 00 & 1 \\ \hline
			\end{tabular}
		\end{center}

		On en déduit les k-maps suivantes~:
		\begin{center}
			\askmapv{$Y_1 = abcy_1 + ab\overline{c}y_2$}{a b c $y_1$ $y_2$}{}{00000000000000000000000001010011}{%
			\color{red}\put(6.1,0.1){\dashbox{0.1}(0.8,1.8){}}%
			\color{green}\put(7.1,1.1){\dashbox{0.2}(0.8,1.8){}}%
			}

			\askmapv{$Y_2 = a\overline{bcy_1} + ab\overline{c}y_2$}{a b c $y_1$ $y_2$}{}{00000000000000001100000001010000}{%
			\color{red}\put(4.1,2.1){\dashbox{0.1}(0.8,1.8){}}%
			\color{green}\put(7.1,1.1){\dashbox{0.2}(0.8,1.8){}}%
			}

			\askmapv{$Z = y_1\overline{y_2}$}{a b c $y_1$ $y_2$}{}{00-000-000-000-000-000-000-0001-}{%
			\color{red}\put(0.1,0.1){\dashbox{0.1}(7.8,0.8){}}%
			}
		\end{center}
			\begin{circuitikz}[scale=0.8, every node/.style={scale=0.8}]
			% \begin{circuitikz}
				\draw
				(0.5\ghs,0.5\gvs) node[not port, rotate=90] (mynotb) {}
				(1\ghs,0.5\gvs) node[not port, rotate=90] (mynotc) {}
				(1.5\ghs,0.5\gvs) node[not port, rotate=90] (mynoty1) {}
				(2\ghs,0.5\gvs) node[not port, rotate=90] (mynoty2) {}
				(2.5\ghs,1.5\gvs) node[and port] (myand11) {}%bottom left
				(2.5\ghs,2.3\gvs) node[and port] (myand12) {}
				(2.5\ghs,3.8\gvs) node[and port] (myand13) {}
				(2.5\ghs,4.6\gvs) node[and port] (myand14) {}
				(2.5\ghs,5.4\gvs) node[and port] (myand15) {}%top left

				(3.5\ghs,1.9\gvs) node[and port] (myand21) {}
				(3.5\ghs,2.7\gvs) node[and port] (myand22) {}
				(3.5\ghs,3.8\gvs) node[and port] (myand23) {}
				(3.5\ghs,5.0\gvs) node[and port] (myand24) {}

				(4.2\ghs,2.3\gvs) node[or port] (myory2) {}% Bottom OR, Y2
				(4.2\ghs,4.4\gvs) node[or port] (myory1) {}% Y1

				(2.5\ghs,6.5\gvs) node[and port] (myandz) {}
				(0,7\gvs) node[shape=coordinate] (top) {}
				(0.0,0.0) node[shape=coordinate] (bottoma) {}%Bottom of line A

				(bottoma |- myand12.in 1) node[shape=coordinate] (dot11) {}
				(bottoma |- myand14.in 1) node[shape=coordinate] (dot12) {}
				($(mynotb.in)-(0.2\ghs,0)$) node[shape=coordinate] (dot21) {}
				(dot21 |- myand14.in 2) node[shape=coordinate] (dot22) {}
				(mynotb |- myand12.in 2) node[shape=coordinate] (dot31) {}
				($(mynotc.in)-(0.2\ghs,0)$) node[shape=coordinate] (dot41) {}
				(dot41 |- myand15.in 2) node[shape=coordinate] (dot42) {}
				(mynotc |- myand11.in 2) node[shape=coordinate] (dot51) {}
				(dot51 |- myand13.in 2) node[shape=coordinate] (dot52) {}
				($(mynoty1.in)-(0.2\ghs,0)$) node[shape=coordinate] (dot61) {}
				(dot61 |- myand15.in 1) node[shape=coordinate] (dot62) {}
				(dot61 |- myandz.in 2) node[shape=coordinate] (dot63) {}
				(mynoty1 |- myand11.in 1) node[shape=coordinate] (dot71) {}
				($(mynoty2.in)-(0.2\ghs,0)$) node[shape=coordinate] (dot81) {}
				(dot81 |- myand13.in 1) node[shape=coordinate] (dot82) {}
				(mynoty2 |- myandz.in 1) node[shape=coordinate] (dot91) {}


				% A
				(bottoma) to[short, -*] (dot11) to[short, -*] (dot12) {} -- (bottoma |- top)
				(bottoma) -- +(0, -0.2\gvs)
				(dot11) -- (myand12.in 1)
				(dot12) -- (myand14.in 1)

				% B
				(dot21) -- +(0,-0.4\gvs)
				(dot21) to[short, *-*] (dot22) -- (dot21 |- top)
				(dot22) -- (myand14.in 2)

				% B'
				(mynotb.out) to[short, -*] (dot31) -- (mynotb |- top)
				(mynotb.in) -- (dot21)
				(dot31) -- (myand12.in 2)

				% C
				(dot41) -- +(0,-0.4\gvs)
				(dot41) to[short, *-*] (dot42) -- (dot41 |- top)
				(dot42) -- (myand15.in 2)

				% C'
				(mynotc.out) to[short, -*] (dot51) to[short, -*] (dot52) -- (mynotc |- top)
				(mynotc.in) -- (dot41)
				(dot51) -- (myand11.in 2)
				(dot52) -- (myand13.in 2)

				% Y1
				(dot61) -- +(0,-0.4\gvs)
				(dot61) to[short, *-*] (dot62) to[short, -*] (dot63) -- (dot61 |- top)
				(dot62) -- (myand15.in 1)
				(dot63) -- (myandz.in 2)

				% Y1'
				(mynoty1.out) to[short, -*] (dot71) -- (mynoty1 |- top)
				(mynoty1.in) -- (dot61)
				(dot71) -- (myand11.in 1)

				% Y2
				(dot81) -- +(0,-0.2\gvs)
				(dot81) to[short, *-*] (dot82) -- (dot81 |- top)
				(dot82) -- (myand13.in 1)

				% Y2'
				(mynoty2.out) to[short, -*] (dot91) -- (mynoty2 |- top)
				(mynoty2.in) -- (dot81)
				(dot91) -- (myandz.in 1)
				
				% AND21
				(myand11.out) -| (myand21.in 2)
				(myand12.out) -| (myand21.in 1)

				% AND22
				(myand23.in 2) to[short, *-] (myand22.in 1)
				(myand14.out) to[short, *-] (myand14.out |- myand23.in 1) to[short, *-] (myand14.out |- myand22.in 2) -- (myand22.in 2)

				% AND23
				(myand14.out |- myand23.in 1) -- (myand23.in 1)
				(myand13.out) -| (myand23.in 2)

				% AND24
				(myand14.out) -| (myand24.in 2)
				(myand15.out) -| (myand24.in 1)

				% OR Y2
				(myand21.out) -| (myory2.in 2)
				(myand22.out) -| (myory2.in 1)
				(myory2.out) |- ($(dot81)-(0,0.2\gvs)$)

				% OR Y1
				(myand23.out) -| (myory1.in 2)
				(myand24.out) -| (myory1.in 1)
				(myory1.out) -- +(0.2\gvs,0) |- ($(dot61)-(0,0.4\gvs)$)

				% Z
				(myandz.out) -- +(0.2\ghs,0)
				(myandz.out) node[anchor=south west] {\Large $Z$}
		
				% Labels
				($(bottoma)+(0,0.15\gvs)$) node[anchor=north east] {\Large $a$}
				(dot21) node[anchor=north east] {\Large $b$}
				(dot41) node[anchor=north east] {\Large $c$}
				(dot61) node[anchor=north east] {\Large $y_1$}
				(dot81) node[anchor=north east] {\Large $y_2$}
				(myory1.out) node[anchor=south west] {\Large $Y_1$}
				($(myory2.out)-(0.1\ghs,0)$) node[anchor=south west] {\Large $Y_2$}
				;
			\end{circuitikz}
	}

\end{Q}	



% C'est beaucoup trop long, c'est complètement débile.
% \begin{Q}
% Construire la table de conditions d'équivalence pour la table de Huffman suivante~:
% \begin{center}
% 	\begin{tabular}{llllll}
% 	\end{tabular}
% \end{center}
% \reponse{
% }
% \end{Q}





\begin{Q}
Dans la table des états de Huffman suivante~:
\begin{center}
	\begin{tabular}{|c|c|c|c|c|c|}\hline
	& \multicolumn{4}{c|}{$ab$} & \\ \hline
	  & 00 & 01 & 11 & 10 & Z \\ \hline
	1 & 1 & 1 & 3 & - & 0 \\ \hline
	2 & 2 & 2 & 3 & 4 & 1 \\ \hline
	3 & 2 & 1 & 3 & 3 & 1 \\ \hline
	4 & 1 & - & 4 & 4 & 0 \\ \hline
	\end{tabular}
\end{center}
\begin{enumerate}
	\item Trouver un codage n'offrant pas de problème de course.
	Déterminer les fonctions de rétroactions $Y_1$ et $Y_2$.
	\reponse{

		Un problème de course survient lorsque la machine doit passer dans un nouvel état dans lequel plus d'une fonction de rétroaction change.
		Par exemple, lorsque l'on passe de l'état $00$ à l'état $11$, $y_1$ et $y_2$ changent de valeur.
		Or, il se peut que l'un des deux signaux soit plus lent que l'autre, entraînant une transition vers un état non désiré.

		Pour éviter ces problèmes, il suffit parfois de choisir un codage des états approprié.
		Dans notre cas, si l'on choisit de coder l'état $1$ par $00$, il faut obligatoirement coder l'état $3$ par un codage ayant une distance de Hamming de 1 à cause de la transition $1 \rightarrow 3$.
		Choisissons de coder $3$ par $01$.

		En suivant la même réflexion pour les deux autres états, on peut trouver le codage suivant~:
		1 $\rightarrow$ 00, 2 $\rightarrow$ 11, 3 $\rightarrow$ 10, 4 $\rightarrow$ 01.

		\begin{center}
			\begin{tabular}{|c|c|c|c|c|c|}\hline
			$Y_1Y_2$ & \multicolumn{4}{c|}{$ab$} & \\ \hline
			$y_1y_2$ & 00 & 01 & 11 & 10 & Z \\ \hline
			00       & 00 & 00 & 10 & -  & 0 \\ \hline
			11       & 11 & 11 & 10 & 01 & 1 \\ \hline
			10       & 11 & 00 & 10 & 10 & 1 \\ \hline
			01       & 00 & -  & 01 & 01 & 0 \\ \hline
			\end{tabular}
		\end{center}

		Et en réarrangeant la table pour qu'on puisse directement en extraire les K-maps~:
		\begin{center}
			\begin{tabular}{|c|c|c|c|c|c|}\hline
			$Y_1Y_2$ & \multicolumn{4}{c|}{$ab$} & \\ \hline
			$y_1y_2$ & 00 & 01 & 11 & 10 & Z \\ \hline
			00       & 00 & 00 & 10 & -  & 0 \\ \hline
			01       & 00 & -  & 01 & 01 & 0 \\ \hline
			11       & 11 & 11 & 10 & 01 & 1 \\ \hline
			10       & 11 & 00 & 10 & 10 & 1 \\ \hline
			\end{tabular}
		\end{center}

		Les K-maps et fonctions logiques correspondantes sont~:

		\begin{center}
			\askmapiv{$Y_1 = a\overline{y_1}y_2 + a\overline{b}y_2 + \overline{a}y_1y_2 + \overline{ab}y_1$}{a b $y_1$ $y_2$}{}{00110-01-1010100}{}
		\end{center}

		\begin{center}
			\askmapiv{$Y_2 = a\overline{y_2} + \overline{ab}y_1 + by_1y_2$}{a b $y_1$ $y_2$}{}{00110-01-0101011}{}
		\end{center}

		\begin{center}
			\askmapiv{$Z = y_1$}{a b $y_1$ $y_2$}{}{00110--1-01--011}{}
		\end{center}
	}
	\item Résoudre les problèmes de course critique si le codage est $1=00, 2=01, 3=11, 4=10$.
	Donner les expressions des nouveaux $Y_1$ et $Y_2$.
	\reponse{
		Les cases en {\color{red}rouge} sont celle présentant des problèmes de course.

		\begin{center}
			\begin{tabular}{|c|c|c|c|c|c|}\hline
			$Y_1Y_2$ & \multicolumn{4}{c|}{$ab$} & \\ \hline
			$y_1y_2$ & 00 & 01 & 11 & 10 & Z \\ \hline
			00       & 00 & 00 & {\color{red}11} & -  & 0 \\ \hline
			01       & 01 & 01 & 11 & {\color{red}10} & 1 \\ \hline
			11       & 01 & {\color{red}00} & 11 & 11 & 1 \\ \hline
			10       & 00 & -  & 10 & 10 & 0 \\ \hline
			\end{tabular}
		\end{center}


		Nous avons maintenant plusieurs problèmes de course, par exemple lorsque l'on se trouve dans l'état $00$ avec les entrées $01$ et que ces dernières passent à $11$, la machine nous envoie à l'état $11$, faisant donc changer les deux variables de rétroaction simultanément.

		Nous pouvons répertorier tous les problèmes de course~:
		\begin{enumerate}
			\item Transition de l'état $00$ à l'état $11$ lorsque les entrées passent de $01$ à $11$.
			\item Transition de l'état $01$ à l'état $10$ lorsque les entrées passent de $00$ à $10$.
			\item Transition de l'état $11$ à l'état $00$ lorsque les entrées passent de $11$ à $01$.
		\end{enumerate}

		Puisque le codage est imposé, nous allons utiliser les \textit{don't care} et les états instables pour régler ces problèmes en effectuant des transitions intermédiaires.
		En illustrant avec des tables de Huffman partielles~:
		\begin{enumerate}
			\item 
			% \begin{center}
				\begin{tabular}{|c|c|c|}\hline
				$Y_1Y_2$ & 01 & 11 \\ \hline
				00 & \textbf{00} & 11 \\ \hline
				01 & \textbf{01} & 11 \\ \hline
				11 & 00 & \textbf{11} \\ \hline
				\end{tabular}
				devient
				\begin{tabular}{|c|c|c|}\hline
				$Y_1Y_2$ & 01 & 11 \\ \hline
				00 & \textbf{00} & \color{green}{01} \\ \hline
				01 & \textbf{01} & 11 \\ \hline
				11 & 00 & \textbf{11} \\ \hline
				\end{tabular}
			% \end{center}

			\item
			\begin{tabular}{|c|c|c|}\hline
				$Y_1Y_2$ & 00 & 10 \\ \hline
				00 & \textbf{00} & - \\ \hline
				01 & \textbf{01} & 10 \\ \hline
				10 & 00 & \textbf{10} \\ \hline
			\end{tabular}
			devient
			\begin{tabular}{|c|c|c|}\hline
				$Y_1Y_2$ & 00 & 10 \\ \hline
				00 & \textbf{00} & \color{green}{10} \\ \hline
				01 & \textbf{01} & \color{green}{00} \\ \hline
				10 & 00 & \textbf{10} \\ \hline
			\end{tabular}

			\item
			\begin{tabular}{|c|c|c|}\hline
				$Y_1Y_2$ & 01 & 11 \\ \hline
				00 & \textbf{00} & 01 \\ \hline
				11 & 00 & \textbf{11} \\ \hline
				10 & - & \textbf{10} \\ \hline
			\end{tabular}
			devient
			\begin{tabular}{|c|c|c|}\hline
				$Y_1Y_2$ & 01 & 11 \\ \hline
				00 & \textbf{00} & 01 \\ \hline
				11 & \color{green}{10} & \textbf{11} \\ \hline
				10 & \color{green}{00} & \textbf{10} \\ \hline
			\end{tabular}
		\end{enumerate}

		On obtient donc une nouvelle table de Huffman sans condition de course~:
		\begin{center}
			\begin{tabular}{|c|c|c|c|c|c|}\hline
			$Y_1Y_2$ & \multicolumn{4}{c|}{$ab$} & \\ \hline
			$y_1y_2$ & 00 & 01 & 11 & 10 & Z \\ \hline
			00       & 00 & 00 & {\color{green}01} & {\color{green}10} & 0 \\ \hline
			01       & 01 & 01 & 11 & {\color{green}00} & 1 \\ \hline
			11       & 01 & {\color{green}10} & 11 & 11 & 1 \\ \hline
			10       & 00 & {\color{green}00} & 10 & 10 & 0 \\ \hline
			\end{tabular}
		\end{center}

		Ce qui donne les fonctions logiques~:

		\begin{center}
			\askmapiv{$Y_1 = ay_1 + by_1y_2 + aby_2 + a\overline{by_2}$}{a b $y_1$ $y_2$}{}{0000000110110111}{}
		\end{center}
		
		\begin{center}
			\askmapiv{$Y_2 = \overline{ab}y_2 + b\overline{y_1}y_2 + ay_1y_2 + ab\overline{y_1}$}{a b $y_1$ $y_2$}{}{0101010000011101}{}
		\end{center}

		\begin{center}
			\askmapiv{$Z = y_2$}{a b $y_1$ $y_2$}{}{0101010-0-01-101}{}
		\end{center}

	}
\end{enumerate}
\end{Q}






\begin{Q}
Réduire la table des états primitive suivante pour qu'elle ne contienne plus que quatre états.
Coder les états de la table réduite en ajoutant les éventuels transitoires pour éviter les courses critiques.
Calculer les fonctions de rétroaction $Y_1$ et $Y_2$.

\begin{center}
	\begin{tabular}{|c|c|c|c|c|c|}\hline
	 & \multicolumn{4}{c|}{$ab$} & \\ \hline
	 & 00 & 01 & 11 & 10 & Z \\ \hline
	1 & 1 & 2 & 3 & 4 & 0 \\ \hline
	2 & 1 & 2 & 3 & 5 & 1 \\ \hline
	3 & 1 & 6 & 3 & 5 & 0 \\ \hline
	4 & - & 2 & 7 & 4 & 0 \\ \hline
	5 & 1 & 2 & 3 & 5 & 1 \\ \hline
	6 & 1 & 6 & 3 & 5 & 0 \\ \hline
	7 & 8 & 2 & 7 & - & 0 \\ \hline
	8 & 8 & 2 & 7 & 4 & 0 \\ \hline
	\end{tabular}
\end{center}
\reponse{
\begin{center}
			\begin{tabular}{|c|c|c|c|c|c|c|c|c|c|c|c|} \hline%\cellcolor{green!25}
				2 & x & \cellcolor{gray!20} & \cellcolor{gray!20} & \cellcolor{gray!20} & \cellcolor{gray!20} & \cellcolor{gray!20} & \cellcolor{gray!20} \\ \hline
				\multirow{2}{*}{3} & \multirow{2}{*}{\cellcolor{red!25}} & \multirow{2}{*}{x} & \multirow{2}{*}{\cellcolor{gray!20}} & \multirow{2}{*}{\cellcolor{gray!20}} & \multirow{2}{*}{\cellcolor{gray!20}} & \multirow{2}{*}{\cellcolor{gray!20}} & \multirow{2}{*}{\cellcolor{gray!20}} \\
				 & \cellcolor{red!25}2-6 & & \cellcolor{gray!20} & \cellcolor{gray!20} & \cellcolor{gray!20} & \cellcolor{gray!20} & \cellcolor{gray!20} \\ \hline
				\multirow{3}{*}{4} & \multirow{3}{*}{\cellcolor{red!25}} & \multirow{3}{*}{x} & \cellcolor{red!25}2-6 & \multirow{3}{*}{\cellcolor{gray!20}} & \multirow{3}{*}{\cellcolor{gray!20}} & \multirow{3}{*}{\cellcolor{gray!20}} & \multirow{3}{*}{\cellcolor{gray!20}} \\
				 & \cellcolor{red!25}3-7 & & \cellcolor{red!25}3-7 & \cellcolor{gray!20} & \cellcolor{gray!20} & \cellcolor{gray!20} & \cellcolor{gray!20} \\
				 & \cellcolor{red!25} & & \cellcolor{red!25}4-5 & \cellcolor{gray!20} & \cellcolor{gray!20} & \cellcolor{gray!20} & \cellcolor{gray!20} \\ \hline
				5 & x & \cellcolor{green!25}OK & x & x & \cellcolor{gray!20} & \cellcolor{gray!20} & \cellcolor{gray!20} \\ \hline
				\multirow{3}{*}{6} & \cellcolor{red!25}2-6 & \multirow{3}{*}{x} & \cellcolor{green!25} & \cellcolor{red!25}2-6 & \multirow{3}{*}{x} & \cellcolor{gray!20} &\cellcolor{gray!20} \\
				 & \cellcolor{red!25}4-5 &  & \cellcolor{green!25}OK & \cellcolor{red!25}3-7 &  & \cellcolor{gray!20} &  \cellcolor{gray!20}\\
				 & \cellcolor{red!25} &  & \cellcolor{green!25} & \cellcolor{red!25}4-5 &  & \cellcolor{gray!20} &  \cellcolor{gray!20}\\ \hline
				\multirow{3}{*}{7} & \cellcolor{red!25}1-8; & \multirow{3}{*}{x} & \cellcolor{red!25}1-8 & \cellcolor{green!25} & \multirow{3}{*}{x} & \cellcolor{red!25}1-8 & \cellcolor{gray!20} \\
				 & \cellcolor{red!25}3-7 &  & \cellcolor{red!25}6-2 & \cellcolor{green!25}OK &  & \cellcolor{red!25}2-6 & \cellcolor{gray!20} \\
				 & \cellcolor{red!25} & & \cellcolor{red!25} & \cellcolor{green!25} &  & \cellcolor{red!25}4-5 & \cellcolor{gray!20} \\ \hline
				\multirow{4}{*}{8} & \cellcolor{red!25} & \multirow{4}{*}{x} & \cellcolor{red!25}1-8  & \cellcolor{green!25}  & \multirow{4}{*}{x}  & \cellcolor{red!25}1-8 & \cellcolor{green!25} \\
				& \cellcolor{red!25}7-3 &  & \cellcolor{red!25}6-2 & \cellcolor{green!25}OK & & \cellcolor{red!25}2-6 & \cellcolor{green!25} OK \\
				& \cellcolor{red!25} &  & \cellcolor{red!25}3-7 & \cellcolor{green!25} & & \cellcolor{red!25}3-7 & \cellcolor{green!25} \\
				& \cellcolor{red!25} &  & \cellcolor{red!25}4-5 & \cellcolor{green!25} & & \cellcolor{red!25}4-5 & \cellcolor{green!25} \\ \hline
				% \multirow{2}{*}{9} & \multirow{2}{*}{x} & \multirow{2}{*}{x} & \cellcolor{red!25}6-8; & \multirow{2}{*}{x} & \cellcolor{green!25}7-11; & \multirow{2}{*}{x} & \multirow{2}{*}{x} & \multirow{2}{*}{x} & \cellcolor{gray!20} &  \cellcolor{gray!20}& \cellcolor{gray!20}\\
				%  &  &  & \cellcolor{red!25}4-12 &  & \cellcolor{green!25} &  &  &  &  \cellcolor{gray!20}&  \cellcolor{gray!20}&\cellcolor{gray!20}\\ \hline
				% \multirow{3}{*}{10} & \multirow{3}{*}{x} & \multirow{3}{*}{x} & \multirow{3}{*}{x} & \cellcolor{green!25}7-11; & \multirow{3}{*}{x} & \multirow{3}{*}{x} & \multirow{3}{*}{x} & \multirow{3}{*}{x} & \multirow{3}{*}{x} & \cellcolor{gray!20} & \cellcolor{gray!20}\\
				%  &  &  &  & \cellcolor{green!25}5-9; &  &  &  &  &  & \cellcolor{gray!20} & \cellcolor{gray!20}\\
				%  &  &  &  & \cellcolor{green!25}2-6; &  &  &  &  &  & \cellcolor{gray!20} & \cellcolor{gray!20}\\ \hline
				% \multirow{2}{*}{11} & \cellcolor{red!25}2-8; & \multirow{2}{*}{x} & \multirow{2}{*}{x} & \multirow{2}{*}{x} & \multirow{2}{*}{x} & \multirow{2}{*}{x} & \cellcolor{green!25}5-9; & \multirow{2}{*}{x} & \multirow{2}{*}{x} & \multirow{2}{*}{x} & \cellcolor{gray!20}\\
				%  & \cellcolor{red!25}3-9 &  &  &  &  &  & \cellcolor{green!25}4-10; &  &  &  & \cellcolor{gray!20}\\ \hline
				% 12 & \multirow{4}{*}{x} & \cellcolor{red!25}5-9; & \multirow{4}{*}{x} & \multirow{4}{*}{x} & \multirow{4}{*}{x} & \cellcolor{red!25}11-7; & \multirow{4}{*}{x} & \cellcolor{red!25}1-7; & \multirow{4}{*}{x} & \multirow{4}{*}{x} & \multirow{4}{*}{x} \\
				%  &  & \cellcolor{red!25}4-12 &  &  &  & \cellcolor{red!25}10-12; &  & \cellcolor{red!25}2-8; &  &  &  \\
				%  &  & \cellcolor{red!25} &  &  &  & \cellcolor{red!25}2-6 &  & \cellcolor{red!25}3-5; &  &  &  \\
				%  &  & \cellcolor{red!25} &  &  &  & \cellcolor{red!25} &  & \cellcolor{red!25}4-12 &  &  &  \\ \hline
				 & 1 & 2 & 3 & 4 & 5 & 6 & 7 \\ \hline
			\end{tabular}
		\end{center}

		On peut donc fusionner les états $5 \rightarrow 2$, $6 \rightarrow 3$, $7 \rightarrow 4$ et $8 \rightarrow 4$, ce qui nous donne bien quatre états.

		La table de Huffman devient~:
		\begin{center}
			\begin{tabular}{|c|c|c|c|c|c|}\hline
			 & \multicolumn{4}{c|}{$ab$} & \\ \hline
			 & 00 & 01 & 11 & 10 & Z \\ \hline
			1 & 1 & 2 & 3 & 4 & 0 \\ \hline
			2 & 1 & 2 & 3 & 2 & 1 \\ \hline
			3 & 1 & 3 & 3 & 2 & 0 \\ \hline
			4 & 4 & 2 & 4 & 4 & 0 \\ \hline
			\end{tabular}
		\end{center}

		En adoptant le codage $1 \rightarrow 00$, $2 \rightarrow 01$, $3 \rightarrow 11$ et $4 \rightarrow 10$, on obtient la table de Huffman codée~:
		\begin{center}
			\begin{tabular}{|c|c|c|c|c|c|}\hline
			$Y_1Y_2$ & \multicolumn{4}{c|}{$ab$} & \\ \hline
			$y_1y_2$ & 00 & 01 & 11 & 10 & Z \\ \hline
			00 & 00 & 01 & {\color{red}11} & 10 & 0 \\ \hline
			01 & 00 & 01 & 11 & 01 & 1 \\ \hline
			11 & {\color{red}00} & 11 & 11 & 01 & 0 \\ \hline
			10 & 10  & {\color{red}01} & 10 & 10 & 0 \\ \hline
			\end{tabular}
		\end{center}

		Les cases mises en évidence en rouge présentent des problèmes de course critique.
		Il est possible de régler ces problèmes avec les ajustements suivants~:
		\begin{center}
			\begin{tabular}{|c|c|c|c|c|c|}\hline
			$Y_1Y_2$ & \multicolumn{4}{c|}{$ab$} & \\ \hline
			$y_1y_2$ & 00 & 01 & 11 & 10 & Z \\ \hline
			00 & 00 & 01 & {\color{green}01} & 10 & 0 \\ \hline
			01 & 00 & 01 & 11 & 01 & 1 \\ \hline
			11 & {\color{green}01} & 11 & 11 & 01 & 0 \\ \hline
			10 & 10  & {\color{green}00} & 10 & 10 & 0 \\ \hline
			\end{tabular}
		\end{center}

		De cette table, découlent les fonctions de rétroaction $Y_1$ et $Y_2$ ainsi que la fonction de sortie $Z$.

		\begin{center}
			\askmapiv{$Y_1 = \overline{b}y_1\overline{y_2} + a\overline{by_2} + by_1y_2 + aby_2 + aby_1$}{a b $y_1$ $y_2$}{}{0010000110100111}{}
		\end{center}

		\begin{center}
			\askmapiv{$Y_2 = y_1y_2 + b\overline{y_1} + ay_2$}{a b $y_1$ $y_2$}{}{0001110101011101}{}
		\end{center}

		\begin{center}
			\askmapiv{$Z = \overline{a}b\overline{y_1} + a\overline{b}y_2$}{a b $y_1$ $y_2$}{}{0000-100010-0000}{}
		\end{center}

		Attention, les cases $y_1y_2ab = 1100$ et $0100$ devraient être un « $-$ » d'après la table de Huffman.
		Cependant, ces deux cases sont des états instables et lors de la transition $11(01) \rightarrow 01(00) \rightarrow 00(00)$\footnote{En utilisant la notation « état présent(état futur) ».}, la sortie doit rester à $0$.

		Il en va de même pour les cases $0011$ et $0111$.


}
\end{Q}


\end{document}
