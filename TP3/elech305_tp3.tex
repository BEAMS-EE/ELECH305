\documentclass[11pt,a4paper]{article}
\usepackage[utf8]{inputenc}
\usepackage[T1]{fontenc}
\usepackage{amsthm} %numéroter les questions
\usepackage[frenchb]{babel}
\usepackage{datetime}
\usepackage{xspace} % typographie IN
\usepackage[hidelinks]{hyperref}% hyperliens
\usepackage[all]{hypcap} %lien pointe en haut des figures
\usepackage[french]{varioref} %voir x p y
\usepackage{fancyhdr}% en têtes
%\input cyracc.def
\usepackage{graphicx} %include pictures
\usepackage{pgfplots}

\usepackage{tikz}
\usetikzlibrary{calc}
\usetikzlibrary{babel}
\usepackage{circuitikz}
% \usepackage{gnuplottex}
\usepackage{float}
\usepackage{ifthen}

\usepackage[top=1.3 in, bottom=1.3 in, left=1.3 in, right=1.3 in]{geometry} % Yeah, that's bad to play with margins
\usepackage[]{pdfpages}
\usepackage[]{attachfile}

\usepackage{amsmath}
\usepackage{enumitem}
\setlist[enumerate]{label=\alph*)}% If you want only the x-th level to use this format, use '[enumerate,x]'
\usepackage{multirow}

\newcommand\version{v1.0.0}

%cyr
\newcommand\textcyr[1]{{\fontencoding{OT2}\fontfamily{wncyr}\selectfont #1}}

%Numero du TP :
\def \tpnumber {TP 3 }

\newboolean{corrige}
\ifx\correction\undefined
\setboolean{corrige}{false}% pas de corrigé
\else
\setboolean{corrige}{true}%corrigé
\fi

%\setboolean{corrige}{false}% pas de corrigé

\newboolean{annexes}
\setboolean{annexes}{true}%annexes
%\setboolean{annexes}{false}% pas de annexes

\definecolor{darkblue}{rgb}{0,0,0.5}

\newboolean{mos}
%\setboolean{mos}{true}%annexes
\setboolean{mos}{false}% pas de annexes

\usepackage{aeguill} %guillemets

%% fancy header & foot
\pagestyle{fancy}
\lhead{[ELEC-H-305] Circuits logiques et numériques\\ \tpnumber}
\rhead{\version\\ page \thepage}
\chead{\ifthenelse{\boolean{corrige}}{Corrigé}{}}
\cfoot{}
%%

\pdfinfo{
/Author (Quentin Delhaye, ULB -- BEAMS)
/Title (\tpnumber, ELEC-H-305)
/ModDate (D:\pdfdate)
}

\hypersetup{
pdftitle={\tpnumber [ELEC-H-305] Choucroute logique et numérique},
pdfauthor={Quentin Delhaye, ULB -- BEAMS  },
pdfsubject={}
}

\theoremstyle{definition}% questions pas en italique
\newtheorem{Q}{Question}[] % numéroter les questions [section] ou non []

\newcommand{\reponse}[1]{% pour intégrer une réponse : \reponse{texte} : sera inclus si \boolean{corrige}
	\ifthenelse {\boolean{corrige}} {\paragraph{Réponse :} \color{darkblue}   #1\color{black}} {}
 }

\newcommand{\addcontentslinenono}[4]{\addtocontents{#1}{\protect\contentsline{#2}{#3}{#4}{}}}

\date{}
\title{\vspace{-2cm}Circuits logiques et numériques [ELEC-H-305] \\  \tpnumber \ifthenelse{\boolean{corrige}}{-- Corrigé}{} \\ \small{\version}}

\setlength{\parskip}{0.2cm plus2mm minus1mm} %espacement entre §
\setlength{\parindent}{0pt}

\begin{document}
% \pagestyle{empty}
\maketitle
\vspace*{-1cm}



\textit{Note : On adopte la convention de notation suivante~: $\overline{ab} = \overline{a} \cdot \overline{b}$ et $\overline{(ab)} = \overline{a} + \overline{b}$.}

\begin{Q}
Par comparaison des tables de vérité, prouver les égalités suivantes.

\begin{enumerate}
	\item $\overline{a}c + \overline{abc} = \overline{ab} + \overline{a}c$
	\reponse{~\\

		\renewcommand{\arraystretch}{1.3} %Allonge légèrement la hauteur des lignes pour qu'on puisse voir les barres au-dessus des variables.
		\begin{tabular}{|c|c|c|c|c|c|c|c|c|} \hline
		$a$ & $b$ & $c$ & $\overline{a}c$ & $\overline{abc}$ & $F_1$ & $\overline{ab}$ & $\overline{a}c$ & $F_2$ \\ \hline
		0 & 0 & 0 & 0 & 1 & 1 & 1 & 0 & 1 \\ \hline
		0 & 0 & 1 & 1 & 0 & 1 & 1 & 1 & 1 \\ \hline
		0 & 1 & 0 & 0 & 0 & 0 & 0 & 0 & 0 \\ \hline
		0 & 1 & 1 & 1 & 0 & 1 & 0 & 1 & 1 \\ \hline
		1 & 0 & 0 & 0 & 0 & 0 & 0 & 0 & 0 \\ \hline
		1 & 1 & 0 & 0 & 0 & 0 & 0 & 0 & 0 \\ \hline
		1 & 1 & 1 & 0 & 0 & 0 & 0 & 0 & 0 \\ \hline
		\end{tabular}
	}
	
	\item $ac + \overline{a}b + b\overline{c} = ac + b$
	\reponse{~\\

		\renewcommand{\arraystretch}{1.3} %Allonge légèrement la hauteur des lignes pour qu'on puisse voir les barres au-dessus des variables.
		\begin{tabular}{|c|c|c|c|c|c|c|c|c|c|} \hline
		$a$ & $b$ & $c$ & $ac$ & $\overline{a}b$ & $b\overline{c}$ & $F_1$ & $ac$ & $b$ & $F_2$ \\ \hline
		0 & 0 & 0 & 0 & 0 & 0 & 0 & 0 & 0 & 0 \\ \hline
		0 & 0 & 1 & 0 & 0 & 0 & 0 & 0 & 0 & 0 \\ \hline
		0 & 1 & 0 & 0 & 1 & 1 & 1 & 0 & 1 & 1 \\ \hline
		0 & 1 & 1 & 0 & 1 & 0 & 1 & 0 & 1 & 1 \\ \hline
		1 & 0 & 0 & 0 & 0 & 0 & 0 & 0 & 0 & 0 \\ \hline
		1 & 0 & 1 & 1 & 0 & 0 & 1 & 1 & 0 & 1 \\ \hline
		1 & 1 & 0 & 0 & 0 & 1 & 1 & 0 & 1 & 1 \\ \hline
		1 & 1 & 1 & 1 & 0 & 0 & 1 & 1 & 1 & 1 \\ \hline
		\end{tabular}
	}

	\item $\overline{( (\overline{a}+\overline{b}) (ab+\overline{c}) )} = ab + c$
	\reponse{~\\

		\renewcommand{\arraystretch}{1.3} %Allonge légèrement la hauteur des lignes pour qu'on puisse voir les barres au-dessus des variables.
		\begin{tabular}{|c|c|c|c|c|c|c|c|c|c|} \hline
		$a$ & $b$ & $c$ & $\overline{a} + \overline{b}$ & $ab + \overline{c}$ & $(\overline{a}+\overline{b}) (ab+\overline{c})$ & $F_1$ & $ab$ & $c$ & $F_2$ \\ \hline
		0 & 0 & 0 & 1 & 1 & 1 & 0 & 0 & 0 & 0 \\ \hline
		0 & 0 & 1 & 1 & 0 & 0 & 1 & 0 & 1 & 1 \\ \hline
		0 & 1 & 0 & 1 & 1 & 1 & 0 & 0 & 0 & 0 \\ \hline
		0 & 1 & 1 & 1 & 0 & 0 & 1 & 0 & 1 & 1 \\ \hline
		1 & 0 & 0 & 1 & 1 & 1 & 0 & 0 & 0 & 0 \\ \hline
		1 & 0 & 1 & 1 & 0 & 0 & 1 & 0 & 1 & 1 \\ \hline
		1 & 1 & 0 & 0 & 1 & 0 & 1 & 1 & 0 & 1 \\ \hline
		1 & 1 & 1 & 0 & 1 & 0 & 1 & 1 & 1 & 1 \\ \hline
		\end{tabular}
	}
\end{enumerate}
\end{Q}





\begin{Q}
Simplifier les expressions suivantes par manipulations algébriques.

\begin{enumerate}
	\item $(a+b)(a+\overline{b})$
	\reponse{
		\begin{align*}
			x + x & = (x + x) \cdot 1&\mbox{, A2b}\\
			& = (x + x) \cdot (x + \overline{x})&\mbox{, A5a}\\
			& = x + x \cdot \overline{x}&\mbox{, A3b}\\
			& = x + 0 & \mbox{, A5b}\\
			& = x &\mbox{, A2a}
		\end{align*}
	}
	\item $a+\overline{a}b$
	\item $\overline{ab}c + \overline{abc} + \overline{a}b\overline{c}$
	\item $\overline{( (a+b) \overline{cd} + e + \overline{f} )}$
	\item $\overline{a}bc + a\overline{bc} + \overline{abc} + a\overline{b}c + abc$
	\item $\overline{( ab + ac )} + \overline{ab}c$
	\item $\overline{( a+b )}\overline{( \overline{a} + b )}$
	\item $a + \overline{a}b + \overline{ab}$
\end{enumerate}
\end{Q}


\end{document}
