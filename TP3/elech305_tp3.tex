\documentclass[11pt,a4paper]{article}
\usepackage[utf8]{inputenc}
\usepackage[T1]{fontenc}
\usepackage{amsthm} %numéroter les questions
\usepackage[frenchb]{babel}
\usepackage{datetime}
\usepackage{xspace} % typographie IN
\usepackage[hidelinks]{hyperref}% hyperliens
\usepackage[all]{hypcap} %lien pointe en haut des figures
\usepackage[french]{varioref} %voir x p y
\usepackage{fancyhdr}% en têtes
%\input cyracc.def
\usepackage{graphicx} %include pictures
\usepackage{pgfplots}

\usepackage{tikz}
\usetikzlibrary{calc}
\usetikzlibrary{babel}
\usepackage{circuitikz}
% \usepackage{gnuplottex}
\usepackage{float}
\usepackage{ifthen}

\usepackage[top=1.3 in, bottom=1.3 in, left=1.3 in, right=1.3 in]{geometry} % Yeah, that's bad to play with margins
\usepackage[]{pdfpages}
\usepackage[]{attachfile}

\usepackage{amsmath}
\usepackage{enumitem}
\setlist[enumerate]{label=\alph*)}% If you want only the x-th level to use this format, use '[enumerate,x]'
\usepackage{multirow}

\newcommand\version{v1.1.1}

%cyr
\newcommand\textcyr[1]{{\fontencoding{OT2}\fontfamily{wncyr}\selectfont #1}}

%Numero du TP :
\def \tpnumber {TP 3 }

\newboolean{corrige}
\ifx\correction\undefined
\setboolean{corrige}{false}% pas de corrigé
\else
\setboolean{corrige}{true}%corrigé
\fi

%\setboolean{corrige}{false}% pas de corrigé

\newboolean{annexes}
\setboolean{annexes}{true}%annexes
%\setboolean{annexes}{false}% pas de annexes

\definecolor{darkblue}{rgb}{0,0,0.5}

\newboolean{mos}
%\setboolean{mos}{true}%annexes
\setboolean{mos}{false}% pas de annexes

\usepackage{aeguill} %guillemets

%% fancy header & foot
\pagestyle{fancy}
\lhead{[ELEC-H-305] Circuits logiques et numériques\\ \tpnumber}
\rhead{\version\\ page \thepage}
\chead{\ifthenelse{\boolean{corrige}}{Corrigé}{}}
\cfoot{}
%%

\pdfinfo{
/Author (Quentin Delhaye, ULB -- BEAMS)
/Title (\tpnumber, ELEC-H-305)
/ModDate (D:\pdfdate)
}

\hypersetup{
pdftitle={\tpnumber [ELEC-H-305] Choucroute logique et numérique},
pdfauthor={Quentin Delhaye, ULB -- BEAMS  },
pdfsubject={}
}

\theoremstyle{definition}% questions pas en italique
\newtheorem{Q}{Question}[] % numéroter les questions [section] ou non []

\newcommand{\reponse}[1]{% pour intégrer une réponse : \reponse{texte} : sera inclus si \boolean{corrige}
	\ifthenelse {\boolean{corrige}} {\paragraph{Réponse :} \color{darkblue}   #1\color{black}} {}
 }

\newcommand{\addcontentslinenono}[4]{\addtocontents{#1}{\protect\contentsline{#2}{#3}{#4}{}}}

\date{}
\title{\vspace{-2cm}Circuits logiques et numériques [ELEC-H-305] \\  \tpnumber \ifthenelse{\boolean{corrige}}{-- Corrigé}{} \\ \small{\version}}

\setlength{\parskip}{0.2cm plus2mm minus1mm} %espacement entre §
\setlength{\parindent}{0pt}

\begin{document}
% \pagestyle{empty}
\maketitle
\vspace*{-1cm}



\textit{Note : On adopte la convention de notation suivante~: $\overline{ab} = \overline{a} \cdot \overline{b}$ et $\overline{(ab)} = \overline{a} + \overline{b}$.}

\begin{Q}
Par comparaison des tables de vérité, prouver les égalités suivantes.

\begin{enumerate}
	\item $\overline{a}c + \overline{abc} = \overline{ab} + \overline{a}c$
	\reponse{~\\

		\renewcommand{\arraystretch}{1.3} %Allonge légèrement la hauteur des lignes pour qu'on puisse voir les barres au-dessus des variables.
		\begin{tabular}{|c|c|c|c|c|c|c|c|c|} \hline
		$a$ & $b$ & $c$ & $\overline{a}c$ & $\overline{abc}$ & $F_1$ & $\overline{ab}$ & $\overline{a}c$ & $F_2$ \\ \hline
		0 & 0 & 0 & 0 & 1 & 1 & 1 & 0 & 1 \\ \hline
		0 & 0 & 1 & 1 & 0 & 1 & 1 & 1 & 1 \\ \hline
		0 & 1 & 0 & 0 & 0 & 0 & 0 & 0 & 0 \\ \hline
		0 & 1 & 1 & 1 & 0 & 1 & 0 & 1 & 1 \\ \hline
		1 & 0 & 0 & 0 & 0 & 0 & 0 & 0 & 0 \\ \hline
		1 & 1 & 0 & 0 & 0 & 0 & 0 & 0 & 0 \\ \hline
		1 & 1 & 1 & 0 & 0 & 0 & 0 & 0 & 0 \\ \hline
		\end{tabular}
	}
	
	\item $ac + \overline{a}b + b\overline{c} = ac + b$
	\reponse{~\\

		\renewcommand{\arraystretch}{1.3} %Allonge légèrement la hauteur des lignes pour qu'on puisse voir les barres au-dessus des variables.
		\begin{tabular}{|c|c|c|c|c|c|c|c|c|c|} \hline
		$a$ & $b$ & $c$ & $ac$ & $\overline{a}b$ & $b\overline{c}$ & $F_1$ & $ac$ & $b$ & $F_2$ \\ \hline
		0 & 0 & 0 & 0 & 0 & 0 & 0 & 0 & 0 & 0 \\ \hline
		0 & 0 & 1 & 0 & 0 & 0 & 0 & 0 & 0 & 0 \\ \hline
		0 & 1 & 0 & 0 & 1 & 1 & 1 & 0 & 1 & 1 \\ \hline
		0 & 1 & 1 & 0 & 1 & 0 & 1 & 0 & 1 & 1 \\ \hline
		1 & 0 & 0 & 0 & 0 & 0 & 0 & 0 & 0 & 0 \\ \hline
		1 & 0 & 1 & 1 & 0 & 0 & 1 & 1 & 0 & 1 \\ \hline
		1 & 1 & 0 & 0 & 0 & 1 & 1 & 0 & 1 & 1 \\ \hline
		1 & 1 & 1 & 1 & 0 & 0 & 1 & 1 & 1 & 1 \\ \hline
		\end{tabular}
	}

	\item $\overline{( (\overline{a}+\overline{b}) (ab+\overline{c}) )} = ab + c$
	\reponse{~\\

		\renewcommand{\arraystretch}{1.3} %Allonge légèrement la hauteur des lignes pour qu'on puisse voir les barres au-dessus des variables.
		\begin{tabular}{|c|c|c|c|c|c|c|c|c|c|} \hline
		$a$ & $b$ & $c$ & $\overline{a} + \overline{b}$ & $ab + \overline{c}$ & $(\overline{a}+\overline{b}) (ab+\overline{c})$ & $F_1$ & $ab$ & $c$ & $F_2$ \\ \hline
		0 & 0 & 0 & 1 & 1 & 1 & 0 & 0 & 0 & 0 \\ \hline
		0 & 0 & 1 & 1 & 0 & 0 & 1 & 0 & 1 & 1 \\ \hline
		0 & 1 & 0 & 1 & 1 & 1 & 0 & 0 & 0 & 0 \\ \hline
		0 & 1 & 1 & 1 & 0 & 0 & 1 & 0 & 1 & 1 \\ \hline
		1 & 0 & 0 & 1 & 1 & 1 & 0 & 0 & 0 & 0 \\ \hline
		1 & 0 & 1 & 1 & 0 & 0 & 1 & 0 & 1 & 1 \\ \hline
		1 & 1 & 0 & 0 & 1 & 0 & 1 & 1 & 0 & 1 \\ \hline
		1 & 1 & 1 & 0 & 1 & 0 & 1 & 1 & 1 & 1 \\ \hline
		\end{tabular}
	}
\end{enumerate}
\end{Q}





\begin{Q}
Simplifier les expressions suivantes par manipulations algébriques.

\begin{enumerate}
	\item $(a+b)(a+\overline{b})$
	\reponse{
		\begin{align*}
			(a+b)(a+\overline{b}) & = aa + a\overline{b} + ab + b\overline{b} & \mbox{, distribution} \\
			& = aa + a\overline{b} + ab + 0 & \mbox{, $x \cdot \overline{x}$ = 1} \\
			& = a \cdot (1 + \overline{b} + b) & \mbox{, factorisation} \\
			& = a & \mbox{, $1 + x = 1$ et $x \cdot 1 = x$}
		\end{align*}
	}

	\item $a+\overline{a}b$
	\reponse{
		\begin{align*}
			a + \overline{a}b & = a + ab + \overline{a}b & \mbox{, $x + xy = x$}\\
			& = a + b \cdot (a + \overline{a}) & \mbox{, factorisation}\\
			& = a + b & \mbox{, $x + \overline{x} = 1$}
		\end{align*}
	}

	\item $\overline{ab}c + \overline{abc} + \overline{a}b\overline{c}$
	\reponse{
		\begin{align*}
		\overline{ab}c + \overline{abc} + \overline{a}b\overline{c} & = \overline{ab} \cdot (c + \overline{c}) + \overline{a}b\overline{c} & \mbox{, factorisation}\\
		& = \overline{ab} + \overline{a}b\overline{c} & \mbox{, $x+\overline{x} = 1$} \\
		& = \overline{a} \cdot (\overline{b} + b\overline{c}) & \mbox{, factorisation} \\
		& = \overline{a} \cdot (\overline{b} + \overline{bc} + b\overline{c}) & \mbox{, $x + xy = x$} \\
		& = \overline{a} \cdot (\overline{b} + \overline{c}) & \mbox{, $xy + \overline{x}y = y$} \\
		& = \overline{ab} + \overline{ac}
		\end{align*}
	}

	\item $\overline{( (a+b) \overline{cd} + e + \overline{f} )}$
	\reponse{
		\begin{align*}
		\overline{( (a+b) \overline{cd} + e + \overline{f} )} & = \overline{( a+b )} + \overline{( \overline{cd} )} \cdot \overline{e} f & \mbox{, de Morgan} \\
		& = \overline{ab} + (c + d) \cdot \overline{e}f
		\end{align*}
	}

	\item $\overline{a}bc + a\overline{bc} + \overline{abc} + a\overline{b}c + abc$
	\reponse{
		\begin{align*}
		\overline{a}bc + a\overline{bc} + \overline{abc} + a\overline{b}c + abc & = bc \cdot (a+\overline{a}) + \overline{bc} \cdot (a+\overline{a}) + a\overline{b}c & \mbox{, factorisation} \\
		& = bc + \overline{bc} + a\overline{b}c & \mbox{, $x + \overline{x} = 1$}
		\end{align*}
	}

	\item $\overline{( ab + ac )} + \overline{ab}c$
	\reponse{
		\begin{align*}
		\overline{( ab + ac )} + \overline{ab}c & = (\overline{a} + \overline{b}) \cdot (\overline{a} + \overline{c}) + \overline{ab}c & \mbox{, de Morgan} \\
		& = \overline{a} + \overline{ac} + \overline{ab} + \overline{bc} + \overline{ab}c & \mbox{, distribution} \\
		& = \overline{a} \cdot (1 + \overline{c} + \overline{b} + \overline{b}c) + \overline{bc} & \mbox{, factorisation} \\
		& = \overline{a} + \overline{bc} & \mbox{, $1+x = 1$}
		\end{align*}
	}

	\item $\overline{(( a+b )}\overline{( \overline{a} + b ))}$
	\reponse{
		\begin{align*}
		\overline{( a+b )}\overline{( \overline{a} + b )} & = (\overline{ab}) + (ab) & \mbox{, de Morgan} \\
		& = \overline{b} \cdot (\overline{a} + a ) & \mbox{, factorisation} \\
		& = \overline{b} \cdot (1) & \mbox{, $x + \overline{x} = 1$} \\
		& = \overline{b} & \mbox{, $x \cdot 1 = x$}
		\end{align*}
	}

	\item $a + \overline{a}b + \overline{ab}$
	\reponse{
		\begin{align*}
		a + \overline{a}b + \overline{ab} & = a + \overline{a} \cdot (b + \overline{b}) & \mbox{, factorisation} \\
		& = a + \overline{a} & \mbox{, $x + \overline{x} = 1$} \\
		& = 1 & \mbox{, $x + \overline{x} = 1$}
		\end{align*}
	}

\end{enumerate}
\end{Q}



\begin{Q}
Produits de sommes (PdS) et sommes de produits (SdP).

\begin{enumerate}
	\item Si $F = \overline{a}b\overline{c} + \overline{ab}c$, calculer $F'$ sous forme de PdS et SdP.
	\reponse{
		\begin{itemize}
			\item SdP~:
				\begin{align*}
				F' & = \overline{( \overline{a}b\overline{c} + \overline{ab}c )} \\
				& = \overline{abc} + \overline{a}bc + a\overline{bc} + a\overline{b}c + ab\overline{c} + abc \\
				& = \overline{abc} + \overline{a}bc + a \\
				& = \overline{bc} + bc + a
				\end{align*}
			\item PdS~: Pour passer $F'$ de SdP (forme disjonctive) à PdS (forme conjonctive), on peut calculer $F''$ et appliquer le principe de dualité au résultat.
			Si $F' = \overline{bc} + bc + a$, alors
				\begin{align*}
				F'' & = \overline{( \overline{bc} + bc + a )} \\
				& = \overline{a} \cdot (b+c) \cdot (\overline{b} + \overline{c}) \\
				& = (\overline{a}b + \overline{a}c) \cdot (\overline{b} + \overline{c}) \\
				& = \overline{a}b\overline{b} + \overline{a}b\overline{c} + \overline{ab}c + \overline{a}c\overline{c} \\
				& = \overline{a}b\overline{c} + \overline{ab}c
				\end{align*}
			Et donc le dual~: $F' = (a + \overline{b} + c) \cdot (a + b + \overline{c})$.
		\end{itemize}
	}

	\item Transformer $(a+\overline{b}) \cdot (b+c)$ en SdP.
	\reponse{
		Il suffit d'appliquer la conjonction~:
		\begin{align*}
		(a+\overline{b}) \cdot (b+c) & = ab + ac + b\overline{b} + \overline{b}c \\
		& = ab + ac + \overline{b}c
		\end{align*}
	}

	\item Transformer $ab + bc + \overline{a}c$ en PdS.
	\reponse{
		\begin{align*}
		F' & = \overline{( ab + bc + \overline{a}c )} \\
		& = (\overline{a} + \overline{b}) \cdot (\overline{b} + \overline{c}) \cdot (a + \overline{c}) \\
		& = (\overline{ab} + \overline{ac} + \overline{bb} + \overline{bc}) \cdot (a + \overline{c}) \\
		& = a\overline{ab} + \overline{abc} + a\overline{ac} + \overline{acc} + a\overline{bb} + \overline{bbc} + a\overline{bc} + \overline{bcc} \\
		& = \overline{abc} + \overline{ac} + a\overline{b} + \overline{bc} + a\overline{bc} + \overline{bc} \\
		& = \overline{ac} + a\overline{b} + \overline{bc} \\
		\Leftrightarrow F & = (a+c) \cdot (\overline{a}+b) \cdot (b+c)
		\end{align*}
	}
\end{enumerate}
\end{Q}



\begin{Q}
Écrire les expressions suivantes sous forme de mintermes.
\begin{enumerate}
	\item $F(a,b,c,d)=a\overline{b}c + \overline{abc} + ab\overline{c}d$
	\reponse{
		Écrire une fonction logique sous forme de mintermes revient à écrire sa forme disjonctive (SdP) canonique, c'est-à-dire développer tous les termes de la fonctions pour que toutes les variables apparaissent dans chaque terme.
		\begin{multline*}
			a\overline{b}cd + a\overline{b}c\overline{d} + \overline{abc}d + \overline{abcd} + ab\overline{c}d
		\end{multline*}
	}
	\item $F(a,b,c,d) = ab + \overline{b}c + cd$
	\reponse{
		\begin{multline*}
			% & abcd + abc\overline{d} + ab\overline{c}d + ab\overline{cd} + a\overline{b}cd + a\overline{b}c\overline{d} + \\
			% & \overline{ab}cd + \overline{ab}c\overline{d} + abcd + a\overline{b}cd + \overline{a}bcd + \overline{ab}cd
			abcd + abc\overline{d} + ab\overline{c}d + ab\overline{cd} + a\overline{b}cd + a\overline{b}c\overline{d} + \overline{ab}cd + \overline{ab}c\overline{d} + \overline{a}bcd
		\end{multline*}
	}
	\item $F(a,b,c,d) = a + d$
	\reponse{
		\begin{multline*}
			abcd + abc\overline{d} + ab\overline{c}d + ab\overline{cd} + a\overline{b}cd + a\overline{b}c\overline{d} + a\overline{bc}d + a\overline{bcd} + \overline{a}bcd + \overline{a}b\overline{c}d + \overline{ab}cd + \overline{abc}d
		\end{multline*}
	}
\end{enumerate}
\end{Q}



\begin{Q}
À partir de la table de vérité, faire la liste des mintermes et des maxtermes de la fonction suivante~: $F(a,b,c,d) = ab + cd$.
\reponse{~\\
	\begin{tabular}{|c|c|c|c|c|} \hline
	$a$ & $b$ & $c$ & $d$ & $F$ \\ \hline
	0 & 0 & 0 & 0 & 0 \\ \hline
	0 & 0 & 0 & 1 & 0 \\ \hline
	0 & 0 & 1 & 0 & 0 \\ \hline
	0 & 0 & 1 & 1 & 1 \\ \hline
	0 & 1 & 0 & 0 & 0 \\ \hline
	0 & 1 & 0 & 1 & 0 \\ \hline
	0 & 1 & 1 & 0 & 0 \\ \hline
	0 & 1 & 1 & 1 & 1 \\ \hline
	1 & 0 & 0 & 0 & 0 \\ \hline
	1 & 0 & 0 & 1 & 0 \\ \hline
	1 & 0 & 1 & 0 & 0 \\ \hline
	1 & 0 & 1 & 1 & 1 \\ \hline
	1 & 1 & 0 & 0 & 1 \\ \hline
	1 & 1 & 0 & 1 & 1 \\ \hline
	1 & 1 & 1 & 0 & 1 \\ \hline
	1 & 1 & 1 & 1 & 1 \\ \hline
	\end{tabular}

	\begin{itemize}
		\item Mintermes (produit des variables tel que la fonction logique est égale à 1)~: $\overline{ab}cd$, $\overline{a}bcd$, $a\overline{b}cd$, $ab\overline{cd}$, $abc\overline{d}$, $ab\overline{c}d$, $abcd$.

		\item Maxtermes (somme des variables telle que la fonction logique est égale à 0)~:	$a + b + c + d$, $a + b + c + \overline{d}$, $a + b + \overline{c} + d$, $a + \overline{b} + c + d$, $a + \overline{b} + c + \overline{d}$, $a + \overline{b} + \overline{c} + d$, $\overline{a} + b + c + d$, $\overline{a} + b + c + \overline{d}$, $\overline{a} + b + \overline{c} + d$.
	\end{itemize}

}
\end{Q}

\end{document}
