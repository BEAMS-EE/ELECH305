\documentclass[11pt,a4paper]{article}
\usepackage[utf8]{inputenc}
\usepackage[T1]{fontenc}
\usepackage{amsthm} %numéroter les questions
\usepackage[frenchb]{babel}
\usepackage{datetime}
\usepackage{xspace} % typographie IN
\usepackage[hidelinks]{hyperref}% hyperliens
\usepackage[all]{hypcap} %lien pointe en haut des figures
\usepackage[french]{varioref} %voir x p y
\usepackage{fancyhdr}% en têtes
%\input cyracc.def
\usepackage{graphicx} %include pictures
\usepackage{pgfplots}

\usepackage{tikz}
\usetikzlibrary{calc,arrows,automata}
\usetikzlibrary{babel}
\usepackage{circuitikz}
% \usepackage{gnuplottex}
\usepackage{float}
\usepackage{ifthen}

\usepackage[top=1.3 in, bottom=1.3 in, left=1.3 in, right=1.3 in]{geometry} % Yeah, that's bad to play with margins
\usepackage[]{pdfpages}
\usepackage[]{attachfile}

\usepackage{amsmath}
\usepackage{amssymb} % checkmark
\usepackage{enumitem}
\setlist[enumerate]{label=\alph*)}% If you want only the x-th level to use this format, use '[enumerate,x]'
\usepackage[]{xcolor}
\usepackage{colortbl}
\usepackage{multirow}
\usepackage{bigdelim}%Braces in tabular

\usepackage{askmaps} % https://github.com/parastuffs/askmaps_custom

\newcommand\encircle[1]{%http://tex.stackexchange.com/questions/123924/indexed-letters-inside-circles
  \tikz[baseline=(X.base)] 
    \node (X) [draw, shape=circle, inner sep=0] {\strut #1};}


\newcommand\version{v1.0.0}

%cyr
\newcommand\textcyr[1]{{\fontencoding{OT2}\fontfamily{wncyr}\selectfont #1}}

%Numero du TP :
\def \tpnumber {TP 10 }

\newboolean{corrige}
\ifx\correction\undefined
\setboolean{corrige}{false}% pas de corrigé
\else
\setboolean{corrige}{true}%corrigé
\fi

%\setboolean{corrige}{false}% pas de corrigé

\newboolean{annexes}
\setboolean{annexes}{true}%annexes
%\setboolean{annexes}{false}% pas de annexes

\definecolor{darkblue}{rgb}{0,0,0.5}

\newboolean{mos}
%\setboolean{mos}{true}%annexes
\setboolean{mos}{false}% pas de annexes

\usepackage{aeguill} %guillemets

%% fancy header & foot
\pagestyle{fancy}
\lhead{[ELEC-H-305] Circuits logiques et numériques\\ \tpnumber}
\rhead{\version\\ page \thepage}
\chead{\ifthenelse{\boolean{corrige}}{Corrigé}{}}
\cfoot{}
%%

\pdfinfo{
/Author (Quentin Delhaye, ULB -- BEAMS)
/Title (\tpnumber, ELEC-H-305)
/ModDate (D:\pdfdate)
}

\hypersetup{
pdftitle={\tpnumber [ELEC-H-305] Choucroute logique et numérique},
pdfauthor={Quentin Delhaye, ULB -- BEAMS  },
pdfsubject={}
}

\theoremstyle{definition}% questions pas en italique
\newtheorem{Q}{Question}[] % numéroter les questions [section] ou non []

\newcommand{\reponse}[1]{% pour intégrer une réponse : \reponse{texte} : sera inclus si \boolean{corrige}
	\ifthenelse {\boolean{corrige}} {\paragraph{Réponse :} \color{darkblue}   #1\color{black}} {}
 }

\newcommand{\addcontentslinenono}[4]{\addtocontents{#1}{\protect\contentsline{#2}{#3}{#4}{}}}

\date{}
\title{\vspace{-2cm}Circuits logiques et numériques [ELEC-H-305] \\  \tpnumber Synthèse de circuits \ifthenelse{\boolean{corrige}}{-- Corrigé}{} \\ \small{\version}}

\setlength{\parskip}{0.2cm plus2mm minus1mm} %espacement entre §
\setlength{\parindent}{0pt}

\newlength{\gvs}% Gate Vertical Space
\gvs=6em
\newlength{\ghs}% Gate Horizontal Space
\ghs=10em

\begin{document}
% \pagestyle{empty}
\maketitle
\vspace*{-1cm}


\begin{Q}
Coder la table de Huffman suivant afin d'éviter les problèmes de course critique en ajoutant si nécessaire un nouvel état.
Calculer les fonctions internes $Z_1$ et $Z_2$.

\begin{center}
	\begin{tabular}{|c|c|c|c|c|c|c|}\hline
	& \multicolumn{4}{c|}{$ab$} & \multicolumn{2}{c|}{} \\ \hline
	  & 00 & 01 & 11 & 10 & $Z_1$ & $Z_2$ \\ \hline
	1 & \textbf{1} & \textbf{1} & 2 & 3 & 0 & 0 \\ \hline
	2 & \textbf{2} & 1 & \textbf{2} & \textbf{2} & 0 & 1 \\ \hline
	3 & 2 & 1 & 4 & \textbf{3} & 1 & 0 \\ \hline
	4 & 1 & \textbf{4} & \textbf{4} & \textbf{4} & 1 & 1 \\ \hline
	\end{tabular}
\end{center}

\reponse{
	En utilisant le codage $1 \rightarrow 00$, $2 \rightarrow 01$, $3 \rightarrow 11$ et $4 \rightarrow 10$.
	On obtient une table de Huffman avec des problèmes de course~:
	
	\begin{center}
		\begin{tabular}{|c|c|c|c|c|c|c|}\hline
		$Y_1Y_2$ & \multicolumn{4}{c|}{$ab$} & \multicolumn{2}{c|}{} \\ \hline
		$y_1y_2$ & 00 & 01 & 11 & 10 & $Z_1$ & $Z_2$ \\ \hline
		00 & \textbf{00} & \textbf{00} & 01 & {\color{red}11} & 0 & 0 \\ \hline
		01 & \textbf{01} & 00 & \textbf{01} & \textbf{01} & 0 & 1 \\ \hline
		11 & 01 & {\color{red}00} & 10 & \textbf{11} & 1 & 0 \\ \hline
		10 & 00 & \textbf{10} & \textbf{10} & \textbf{10} & 1 & 1 \\ \hline
		\end{tabular}
	\end{center}

	Un seul problème de course peut être résolu, mais un subsiste~:
	
	\begin{center}
		\begin{tabular}{|c|c|c|c|c|c|c|}\hline
		$Y_1Y_2$ & \multicolumn{4}{c|}{$ab$} & \multicolumn{2}{c|}{} \\ \hline
		$y_1y_2$ & 00 & 01 & 11 & 10 & $Z_1$ & $Z_2$ \\ \hline
		00 & \textbf{00} & \textbf{00} & 01 & {\color{red}11} & 0 & 0 \\ \hline
		01 & \textbf{01} & 00 & \textbf{01} & \textbf{01} & 0 & 1 \\ \hline
		11 & 01 & {\color{green}01} & 10 & \textbf{11} & 1 & 0 \\ \hline
		10 & 00 & \textbf{10} & \textbf{10} & \textbf{10} & 1 & 1 \\ \hline
		\end{tabular}
	\end{center}

	Pour résoudre le problème restant, il faut ajoutant une troisième variable d'état qui permettra d'ajouter des transitions avec des distances de Hamming de 1.

	\begin{center}
		\begin{tabular}{|c|c|c|c|c|c|c|}\hline
		$Y_1Y_2Y_3$ & \multicolumn{4}{c|}{$ab$} & \multicolumn{2}{c|}{} \\ \hline
		$y_1y_2y_3$ & 00 & 01 & 11 & 10 & $Z_1$ & $Z_2$ \\ \hline
		000 & \textbf{000} & \textbf{000} & 001 & {\color{green}100} & 0 & 0 \\ \hline
		001 & \textbf{001} & 000 & \textbf{001} & \textbf{001} & 0 & 1 \\ \hline
		011 & 001 & 001 & 010 & \textbf{011} & 1 & 0 \\ \hline
		010 & 000 & \textbf{010} & \textbf{010} & \textbf{010} & 1 & 1 \\ \hline
		100 & - & - & - & 101 & 0 & 0 \\ \hline
		101 & - & - & - & 111 & 0 & 0 \\ \hline
		111 & - & - & - & 011 & 0 & 0 \\ \hline
		110 & - & - & - & - & - & - \\ \hline
		\end{tabular}
	\end{center}

	On en déduit les K-maps suivantes~:
	\begin{center}
			\askmapv{$Y_1 = b\overline{y_1y_2y_3} + a\overline{y_2}$}{a b $y_1$ $y_2$ $y_3$}{}{0000000010000000--------11-0----}{}
	\end{center}

	\begin{center}
			\askmapv{$Y_2 = ay_3 + by_2 + y_1y_2\overline{y_3}$}{a b $y_1$ $y_2$ $y_3$}{}{0000001000110011--------01-1----}{}
	\end{center}

	\begin{center}
			\askmapv{$Y_3 = \overline{y_1}y_3 + \overline{b}y_2y_3 + by_1\overline{y_2} + a\overline{y_2}$}{a b $y_1$ $y_2$ $y_3$}{}{0101000101011100--------11-1----}{}
	\end{center}

	\begin{center}
			\askmapv{$Z_1 = \overline{a}y_2$}{a b $y_1$ $y_2$ $y_3$}{}{00--001-00110011--------00-0----}{}
	\end{center}

	\begin{center}
			\askmapv{$Z_2 = \overline{by_1}y_3 + b\overline{y_2}y_3 + y_2\overline{y_3}$}{a b $y_1$ $y_2$ $y_3$}{}{01--00100110-11---------00-0----}{}
	\end{center}

}
\end{Q}


\begin{Q}
Une machine de Mealy est représentée dans la table de Huffman suivante.
Quelle est l'expression logique de $Z$ ?

\begin{center}
	\begin{tabular}{|c|c|c|c|c|}\hline
	$Y_1Y_2/Z$ & \multicolumn{4}{c|}{$ab$} \\ \hline
	$y_1y_2$ & 00 & 01 & 11 & 10 \\ \hline
	00 & \textbf{00/1} & 01 & \textbf{00/1} & - \\ \hline
	01 & - & 11 & \textbf{01/1} & \textbf{01/0} \\ \hline
	11 & \textbf{11/0} & \textbf{11/0} & 01 & 01 \\ \hline
	10 & 00 & 11 & \textbf{10/1} & \textbf{10/1} \\ \hline
	\end{tabular}
\end{center}

\reponse{
	\begin{center}
			\askmapiv{$Y_1 = ay_1\overline{y_2} + bya_1\overline{y_2} + \overline{a}y_2$}{a b $y_1$ $y_2$}{}{0-010111-0100010}{}
	\end{center}

	\begin{center}
			\askmapiv{$Y_2 = y_2 + \overline{a}b$}{a b $y_1$ $y_2$}{}{0-011111-1010101}{}
	\end{center}

	\begin{center}
			\askmapiv{$Z = \overline{y_2}$}{a b $y_1$ $y_2$}{}{1-101--0-010111-}{}
	\end{center}
}
\end{Q}

\end{document}
