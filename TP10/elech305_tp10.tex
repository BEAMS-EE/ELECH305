\documentclass[11pt,a4paper]{article}
\usepackage[utf8]{inputenc}
\usepackage[T1]{fontenc}
\usepackage{amsthm} %numéroter les questions
\usepackage[frenchb]{babel}
\usepackage{datetime}
\usepackage{xspace} % typographie IN
\usepackage[hidelinks]{hyperref}% hyperliens
\usepackage[all]{hypcap} %lien pointe en haut des figures
\usepackage[french]{varioref} %voir x p y
\usepackage{fancyhdr}% en têtes
%\input cyracc.def
\usepackage{graphicx} %include pictures
\usepackage{pgfplots}

\usepackage{tikz}
\usetikzlibrary{calc,arrows,automata}
\usetikzlibrary{babel}
\usepackage{circuitikz}
% \usepackage{gnuplottex}
\usepackage{float}
\usepackage{subfig}
\usepackage{caption}
\usepackage{ifthen}

\usepackage[top=1.3 in, bottom=1.3 in, left=1.3 in, right=1.3 in]{geometry} % Yeah, that's bad to play with margins
\usepackage[]{pdfpages}
\usepackage[]{attachfile}

\usepackage{amsmath}
\usepackage{amssymb} % checkmark
\usepackage{enumitem}
\setlist[enumerate]{label=\alph*)}% If you want only the x-th level to use this format, use '[enumerate,x]'
\usepackage[]{xcolor}
\usepackage{colortbl}
\usepackage{multirow}
\usepackage{bigdelim}%Braces in tabular

\usepackage{askmaps} % https://github.com/parastuffs/askmaps_custom

\newcommand\encircle[1]{%http://tex.stackexchange.com/questions/123924/indexed-letters-inside-circles
  \tikz[baseline=(X.base)] 
    \node (X) [draw, shape=circle, inner sep=0] {\strut #1};}


\newcommand\version{v1.2.0}

%cyr
\newcommand\textcyr[1]{{\fontencoding{OT2}\fontfamily{wncyr}\selectfont #1}}

%Numero du TP :
\def \tpnumber {TP 10 }

\newboolean{corrige}
\ifx\correction\undefined
\setboolean{corrige}{false}% pas de corrigé
\else
\setboolean{corrige}{true}%corrigé
\fi

%\setboolean{corrige}{false}% pas de corrigé

\newboolean{annexes}
\setboolean{annexes}{true}%annexes
%\setboolean{annexes}{false}% pas de annexes

\definecolor{darkblue}{rgb}{0,0,0.5}

\newboolean{mos}
%\setboolean{mos}{true}%annexes
\setboolean{mos}{false}% pas de annexes

\usepackage{aeguill} %guillemets

%% fancy header & foot
\pagestyle{fancy}
\lhead{[ELEC-H-305] Circuits logiques et numériques\\ \tpnumber}
\rhead{\version\\ page \thepage}
\chead{\ifthenelse{\boolean{corrige}}{Corrigé}{}}
\cfoot{}
%%

\pdfinfo{
/Author (Quentin Delhaye, ULB -- BEAMS)
/Title (\tpnumber, ELEC-H-305)
/ModDate (D:\pdfdate)
}

\hypersetup{
pdftitle={\tpnumber [ELEC-H-305] Choucroute logique et numérique},
pdfauthor={Quentin Delhaye, ULB -- BEAMS  },
pdfsubject={}
}

\theoremstyle{definition}% questions pas en italique
\newtheorem{Q}{Question}[] % numéroter les questions [section] ou non []

\newcommand{\reponse}[1]{% pour intégrer une réponse : \reponse{texte} : sera inclus si \boolean{corrige}
	\ifthenelse {\boolean{corrige}} {\paragraph{Réponse :} \color{darkblue}   #1\color{black}} {}
 }

\newcommand{\addcontentslinenono}[4]{\addtocontents{#1}{\protect\contentsline{#2}{#3}{#4}{}}}

\date{}
\title{\vspace{-2cm}Circuits logiques et numériques [ELEC-H-305] \\  \tpnumber Synthèse de circuits \ifthenelse{\boolean{corrige}}{-- Corrigé}{} \\ \small{\version}}

\setlength{\parskip}{0.2cm plus2mm minus1mm} %espacement entre §
\setlength{\parindent}{0pt}

\newlength{\gvs}% Gate Vertical Space
\gvs=6em
\newlength{\ghs}% Gate Horizontal Space
\ghs=10em

\begin{document}
% \pagestyle{empty}
\maketitle
\vspace*{-1cm}


\begin{Q}
Coder la table de Huffman suivant afin d'éviter les problèmes de course critique en ajoutant si nécessaire un nouvel état.
Calculer les fonctions internes $Z_1$ et $Z_2$.

\begin{center}
	\begin{tabular}{|c|c|c|c|c|c|c|}\hline
	& \multicolumn{4}{c|}{$ab$} & \multicolumn{2}{c|}{} \\ \hline
	  & 00 & 01 & 11 & 10 & $Z_1$ & $Z_2$ \\ \hline
	1 & \textbf{1} & \textbf{1} & 2 & 3 & 0 & 0 \\ \hline
	2 & \textbf{2} & 1 & \textbf{2} & \textbf{2} & 0 & 1 \\ \hline
	3 & 2 & 1 & 4 & \textbf{3} & 1 & 0 \\ \hline
	4 & 1 & \textbf{4} & \textbf{4} & \textbf{4} & 1 & 1 \\ \hline
	\end{tabular}
\end{center}

\reponse{
	En utilisant le codage $1 \rightarrow 00$, $2 \rightarrow 01$, $3 \rightarrow 11$ et $4 \rightarrow 10$.
	On obtient une table de Huffman avec des problèmes de course~:
	
	\begin{center}
		\begin{tabular}{|c|c|c|c|c|c|c|}\hline
		$Y_1Y_2$ & \multicolumn{4}{c|}{$ab$} & \multicolumn{2}{c|}{} \\ \hline
		$y_1y_2$ & 00 & 01 & 11 & 10 & $Z_1$ & $Z_2$ \\ \hline
		00 & \textbf{00} & \textbf{00} & 01 & {\color{red}11} & 0 & 0 \\ \hline
		01 & \textbf{01} & 00 & \textbf{01} & \textbf{01} & 0 & 1 \\ \hline
		11 & 01 & {\color{red}00} & 10 & \textbf{11} & 1 & 0 \\ \hline
		10 & 00 & \textbf{10} & \textbf{10} & \textbf{10} & 1 & 1 \\ \hline
		\end{tabular}
	\end{center}

	Un seul problème de course peut être résolu, mais un subsiste~:
	
	\begin{center}
		\begin{tabular}{|c|c|c|c|c|c|c|}\hline
		$Y_1Y_2$ & \multicolumn{4}{c|}{$ab$} & \multicolumn{2}{c|}{} \\ \hline
		$y_1y_2$ & 00 & 01 & 11 & 10 & $Z_1$ & $Z_2$ \\ \hline
		00 & \textbf{00} & \textbf{00} & 01 & {\color{red}11} & 0 & 0 \\ \hline
		01 & \textbf{01} & 00 & \textbf{01} & \textbf{01} & 0 & 1 \\ \hline
		11 & 01 & {\color{green}01} & 10 & \textbf{11} & 1 & 0 \\ \hline
		10 & 00 & \textbf{10} & \textbf{10} & \textbf{10} & 1 & 1 \\ \hline
		\end{tabular}
	\end{center}

	Pour résoudre le problème restant, il faut ajoutant une troisième variable d'état qui permettra d'ajouter des transitions avec des distances de Hamming de 1.

	\begin{center}
		\begin{tabular}{|c|c|c|c|c|c|c|}\hline
		$Y_1Y_2Y_3$ & \multicolumn{4}{c|}{$ab$} & \multicolumn{2}{c|}{} \\ \hline
		$y_1y_2y_3$ & 00 & 01 & 11 & 10 & $Z_1$ & $Z_2$ \\ \hline
		000 & \textbf{000} & \textbf{000} & 001 & {\color{green}100} & 0 & 0 \\ \hline
		001 & \textbf{001} & 000 & \textbf{001} & \textbf{001} & 0 & 1 \\ \hline
		011 & 001 & 001 & 010 & \textbf{011} & 1 & 0 \\ \hline
		010 & 000 & \textbf{010} & \textbf{010} & \textbf{010} & 1 & 1 \\ \hline
		100 & - & - & - & 101 & 0 & 0 \\ \hline
		101 & - & - & - & 111 & 0 & 0 \\ \hline
		111 & - & - & - & 011 & 0 & 0 \\ \hline
		110 & - & - & - & - & - & - \\ \hline
		\end{tabular}
	\end{center}

	On en déduit les K-maps suivantes~:
	\begin{center}
			\askmapv{$Y_1 = b\overline{ay_2y_3} + y_1\overline{y_2}$}{$y_1$ a b $y_2$ $y_3$}{}{0000000010000000--------11-0----}{}
	\end{center}

	\begin{center}
			\askmapv{$Y_2 = y_1y_3 + by_2 + ay_2\overline{y_3}$}{$y_1$ a b $y_2$ $y_3$}{}{0000001000110011--------01-1----}{}
	\end{center}

	\begin{center}
			\askmapv{$Y_3 = \overline{a}y_3 + \overline{b}y_2y_3 + ba\overline{y_2} + y_1\overline{y_2}$}{$y_1$ a b $y_2$ $y_3$}{}{0101000101011100--------11-1----}{}
	\end{center}

	\begin{center}
			\askmapv{$Z_1 = \overline{y_1}y_2$}{a b $y_1$ $y_2$ $y_3$}{}{00--001-00110011--------00-0----}{}
	\end{center}

	\begin{center}
			\askmapv{$Z_2 = \overline{ba}y_3 + b\overline{y_2}y_3 + y_2\overline{y_3}$}{$y_1$ a b $y_2$ $y_3$}{}{01--00100110-11---------00-0----}{}
	\end{center}

}
\end{Q}


\begin{Q}
Une machine de Mealy est représentée dans la table de Huffman suivante.
Quelle est l'expression logique de $Z$ ?

\begin{center}
	\begin{tabular}{|c|c|c|c|c|}\hline
	$Y_1Y_2/Z$ & \multicolumn{4}{c|}{$ab$} \\ \hline
	$y_1y_2$ & 00 & 01 & 11 & 10 \\ \hline
	00 & \textbf{00/1} & 01 & \textbf{00/1} & - \\ \hline
	01 & - & 11 & \textbf{01/1} & \textbf{01/0} \\ \hline
	11 & \textbf{11/0} & \textbf{11/0} & 01 & 01 \\ \hline
	10 & 00 & 11 & \textbf{10/1} & \textbf{10/1} \\ \hline
	\end{tabular}
\end{center}

\reponse{
	\begin{center}
			\askmapiv{$Y_1 = ay_1\overline{y_2} + by_1\overline{y_2} + \overline{a}y_2$}{a b $y_1$ $y_2$}{}{0-010111-0100010}{}
	\end{center}

	\begin{center}
			\askmapiv{$Y_2 = y_2 + \overline{a}b$}{a b $y_1$ $y_2$}{}{0-011111-1010101}{}
	\end{center}

	\begin{center}
			\askmapiv{$Z = \overline{y_2} + ab$}{a b $y_1$ $y_2$}{}{1-1010-0-010111-}{}
	\end{center}
}
\end{Q}





\begin{Q}
	Si on permet ou non l'équivalence de lignes dont la sortie est différente, montrer les deux automates différents (équations logiques et logigrammes) auxquels on parviendrait dans le cas de la table non complètement réduite suivante~:

	\begin{center}

	\begin{tabular}{|l|l|l|l|l|l|l|}
	\hline
	$Y_1$ $Y_2$ & 00         & 01         & 11         & 10         & ab & Z \\ \hline
	1           & \textbf{1} & 2          & 3          & 4          &    & 0 \\ \cline{1-5} \cline{7-7}
	2           & -          & \textbf{2} & \textbf{2} & \textbf{2} &    & 1 \\ \cline{1-5} \cline{7-7}
	3           & 1          & 2          & \textbf{3} & -          &    & 1 \\ \cline{1-5} \cline{7-7}
	4           & \textbf{4} & -          & 2          & \textbf{4} &    & 1 \\ \cline{1-5} \cline{7-7}
	\end{tabular}

	\end{center}

	Comment appelle-t-on ces deux types de machines ?

\reponse{
	\begin{enumerate}
		\item Si on ne permet pas l'équivalence des états dont la sortie est différente, on est dans le cas d'une machine de Moore.

		La table d'équivalence correspondante est donc~:
		\begin{center}
			\begin{tabular}{cccc} \cline{1-2}
				\multicolumn{1}{|c|}{2}	& 						\multicolumn{1}{c|}{\cellcolor{red!25}x} 					& 																	& \\ \cline{1-3}
				\multicolumn{1}{|c|}{3}	& 						\multicolumn{1}{c|}{\cellcolor{red!25}x} 					& \multicolumn{1}{c|}{\cellcolor{green!25}OK}						& \\ \cline{1-4}
				\multicolumn{1}{|c|}{\multirow{2}{*}{4}} & 		\multicolumn{1}{c|}{\cellcolor{red!25}}						& \multicolumn{1}{c|}{\cellcolor{green!25}} 						& \multicolumn{1}{c|}{\cellcolor{red!25}2-3}\\
				\multicolumn{1}{|c|}{} & 						\multicolumn{1}{c|}{\multirow{-2}{*}{\cellcolor{red!25}x}} & \multicolumn{1}{c|}{\multirow{-2}{*}{\cellcolor{green!25}OK}} 	& \multicolumn{1}{c|}{\cellcolor{red!25}1-4} \\ \cline{1-4}
			 & 													\multicolumn{1}{|c|}{1} & 									  \multicolumn{1}{c|}{2} 											& \multicolumn{1}{c|}{3} \\ \cline{2-4}
			\end{tabular}
		\end{center}

		On peut choisir de fusionner les états 2 et 3, ou les états 2 et 4.
		Si on choisit de fusionner les états 2 et 3, on obtient la table de Huffman réduite suivante~:
		\begin{center}
			\begin{tabular}{|l|l|l|l|l|l|l|}
			\hline
			$Y_1$ $Y_2$ & 00         & 01         & 11         & 10         & ab & Z \\ \hline
			1           & \textbf{1} & 2          & 2          & 4          &    & 0 \\ \cline{1-5} \cline{7-7}
			2           & 1          & \textbf{2} & \textbf{2} & \textbf{2} &    & 1 \\ \cline{1-5} \cline{7-7}
			4           & \textbf{4} & -          & 2          & \textbf{4} &    & 1 \\ \cline{1-5} \cline{7-7}
			\end{tabular}
		\end{center}

		Il faut ensuite choisir un système de codage pour les états afin d'en déduire les fonctions logiques.
		Prenons $1$, $2$ et $4$ codés respectivement $00$, $01$ et $11$\footnote{Puisque le code $10$ n'est pas utilisé, cet état n'existe pas et toutes ses valeurs sont remplacées par des \textit{don't care}.}.
		On obtient alors la même table de Huffman de laquelle on peut déduire des k-maps et les fonctions logiques du système.

		\begin{center}
			\begin{tabular}{|l|l|l|l|l|l|l|}
			\hline
			$Y_1$ $Y_2$ & 00         & 01         & 11         & 10         & ab & Z \\ \hline
			00           & \textbf{00} & 01          & 01          & 11          &    & 0 \\ \cline{1-5} \cline{7-7}
			01           & 00          & \textbf{01} & \textbf{01} & \textbf{01} &    & 1 \\ \cline{1-5} \cline{7-7}
			11           & \textbf{11} & -          & 01          & \textbf{11} &    & 1 \\ \cline{1-5} \cline{7-7}
			10           & -           & -          & -           & -           &    & - \\ \cline{1-5} \cline{7-7}
			\end{tabular}

			\askmapiv{$Y_1 = y_1\overline{b} + \overline{y_2}a\overline{b}$}{a b $y_1$ $y_2$}{}{00-100--10-100-0}{%
			\color{red}\put(0.1,0.1){\dashbox{0.3}(0.8,1.8){}}%
			\color{red}\put(3.1,0.1){\dashbox{0.3}(0.8,1.8){}}%
			\color{blue}\put(3.2,3.2){\dashbox{0.1}(0.6,0.6){}}%
			\color{blue}\put(3.2,0.2){\dashbox{0.1}(0.6,0.6){}}%
			}

			\askmapiv{$Y_2 = y_1 + a + b$}{a b $y_1$ $y_2$}{}{00-111--11-111-1}{%
			\color{red}\put(0.1,0.1){\dashbox{0.3}(3.8,1.8){}}%
			\color{green}\put(1.1,0.2){\dashbox{0.2}(1.8,3.7){}}%
			\color{blue}\put(2.1,0.3){\dashbox{0.1}(1.7,3.5){}}%
			}

			\askmapiv{$Z = y_2$}{a b $y_1$ $y_2$}{}{0--1-1---1-1-1-1}{%
			\color{red}\put(0.1,1.1){\dashbox{0.1}(3.8,1.8){}}%
			}
		\end{center}

		\item Si on permet l'équivalence des états dont la sortie est différente, on est dans le cas d'une machine de Mealy.
		Cependant, une restriction reste toujours de mise~: la fusion d'états futurs stables dont la sortie est différente est interdite.
		Considérons par exemple les états $1$ et $4$. Pour l'entrée $ab = 00$, les deux états futurs (respectivement $1$ et $4$) sont stables et leur sortie vaut respectivement $0$ et $1$, empêchant la fusion de ces deux états.
		Par contre, si l'on considère les états $2$ et $3$ pour les entrées $ab = 11$, les deux états futurs sont stables, mais ayant la même sortie, la fusion est potentiellement possible.

		Forts de ces considérations, nous pouvons dresser la table d'équivalence suivante~:
		\begin{center}
			\begin{tabular}{cccc} \cline{1-2}
				\multicolumn{1}{|c|}{\multirow{2}{*}{2}}	&	\multicolumn{1}{c|}{\cellcolor{green!25}2-3} 					& 																	& \\
				\multicolumn{1}{|c|}{}						&	\multicolumn{1}{c|}{\cellcolor{green!25}2-4}					& 																	& \\ \cline{1-3}
				\multicolumn{1}{|c|}{3}	& 						\multicolumn{1}{c|}{\cellcolor{green!25}OK}						& \multicolumn{1}{c|}{\cellcolor{green!25} OK}						& \\ \cline{1-4}
				\multicolumn{1}{|c|}{\multirow{2}{*}{4}} & 		\multicolumn{1}{c|}{\cellcolor{red!25}}							& \multicolumn{1}{|c|}{\cellcolor{green!25}} 						& \multicolumn{1}{c|}{\cellcolor{red!25}2-3}\\
				\multicolumn{1}{|c|}{} & 						\multicolumn{1}{c|}{\multirow{-2}{*}{\cellcolor{red!25}x}} 	& \multicolumn{1}{|c|}{\multirow{-2}{*}{\cellcolor{green!25}OK}} 	& \multicolumn{1}{c|}{\cellcolor{red!25}1-4} \\ \cline{1-4}
			 & 													\multicolumn{1}{|c|}{1} & 										\multicolumn{1}{c|}{2} 											& \multicolumn{1}{c|}{3} \\ \cline{2-4}
			\end{tabular}
		\end{center}

		Fusionnons les états $1$ et $3$ dans l'état $1$, et les états $2$ et $4$ dans l'état $2$~:
		\begin{center}
		\begin{tabular}{|l|l|l|l|l|l|l|}
		\hline
		$Y$ & 00         & 01         & 11         & 10         & ab & Z \\ \hline
		1           & \textbf{1} & 2          & \textbf{1}          & 2          &    & ? \\ \cline{1-5} \cline{7-7}
		2           & \textbf{2}          & \textbf{2} & \textbf{2} & \textbf{2} &    & ? \\ \cline{1-5} \cline{7-7}
		\end{tabular}
		\end{center}

		La sortie dépendant maintenant de l'état ainsi que des entrées, définir une sortie pour l'état tout entier n'a plus de sens.
		Nous allons donc assigner une sortie pour chaque état futur.
		Les états futurs fusionnés comportant au moins un état stable récupèrent la valeur de la sortie de l'état stable en question\footnote{On voit ici pourquoi fusionner deux états futurs stables ayant des sorties différentes est interdit~; on ne saurait pas la sortie de quel état stable récupérer.}.
		Par exemple, lors de la fusion des états $1$ et $3$ pour les entrées $ab = 00$, nous avons fusionné un état futur «~$1$~» stable dont la sortie valait $1$ avec un état instable «~$1$~» dont la sortie valait $0$.
		La sortie de l'état futur fusionné vaut donc $1$.

		La sortie des états futurs résultant de la fusion d'états futurs instables est déterminée de sorte à éviter les aléas, comme nous l'avons déjà fait dans les séances précédentes.
		Par exemple, la transition de l'état $1$ vers l'état $2$ lorsque les entrées $ab$ passent de $11$ à $01$ conserve la même sortie~: $1$.
		La sortie de la transition est donc aussi fixée à $1$ afin d'éviter un \textit{glitch}. La table~\ref{tab:sortie-transition} reprend les règles à observer dans de pareilles situations.
		\begin{table}[H]
			\centering
			\begin{tabular}{ccc}
			$Z_p$ & $Z_f$ & $Z_t$ \\ \hline
			0 & 0 & 0 \\
			0 & 1 & - \\
			1 & 0 & - \\
			1 & 1 & 1 \\
			\end{tabular}
			\caption{Valeur de la sortie lors de la transition entre états, avec $Z_p$ la sortie de l'état présent, $Z_f$ la sortie de l'état futur et $Z_t$ la sortie de la transition.}
			\label{tab:sortie-transition}
		\end{table}


		On obtient alors la table de Huffman réduite suivante pour la machine de Mealy~:

		\begin{center}
		\begin{tabular}{|l|l|l|l|l|l}
		\hline
		$Y$ & 00         & 01         & 11         & 10         & \multicolumn{1}{l|}{ab} \\ \hline
		1           & \textbf{1/0} & 2/1          & \textbf{1/1}          & 2/1          &  \\ \cline{1-5}
		2           & \textbf{2/1}          & \textbf{2/1} & \textbf{2/1} & \textbf{2/1} &  \\ \cline{1-5}
		\end{tabular}
		\end{center}

		En codant l'état $1$ par $0$ et l'état $2$ par $1$, on peut obtenir les k-maps et fonctions logiques suivantes~:
		\begin{center}
		\askmapiii{$Y = y + \overline{a}b + a\overline{b}$}{a b y}{}{01111101}{%
			\color{red}\put(0.1,0.1){\dashbox{0.2}(3.8,0.8){}}%
			\color{blue}\put(1.1,0.2){\dashbox{0.1}(0.8,1.7){}}%
			\color{green}\put(3.1,0.2){\dashbox{0.1}(0.7,1.7){}}%
			}

		\askmapiii{$Z = y + a + b$}{a b y}{}{01111111}{%
			\color{red}\put(0.1,0.1){\dashbox{0.2}(3.8,0.8){}}%
			\color{blue}\put(1.1,0.2){\dashbox{0.1}(1.8,1.7){}}%
			\color{green}\put(2.1,0.3){\dashbox{0.1}(1.6,1.5){}}%
			}
		\end{center}
	\end{enumerate}
}
\end{Q}


















\begin{Q}
	En codant les états 1, 2, 3 et 4 par $y_1y_2$ = 00, 01, 11 et 10 respectivement, calculer les fonctions d'excitation des organes de mémoire pour l'automate suivant :


	\begin{center}
		\begin{tabular}{|l|l|l|l|l|l}
		\hline
		$Y_1$ $Y_2$ & 00         & 01         & 11         & 10         & \multicolumn{1}{l|}{ab} \\ \hline
		1           & \textbf{1} & \textbf{1} & 2          & -          &                         \\ \cline{1-5}
		2           & \textbf{2} & 3          & \textbf{2} & \textbf{2} &                         \\ \cline{1-5}
		3           & 4          & \textbf{3} & 2          & -          &                         \\ \cline{1-5}
		4           & \textbf{4} & 1          & 2          & \textbf{-} &                         \\ \cline{1-5}
		\end{tabular}
	\end{center}
	Comme organes de mémoire on considérera des flip-flops D puis des flip-flops SRc.

	\reponse{
		Commençons par coder la table de Huffman~:
		\begin{center}
			\begin{tabular}{|l|l|l|l|l|l}
			\hline
			$Y_1$ $Y_2$ & 00         & 01         & 11         & 10         & \multicolumn{1}{l|}{ab} \\ \hline
			00           & \textbf{00} & \textbf{00} & 01          & -          &                         \\ \cline{1-5}
			01           & \textbf{01} & 11          & \textbf{01} & \textbf{01} &                         \\ \cline{1-5}
			11           & 10          & \textbf{11} & 01          & -          &                         \\ \cline{1-5}
			10           & \textbf{10} & 00          & 01          & \textbf{-} &                         \\ \cline{1-5}
			\end{tabular}
		\end{center}

		À titre de rappel, les termes d'excitation sont repris dans la table~\ref{tab:fct-exc} et les tables d'excitation dans les tables~\ref{tab:exc-flip-flop}.

		\begin{table}[H]
		\centering
			\begin{tabular}{cccl}
			$Q$ & $Q^+$ & & \\ \hline
			0 & 0 & $\mu_0$ & Maintien à 0 \\
			0 & 1 & $\varepsilon$ & Enclenchement \\
			1 & 0 & $\delta$ & Désenclenchement \\
			1 & 1 & $\mu_1$ & Maintien à 1\\
			\end{tabular}
		\caption{Termes d'excitation, $Q$ étant l'état courant et $Q^+$ l'état futur.}
		\label{tab:fct-exc}
		\end{table}
		\begin{table}[ht]
			\center
			\subfloat[\label{tab:exc-D}Flip-flop D]{%
				\parbox{.3\linewidth}{% Embed the content of the subfloat into a parbox to make it wider. Otherwise, the width of the subfloat is set by the width of the table, and so is the caption.
				\center
					\begin{tabular}{c|c}
					& D \\ \hline
					$\mu_0$ & 0 \\
					$\varepsilon$ & 1 \\
					$\delta$ & 0 \\
					$\mu_1$ & 1
					\end{tabular}
				}
			}
			\subfloat[\label{tab:exc-SR}Flip-flop SR]{%
				\parbox{.3\linewidth}{%
					\center
					\begin{tabular}{c|cc}
					& S & R \\ \hline
					$\mu_0$ & 0 & - \\
					$\varepsilon$ & 1 & 0 \\
					$\delta$ & 0 & 1 \\
					$\mu_1$ & - & 0
					\end{tabular}
				}
			}
			\caption{Tables d'excitation des flip-flops (a) D et (b) SR.}
			\label{tab:exc-flip-flop}
		\end{table}

		La première étape consiste à dresser la table d'états à partir de la table de Huffman.
		Pour cela, on remplace les termes des états stables par leur terme de maintien correspondant~: $00$ devient $\mu_0\mu_0$, $01$ devient $\mu_0\mu_1$, etc.
		Quant aux termes de transition, ils dépendent des changements d'état. Par exemple, si l'on passe de l'état $00$ à l'état $01$, le premier terme de l'état futur est maintenu à zéro, tandis que le second passe de $0$ à $1$ (il est enclenché), résultant en une transition $\mu_0\varepsilon$.
		On obtient alors la table d'état suivante~:
		\begin{center}
			\begin{tabular}{|l|l|l|l|l|l} \hline
			$Q^+_1$ $Q^+_2$ & 00         & 01         & 11         & 10         & \multicolumn{1}{l|}{ab} \\ \hline
			00           & $\mu_0\mu_0$ & $\mu_0\mu_0$ & $\mu_0\varepsilon$ & - & \\ \cline{1-5}
			01           & $\mu_0\mu_1$ & $\varepsilon\mu_1$ & $\mu_0\mu_1$ & $\mu_0\mu_1$ & \\ \cline{1-5}
			11           & $\mu_1\delta$ & $\mu_1\mu_1$ & $\delta\mu_1$ & - & \\ \cline{1-5}
			10           & $\mu_1\mu_0$ & $\delta\mu_0$ & $\delta\varepsilon$ & - & \\ \cline{1-5}
			\end{tabular}
		\end{center}

		Étant donné que chaque état futur est composé de deux termes, deux flip-flops seront nécessaires, un pour chacune des deux «~sous-tables d'états~» suivantes~:
		\begin{center}
			\begin{tabular}{|l|l|l|l|l|l} \hline
			$Q^+_1$ & 00         & 01         & 11         & 10         & \multicolumn{1}{l|}{ab} \\ \hline
			00           & $\mu_0$ & $\mu_0$ & $\mu_0$ & - & \\ \cline{1-5}
			01           & $\mu_0$ & $\varepsilon$ & $\mu_0$ & $\mu_0$ & \\ \cline{1-5}
			11           & $\mu_1$ & $\mu_1$ & $\delta$ & - & \\ \cline{1-5}
			10           & $\mu_1$ & $\delta$ & $\delta$ & - & \\ \cline{1-5}
			\end{tabular}
			\begin{tabular}{|l|l|l|l|l|l} \hline
			$Q^+_2$ & 00         & 01         & 11         & 10         & \multicolumn{1}{l|}{ab} \\ \hline
			00           & $\mu_0$ & $\mu_0$ & $\varepsilon$ & - & \\ \cline{1-5}
			01           & $\mu_1$ & $\mu_1$ & $\mu_1$ & $\mu_1$ & \\ \cline{1-5}
			11           & $\delta$ & $\mu_1$ & $\mu_1$ & - & \\ \cline{1-5}
			10           & $\mu_0$ & $\mu_0$ & $\varepsilon$ & - & \\ \cline{1-5}
			\end{tabular}
		\end{center}


		\begin{enumerate}
			\item Considérons d'abord un flip-flop D.
			D'après la table d'excitation de l'organe table~\ref{tab:exc-D}, on peut établir les deux tables d'état des deux flip-flops $D_1$ et $D_2$~:
			\begin{center}
				\begin{tabular}{|l|l|l|l|l|l} \hline
				$D_1$ & 00         & 01         & 11         & 10         & \multicolumn{1}{l|}{ab} \\ \hline
				00           & 0 & 0 & 0 & - & \\ \cline{1-5}
				01           & 0 & 1 & 0 & 0 & \\ \cline{1-5}
				11           & 1 & 1 & 0 & - & \\ \cline{1-5}
				10           & 1 & 0 & 0 & - & \\ \cline{1-5}
				\end{tabular}
				\begin{tabular}{|l|l|l|l|l|l} \hline
				$D_2$ & 00         & 01         & 11         & 10         & \multicolumn{1}{l|}{ab} \\ \hline
				00           & 0 & 0 & 1 & - & \\ \cline{1-5}
				01           & 1 & 1 & 1 & 1 & \\ \cline{1-5}
				11           & 0 & 1 & 1 & - & \\ \cline{1-5}
				10           & 0 & 0 & 1 & - & \\ \cline{1-5}
				\end{tabular}
			\end{center}

			Les k-maps et fonctions d'excitation sont maintenant immédiates~:
			\begin{center}
			\askmapiv{$D_1 = y_1\overline{a} + y_2\overline{a}b$}{a b $y_1$ $y_2$}{}{00110101-0--0000}{%
			\color{red}\put(0.1,0.1){\dashbox{0.1}(0.8,1.8){}}%
			\color{red}\put(3.1,0.1){\dashbox{0.1}(0.8,1.8){}}%
			\color{green}\put(1.1,1.1){\dashbox{0.2}(0.8,1.8){}}%
			}

			\askmapiv{$D_2 = y_2b + \overline{y_1}y_2 + a$}{a b $y_1$ $y_2$}{}{01000101-1--1111}{%
			\color{red}\put(1.1,1.1){\dashbox{0.1}(1.8,1.7){}}%
			\color{blue}\put(0.1,2.1){\dashbox{0.1}(3.8,0.8){}}%
			\color{green}\put(2.1,0.1){\dashbox{0.2}(1.8,3.8){}}%
			}
			\end{center}


			\item Passons ensuite au flip-flop SR.
			En utilisant les mêmes sous-tables et la table d'excitation~\ref{tab:exc-SR}, on peut obtenir de nouvelles tables d'état~:
			\begin{center}
				\begin{tabular}{|l|l|l|l|l|l} \hline
				$S_1R_1$ & 00         & 01         & 11         & 10         & \multicolumn{1}{l|}{ab} \\ \hline
				00           & 0- & 0- & 0- & -- & \\ \cline{1-5}
				01           & 0- & 10 & 0- & 0- & \\ \cline{1-5}
				11           & -0 & -0 & 01 & -- & \\ \cline{1-5}
				10           & -0 & 01 & 01 & -- & \\ \cline{1-5}
				\end{tabular}
				\begin{tabular}{|l|l|l|l|l|l} \hline
				$S_2R_2$ & 00         & 01         & 11         & 10         & \multicolumn{1}{l|}{ab} \\ \hline
				00           & 0- & 0- & 10 & -- & \\ \cline{1-5}
				01           & -0 & -0 & -0 & -0 & \\ \cline{1-5}
				11           & 01 & -0 & -0 & -- & \\ \cline{1-5}
				10           & 0- & 0- & 10 & -- & \\ \cline{1-5}
				\end{tabular}
			\end{center}

			De ces tables, on peut déduire des tables de Karnaugh ainsi que des fonctions d'excitation.
			\begin{center}
			\askmapiv{$S_1 = y_2\overline{a}b$}{a b $y_1$ $y_2$}{}{00--010--0--0000}{%
			\color{red}\put(1.1,1.1){\dashbox{0.1}(0.8,1.8){}}%
			}

			\askmapiv{$R_1 = a + \overline{y_2}b$}{a b $y_1$ $y_2$}{}{--00-010------11}{%
			\color{red}\put(1.1,0.1){\dashbox{0.1}(1.8,0.8){}}%
			\color{red}\put(1.1,3.1){\dashbox{0.1}(1.8,0.8){}}%
			\color{blue}\put(2.1,0.2){\dashbox{0.2}(1.8,3.6){}}%
			}

			\askmapiv{$S_2 = a$}{a b $y_1$ $y_2$}{}{0-000-0-----1-1-}{%
			\color{green}\put(2.1,0.1){\dashbox{0.2}(1.8,3.8){}}%
			}

			\askmapiv{$R_2 = y_1 \overline{b}$}{a b $y_1$ $y_2$}{}{-0-1-0-0-0--0000}{%
			\color{red}\put(0.1,0.1){\dashbox{0.1}(0.8,1.8){}}%
			\color{red}\put(3.1,0.1){\dashbox{0.1}(0.8,1.8){}}%
			}
			\end{center}

		\end{enumerate}
	}

\end{Q}










\begin{Q}
Calculer les fonctions d'excitation des organes de mémoire JKc pour l'automate suivant :


\begin{center}
	\begin{tabular}{|l|l|l|l|l|l}
	\hline
	$Y_1$ $Y_2$ & 00         & 01         & 11         & 10         & \multicolumn{1}{l|}{ab} \\ \hline
	00           & \textbf{00} & \textbf{00} & \textbf{00} & 01          &                         \\ \cline{1-5}
	01           & 11          & 10          & 00          & \textbf{01} &                         \\ \cline{1-5}
	11           & \textbf{11} & 10          & \textbf{11} & \textbf{11} &                         \\ \cline{1-5}
	10           & 00          & \textbf{10} & 11          & 01          &                         \\ \cline{1-5}
	\end{tabular}
\end{center}


\reponse{
	Suivant la table d'excitation du FF JK~:
	\begin{center}
		\begin{tabular}{c|cc}
			& J & K \\ \hline
			$\mu_0$ & 0 & - \\
			$\varepsilon$ & 1 & - \\
			$\delta$ & - & 1 \\
			$\mu_1$ & - & 0
		\end{tabular}
	\end{center}

	On peut réécrire la table de Huffman~:
	\begin{center}
		\begin{tabular}{|c|c|c|c|c|}\hline
		$JK$ & \multicolumn{4}{c|}{$ab$} \\ \hline
		$y_1y_2$ & 00 & 01 & 11 & 10 \\ \hline
		00 & $\mu_0\mu_0$ & $\mu_0\mu_0$ & $\mu_0\mu_0$ & $\mu_0\varepsilon$ \\ \hline
		01 & $\delta\mu_1$ & $\epsilon\delta$ & $\mu_0\delta$ & $\mu_0\mu_1$ \\ \hline
		11 & $\mu_1\mu_1$ & $\mu_1\delta$ & $\mu_1\mu_1$ & $\mu_1\mu_1$ \\ \hline
		10 & $\delta\mu_0$ & $\mu_1\mu_0$ & $\mu_1\varepsilon$ & $\delta\varepsilon$ \\ \hline
		\end{tabular}
	\end{center}

	Table qu'on sépare en deux et dont on remplace les termes d'excitation~:
	\begin{center}
		\begin{tabular}{|c|c|c|c|c|}\hline
		$J_1K_1$ & \multicolumn{4}{c|}{$ab$} \\ \hline
		$y_1y_2$ & 00 & 01 & 11 & 10 \\ \hline
		00 & $0-$ & $0-$ & $0-$ & $0-$ \\ \hline
		01 & $-1$ & $1-$ & $0-$ & $0-$ \\ \hline
		11 & $-0$ & $-0$ & $-0$ & $-0$ \\ \hline
		10 & $-1$ & $-0$ & $-0$ & $-1$ \\ \hline
		\end{tabular}
	\end{center}

	\begin{center}
		\begin{tabular}{|c|c|c|c|c|}\hline
		$J_2K_2$ & \multicolumn{4}{c|}{$ab$} \\ \hline
		$y_1y_2$ & 00 & 01 & 11 & 10 \\ \hline
		00 & $0-$ & $0-$ & $0-$ & $1-$ \\ \hline
		01 & $-0$ & $-1$ & $-1$ & $-0$ \\ \hline
		11 & $-0$ & $-1$ & $-0$ & $-0$ \\ \hline
		10 & $0-$ & $0-$ & $1-$ & $1-$ \\ \hline
		\end{tabular}
	\end{center}

	De ces tables de Huffman, on peut déduire les K-maps et donc les équations logiques des deux modules JK.

	\begin{center}
		\askmapiv{$J_1 = \overline{a}y_2$}{a b $y_1$ $y_2$}{}{0---01--00--00--}{}
	\end{center}
	\begin{center}
		\askmapiv{$K_1 = \overline{y_1} + \overline{aby_2} + a\overline{b}y_2$}{a b $y_1$ $y_2$}{}{-110--00--01--00}{}
	\end{center}
	\begin{center}
		\askmapiv{$J_2 = a\overline{b} + ay_1$}{a b $y_1$ $y_2$}{}{0-0-0-0-1-1-0-1-}{}
	\end{center}
	\begin{center}
		\askmapiv{$K_2 = b\overline{y_1} + \overline{a}b$}{a b $y_1$ $y_2$}{}{-0-0-1-1-0-0-1-0}{}
	\end{center}


}


\end{Q}

\end{document}
