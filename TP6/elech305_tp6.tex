\documentclass[11pt,a4paper]{article}
\usepackage[utf8]{inputenc}
\usepackage[T1]{fontenc}
\usepackage{amsthm} %numéroter les questions
\usepackage[frenchb]{babel}
\usepackage{datetime}
\usepackage{xspace} % typographie IN
\usepackage[hidelinks]{hyperref}% hyperliens
\usepackage[all]{hypcap} %lien pointe en haut des figures
\usepackage[french]{varioref} %voir x p y
\usepackage{fancyhdr}% en têtes
%\input cyracc.def
\usepackage{graphicx} %include pictures
\usepackage{pgfplots}

\usepackage{tikz}
\usetikzlibrary{calc}
\usetikzlibrary{babel}
\usepackage{circuitikz}
% \usepackage{gnuplottex}
\usepackage{float}
\usepackage{ifthen}

\usepackage[top=1.3 in, bottom=1.3 in, left=1.3 in, right=1.3 in]{geometry} % Yeah, that's bad to play with margins
\usepackage[]{pdfpages}
\usepackage[]{attachfile}

\usepackage{amsmath}
\usepackage{amssymb} % checkmark
\usepackage{enumitem}
\setlist[enumerate]{label=\alph*)}% If you want only the x-th level to use this format, use '[enumerate,x]'
\usepackage{multirow}
\usepackage{bigdelim}%Braces in tabular

\usepackage{askmaps} % https://github.com/parastuffs/askmaps_custom

\newcommand\version{v1.2.0}

%cyr
\newcommand\textcyr[1]{{\fontencoding{OT2}\fontfamily{wncyr}\selectfont #1}}

%Numero du TP :
\def \tpnumber {TP 6 }

\newboolean{corrige}
\ifx\correction\undefined
\setboolean{corrige}{false}% pas de corrigé
\else
\setboolean{corrige}{true}%corrigé
\fi

%\setboolean{corrige}{false}% pas de corrigé

\newboolean{annexes}
\setboolean{annexes}{true}%annexes
%\setboolean{annexes}{false}% pas de annexes

\definecolor{darkblue}{rgb}{0,0,0.5}

\newboolean{mos}
%\setboolean{mos}{true}%annexes
\setboolean{mos}{false}% pas de annexes

\usepackage{aeguill} %guillemets

%% fancy header & foot
\pagestyle{fancy}
\lhead{[ELEC-H-305] Circuits logiques et numériques\\ \tpnumber}
\rhead{\version\\ page \thepage}
\chead{\ifthenelse{\boolean{corrige}}{Corrigé}{}}
\cfoot{}
%%

\pdfinfo{
/Author (Quentin Delhaye, ULB -- BEAMS)
/Title (\tpnumber, ELEC-H-305)
/ModDate (D:\pdfdate)
}

\hypersetup{
pdftitle={\tpnumber [ELEC-H-305] Choucroute logique et numérique},
pdfauthor={Quentin Delhaye, ULB -- BEAMS  },
pdfsubject={}
}

\theoremstyle{definition}% questions pas en italique
\newtheorem{Q}{Question}[] % numéroter les questions [section] ou non []

\newcommand{\reponse}[1]{% pour intégrer une réponse : \reponse{texte} : sera inclus si \boolean{corrige}
	\ifthenelse {\boolean{corrige}} {\paragraph{Réponse :} \color{darkblue}   #1\color{black}} {}
 }

\newcommand{\addcontentslinenono}[4]{\addtocontents{#1}{\protect\contentsline{#2}{#3}{#4}{}}}

\date{}
\title{\vspace{-2cm}Circuits logiques et numériques [ELEC-H-305] \\  \tpnumber Méthode de Quine-McCluskey \ifthenelse{\boolean{corrige}}{-- Corrigé}{} \\ \small{\version}}

\setlength{\parskip}{0.2cm plus2mm minus1mm} %espacement entre §
\setlength{\parindent}{0pt}

\begin{document}
% \pagestyle{empty}
\maketitle
\vspace*{-1cm}




\begin{Q}
Trouver les implicants premiers par la méthode de Quine - Mc Cluskey :

\textit{Note~: $\sum_m$ désigne une somme de mintermes, $\sum_d$ une somme de « don't care ».}

\begin{enumerate}
	\item $f(a,b,c,d)=\sum_m(0,4,5,8,12,13)$
	\reponse{
		Il convient d'abord de convertir les nombres décimaux dans leur forme binaire~:
		\begin{center}
			\begin{tabular}{c|cccc}
			 & a & b & c & d \\ \hline
			 0 & 0 & 0 & 0 & 0 \\
			 4 & 0 & 1 & 0 & 0 \\
			 5 & 0 & 1 & 0 & 1 \\
			 8 & 1 & 0 & 0 & 0 \\
			 12 & 1 & 1 & 0 & 0 \\
			 13 & 1 & 1 & 0 & 1 \\
			\end{tabular}
		\end{center}

		On peut ensuite établir les différents groupes et fusionner ceux-ci.
		La marque $\checkmark$ indique qu'une ligne a été regroupée avec une autre.
		Les nombres entre parenthèses indiquent quels nombres ont été fusionnés.

		\begin{minipage}{0.5\textwidth}
			\begin{tabular}{ccccccc}
			 & a & b & c & d & & \\
			 $G_0$ & 0 & 0 & 0 & 0 & 0 & $\checkmark$ \\ \hline
			 $G_1$ & 0 & 1 & 0 & 0 & 4 & $\checkmark$ \\
			 & 1 & 0 & 0 & 0 & 8 & $\checkmark$ \\ \hline
			 $G_2$ & 1 & 1 & 0 & 0 & 12 & $\checkmark$ \\
			 & 0 & 1 & 0 & 1 & 5 & $\checkmark$ \\ \hline
			 $G_3$ & 1 & 1 & 0 & 1 & 13 & $\checkmark$ \\
			\end{tabular}
		\end{minipage}%
		\begin{minipage}{0.5\textwidth}
			\begin{tabular}{ccccccc}
			 & a & b & c & d & & \\
			 $G_{0'}$ & 0 & -- & 0 & 0 & (0,4) & $\checkmark$ \\
			 & -- & 0 & 0 & 0 & (0,8) & $\checkmark$ \\ \hline
			 $G_{1'}$ & -- & 1 & 0 & 0 & (4,12) & $\checkmark$ \\
			 & 0 & 1 & 0 & -- & (4,5) & $\checkmark$ \\
			 & 1 & -- & 0 & 0 & (8,12) & $\checkmark$ \\ \hline
			 $G_{2'}$ & 1 & 1 & 0 & -- & (12,13) & $\checkmark$ \\
			 & -- & 1 & 0 & 1 & (5,13) & $\checkmark$ \\
			\end{tabular}
		\end{minipage}
		\begin{center}
			\begin{tabular}{cccccccc}
			$G_{0''}$ & -- & -- & 0 & 0 & (0,4|8,12) & Fusion & \rdelim\}{2}{10mm}[IP1] \\
			& -- & -- & 0 & 0 & (0,8|4,12) & Fusion \\ \hline
			$G_{1''}$ & -- & 1 & 0 & -- & (4,12|5,13) & Fusion & \rdelim\}{2}{10mm}[IP2] \\
			& -- & 1 & 0 & -- & (4,5|12,13) & Fusion \\
			\end{tabular}
		\end{center}

		Les implicants premiers sont donc $\overline{cd}$ (IP1) et $b\overline{c}$ (IP2).
	}

	\item $f(a,b,c,d)=\sum_m(2,4,8,9,10,11,13,15)$
	\reponse{
		$ad, a\overline{b}, \overline{b}c\overline{d}, \overline{a}b\overline{cd}$
	}

	\item $f(a,b,c,d) = \sum_m(0,1,2,7,8,9,10)$
	\reponse{
		$\overline{a}bcd, \overline{bc}, \overline{bd}$
	}

	\item $f(a,b,c,d) = \sum_m(0,2,5,6,7,8,10,12,13,14,15)$
	\reponse{
		$ab, bc, \overline{bd}, bd, a\overline{d}, c\overline{d}$
	}

	\item $f(a,b,c,d)=\sum_m(2,3,4,10,12,13)+\sum_d(11,14,15)$
	\reponse{
		$\overline{b}c, b\overline{c}\overline{d}, ac, ab$
	}

	\item $f(a,b,c,d,e)=\sum_m(0,2,4,5,6,9,10) + \sum_d(7,11,12,13,14,15)$
	\reponse{
		$c\overline{d}, ad, ac,b, \overline{ad}$
	}

	\item $f(a,b,c,d,e)=\sum_m(0,1,2,4,11,15,17,20,21,31)+\sum_d(5,6,16,18,22,27)$
	\reponse{
		$\overline{bd}, bde, \overline{be}$
	}

	\item $f(a,b,c,d,e)=\sum_m(0,13,18,26,28,29) + \sum_d(2,8,10,12,16,24)$
	\reponse{
		$\overline{ce}, b\overline{de}, bc\overline{d}$
	}

	\item $f(a,b,c,d,e)=\sum_m(0,2,5,7,8,9,10,11,13,23,26,27,29) + \sum_d(3,12,15,18,19,21,22,31)$
	\reponse{
		$\overline{abce}, \overline{a}b\overline{c}e, \overline{ace}, \overline{ac}d, \overline{bc}d, \overline{a}b\overline{d}, a\overline{b}d, \overline{c}d, ce, de$
	}

	\item $f(a,b,c,d,e,f)=\sum_m(16,28,53,60,63)$
	\reponse{
		$abcdef, bcd\overline{ef}, ab\overline{c}d\overline{e}f, \overline{a}b\overline{cdef}$
	}

	\item $f(a,b,c,d,e,f)=\sum_m(1,2,3,15,17,18,19,26,32,48,63)$
	\reponse{
		$abcedf, a\overline{cdef}, \overline{a}b\overline{d}e\overline{f}, \overline{ab}cdef, \overline{acde}, \overline{acd}f$
	}

	\item $f(a,b,c,d,e,f)=\sum_m(1,2,3,15,16,17,18,19,26,28,29,30,32,48,63)$
	\reponse{
		$\overline{ab}cdef, abcdef, a\overline{cdef},\overline{a}b\overline{d}e\overline{f}, \overline{a}bce\overline{f}, \overline{a}bcd\overline{e}, \overline{acd}f, \overline{acd}e, \overline{a}b\overline{cd}$
	}

	\item $f(a,b,c,d,e,f)=\sum_m(0,7,11,13,16,23,28,31) + \sum_d(1,2,17,19,25)$
	\reponse{
		$\overline{ab}c\overline{d}ef, \overline{ab}cd\overline{e}f, \overline{a}bcd\overline{ef}, \overline{abcdf}, \overline{a}b\overline{cd}f, \overline{a}b\overline{de}f, \overline{ac}def, \overline{a}b\overline{c}ef, \overline{a}bdef, \overline{acde}$
	}

	\item $f(a,b,c,d,e,f)=\sum_m(1,2,3,15,17,18,19,26,32,48,63) + \sum_d(16,28,29,30)$
	\reponse{
		$\overline{ab}cdef, abcdef, b\overline{cdef}, \overline{a}b\overline{d}e\overline{f}, \overline{a}bce\overline{f}, \overline{a}bcd\overline{e}, \overline{acd}f, \overline{acd}e, \overline{a}b\overline{cd}$
	}

	\item $f(a,b,c,d,e,f)=\sum_m(0,1,2,5,14,16,18,24,26,30) + \sum_d(3,13,28)$
	\reponse{
		$\overline{abce}f, \overline{a}b\overline{def}, \overline{ab}d\overline{e}f, \overline{a}b\overline{d}e\overline{f}, \overline{a}cde\overline{f}, \overline{abcd}, \overline{acdf}, \overline{a}b\overline{df}, \overline{a}bc\overline{f}$
	}

	\item $f(a,b,c,d,e,f)=\sum_m(7,8,9,13,17,41,45,57) + \sum_d(1,15,44,56)$
	\reponse{
		$\overline{ab}c\overline{de}, \overline{acde}f, \overline{ab}def, ac\overline{de}f, a\overline{b}cd\overline{e}, abc\overline{de}, \overline{b}c\overline{d}e$
	}

	\item $f(a,b,c,d,e,f,g)=\sum_m(28,39,52,65,102,103,120)$
	\reponse{
		$a\overline{bcdef}g, \overline{ab}cde\overline{fg}, \overline{a}bc\overline{d}e\overline{fg}, abcd\overline{efg}, b\overline{cd}efg, ab\overline{cd}ef$
	}
\end{enumerate}
\end{Q}














\begin{Q}
Trouver la fonction simplifiée :
\begin{enumerate}
	\item $f(a,b,c,d)=\sum_m(2,3,4,10,12,13)+\sum_d(11,14,15)$
	\reponse{
		Tableau de couverture avec les implicants premiers de l'exercice précédent :
		$i_1$ = $b\overline{cd}$, $i_2$ = $\overline{b}c$, $i_3$ = $ac$ et $i_4$ = $ab$.

		\begin{center}
			\begin{tabular}{ccccc}
			 & $i_1$ & $i_2$ & $i_3$ & $i_4$ \\ \hline
			 0010 & & $\checkmark$ & & \\ \hline
			 0011 & & $\checkmark$ & & \\ \hline
			 0100 & $\checkmark$ & & & \\ \hline
			 1010 & & $\checkmark$ & $\checkmark$ & \\ \hline
			 1100 & $\checkmark$ &  &  & $\checkmark$ \\ \hline
			 1101 & & & & $\checkmark$ \\ \hline
			 % 1011 &  & $\checkmark$ & $\checkmark$ & \\ \hline
			 % 1110 & & & $\checkmark$ & $\checkmark$ \\ \hline
			 % 1111 & & & $\checkmark$ & $\checkmark$ \\
			\end{tabular}
		\end{center}

		On peut en déduire l'équation de couverture suivante~:
		\[1 = i_2 \cdot i_1 \cdot (i_2 + i_3) \cdot (i_1 + i_4) \cdot i_4 = i_2i_1i_4\]
		En utilisant les axiomes et théorèmes suivants~:
		\begin{align*}
		& x \cdot x = x\\
		& x \cdot (x + y) = x\\
		& (x + y) \cdot (x + z) = x + y \cdot z
		\end{align*}

		On trouve ainsi $f(a,b,c,d)=\overline{b}c+b\overline{cd}+ab$
	}

	\item $f(a,b,c,d)=\sum_m(0,2,4,5,10,11,13,15)+\sum_d(6,8)$
	\reponse {
		$f(a,b,c,d)=\overline{a}b\overline{c}+\overline{b}\overline{d}+a\overline{b}c+abd$
	ou $f(a,b,c,d)=a\overline{b}c+b\overline{c}d+abd+\overline{ad}$
	ou $f(a,b,c,d)=\overline{a}b\overline{c} + a\overline{b}c + abd + \overline{a}\overline{d}$
	ou $f(a,b,c,d)=b\overline{c}d + a\overline{b}c + acd + \overline{a}\overline{d}$
	ou $f(a,b,c,d)=b\overline{c}d + acd + \overline{a}\overline{d} + \overline{b}\overline{d}$
	ou $f(a,b,c,d)=\overline{a}b\overline{c} + acd + abd + \overline{b}\overline{d}$
	ou $f(a,b,c,d)=\overline{a}b\overline{c} + b\overline{c}d + acd + \overline{b}\overline{d}$
	}

	\item $f(a,b,c,d)=\sum_m(1,5,6,9,11,12,14,15)+ \sum_d(2,3,4,13)$
	\reponse{
	En ne gardant que l'une des fonctions les plus réduites~:
	
		$f(a,b,c,d) = \overline{c}d + b\overline{d} + ad$
	}

	\item $f(a,b,c,d,e)=\sum_m(3,7,14,15,19,22,23,27,31)+ \sum_d(0,6,11,24,25,28,29,30)$
	\reponse{
		$f(a,b,c,d) = de + cd$
	}

\end{enumerate}

\end{Q}



\end{document}
